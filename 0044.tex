% \subsection{44 天使様の手をとって}
\subsection{牵起天使大人的手}

% 周達が住む地域から車で小一時間ほど離れた地域にある有名な神社に到着すると、やはりというか人数はテレビで見た時よりもかなり減っていたものの、人が途絶える事はなさそうだった。\\
从住的地方出发,开车将近一个小时,周一行到达了这座在这一片有名的神社。虽然如他们所料,比起在电视里人已经少了不少,但看来还是不至于到没人了的地步。\\

% 「大分人が減ってはいるけど、それでもやっぱり結構居るわねえ」
「虽然看起来人已经少了挺多了,不过还是剩下不少啊」

% 「そうだね」
「是呢」

% 「真昼ちゃん、はぐれないようにしてね。私達も気を付けるしスマホがあるから集まるのは簡単だと思うけど、それでもやっぱり一緒に参拝したいものね」
「小真昼,不要走丢了哦。虽然我们也会留意,你也有电话要汇合也不算麻烦,但还是想要一起去参拜呢」

% 「はい」\\
「好的」\\

% 着物に身を包んでいる真昼が一番動きにくいし、足も遅い。靴はブーツとはいえ、着物は歩幅が制限されるので歩みは他の人と比べれば遅いものだろう。
身着和服的真昼在一行里动起来最不方便,步速也是最慢的。虽说鞋子穿的是长靴,但穿着和服还是会限制步幅,因而走的比较慢也是自然。

% 人混みをかき分ける、ほどではないものの、やはり肩がぶつかりやすい程度には居るので、こちらも気を配ってやらなければならない。\\
虽然到不了得挤着人才能前进的程度,但还是经常会撞到肩,所以这边还是有留个心眼的必要的。\\

% 「じゃあ、行きましょうか」\\
「那就出发吧」\\

% 志保子の先導で人混みの中まずは手水舎に向かって手と口を清める事となったのだが、やはり真昼に視線が吸い寄せられている人間の多い事。\\
志保子领着一行扎进人群里,打算首先去趟水屋漱口、洗手,但果不其然真昼实在是吸引人眼球。\\

% 着物を着ている人間も少なからず居るし、着物を着てきた真昼がそう目立つ訳でもない、という事はなかった。
穿着和服来的人也不算少,按理就算穿着和服真昼也不会太过显眼——然而事实并非如此。

% そもそも何も着飾っていない制服姿ですら人目を引くのだ。清楚系な正統派美少女が和装していて目立たない筈がない。\\
不如说就算没有装饰只是穿着校服的样子就已经够吸引人了。清纯系正统美少女的和服姿态,要说不吸引眼球那是不可能的。\\

% 口を清めている仕草すら美しいので、視線を集めている。\\
就连漱口的动作都是那么美丽,吸引着旁人的视线。\\

% 「……どうかしましたか?」
「……怎么了吗?」

% 「何でも」\\
「没啥」\\

% 何か面白くない、とは思ったものの、それを口にする事はせず、周も両親達と同じように手と口を清めて前を歩く両親の後をついていく。\\
真是无聊——周这么想着,但并没有说出口,而是跟着父母一样洗完手漱完口,追了上去。

% 一応真昼に歩みを合わせてはいるのだが、やはり和装は普段着にでもしていないと裾さばきが難しいらしく、人混みのせいもあっていつもより遅々とした進みとなっていた。\\
虽然周也在放慢步子等真昼,但果然平常没有穿惯和服下摆的处理很是麻烦,加之人也多,真昼的步伐变得比平常还要慢。\\

% 「真昼、大丈夫か」
「真昼,还好吗」

% 「はい、これくらい……ひゃっ」\\
「嗯,现在还……咿呀」\\

% 他の参拝者に肩がぶつかって体勢を崩して転びかけているので、周が腕で押さえた。\\
被其他的参拜者撞到了肩,身体失去平衡的真昼眼看就要摔倒,周赶紧伸手过去接住。\\

% 「大丈夫じゃなさそうだな」
「看上去就不太妙呢」

% 「……すみません」
「……抱歉」

% 「ほら、手を貸せ」\\
「好啦,手伸过来」\\

% 流石に慣れない格好で歩かせているので、気遣わない訳にはいかない。
毕竟真昼现在是在以一副不习惯的姿势走着,很有必要照顾下她。

% 袖から覗く小さな掌に手を伸ばせば、真昼がこちらを見上げてくる。\\
周把手伸向衣袖里的那小小的手掌,真昼则仰起头看向自己。\\

% もしかして嫌だったのかと手を引っ込めようとすれば、慌てて掌を重ねてまたじっと周を見上げてくるので、周は訳が分からずに見つめ返してしまった。
周突然担心真昼会不会不喜欢这样,正打算把手收回来,真昼却慌忙把手放在周的手中仰头看向周,搞的周也迷糊了起来,看回了真昼。

% じ、と見ていたら、先に真昼が視線を逸らして周の掌をきゅっと握る。\\
两人这么盯了一会,先移开了视线的真昼握起了周的手。\\

% 何なんだと首を捻る間もなく流れに乗って賽銭箱の前までたどり着きそうだったので、周は繋いだ手の感触を確かに感じながら、小さな疑問を胸にしまいこんだ。\\
连让周发表疑问的空隙都没有,两人便乘势走到了赛钱箱面前,于是周只好一边清楚地感受着牵着的那只手传来的感触,一边把这小小的疑问埋在了心里。\\

% \\


% 「結構長く願ってたけど、何願ってたんだ」\\
「花了那么久,许了个什么愿啊」\\

% 参拝を終えて少し列から離れたところで、静かに祈っていた真昼に問いかけてみる。
趁参拜完稍稍离开队伍的时候,周向刚刚静静地许愿的真昼问道。

% これぞ見本といった風な美しい所作で参拝した真昼は、周の倍くらい瞳を閉じて手を合わせていた。その後の礼の優美さに気をとられかけていたが、彼女が何か願い事をしていると思い出して聞いたのだ。\\
刚刚以那般称得上是示范的美丽动作参拜的真昼,闭眼合掌的时间有周的两倍长。周那之后看真昼那优雅的动作看的入了迷,现在回过神来才想起来问她许了什么愿。\\

% 「無病息災ですかね」
「只是无病无灾啦」

% 「すごい無難なやつ」\\
「真是个平淡的家伙啊」\\

% 真昼らしいといえば、真昼らしい。
要说的话,倒也是真昼的风格。

% あまり本人は物欲やら金銭欲やら名誉欲等はないので何を願うかと思っていたので、予想の範疇内のもので拍子抜けしたと言えばよいのか。\\
想着这人既没物欲又没钱欲还没名欲还能许什么愿,结果一如所料,甚至让人感觉有点失望。\\

% 「それと」
「还有」

% 「それと?」
「什么?」

% 「……このまま、穏やかな日々をすごせますように、と」\\
「……想一直过着这样,平稳的日子」\\

% これもまた、真昼らしい願いだった。
这个愿望也很有真昼的风格。

% 刺激や変化をあまり好まない真昼が願いそうな事であり、平和と静穏を好む真昼ならではの願いだろう。\\
像是不大喜欢刺激和变化的真昼会许的愿望,而且也只有喜欢平稳和安宁的真昼会许出这种愿望吧。\\

% 「うちの母さんいたら穏やかでないけどな」
「要我妈在那可就不平稳咯」

% 「それはそれで楽しめるものですよ」\\
「那样也有那样的乐趣哦」\\

% そういうものなのか……とは思ったものの、本人が楽しそうなので口は挟まず、柔らかい表情の真昼の手を取る。
是这么回事么……周虽然怀疑,但看着真昼本人很高兴的样子,便闭上了嘴,以一副温柔的表情牵起了她的手。

% まだ人混みから完全に抜けきった訳でもないし、先にお参りを終え少し離れた位置で待っている両親のところにたどり着くまでに転ばれても困る。\\
毕竟人还不少,还不能完全放手,要是走去先参拜完了在稍远处等着的父母这段路摔着了也麻烦。\\

% そういった意味で手を繋いだのだが、真昼は小さく瞬きをして、少しだけ恥ずかしそうに瞳を伏せてから周の手を握り返した。\\
虽然周是抱着这样的想法牵起了手,但真昼却微微眨了眨眼,略带害羞地垂着眼握住了周的手。\\

% 「二人とも、こっちよー」\\
「你们两个,在这边哦~」\\

% 志保子の声は明るくハキハキとしていて分かりやすい。
志保子的声音十分明亮而富有活力,很容易分辨出来。

% 促されるように二人で両親のもとに向かえば、志保子が目を丸くして、それから口元に手を添えて微笑ましそうにこちらを眺めてくる。\\
像是被催着般,两人走向父母在的地方,这时志保子则瞪大了眼,然后用手捂着嘴似是在微笑般望着这边。\\

% 「あらあら」
「哎呀哎呀」

% 「何だよ」
「咋了啊」

% 「ナチュラルに手を繋いでいるのねえ、と」\\
「想着你俩自然地就牵起手来了呢」\\

% 言われて、志保子の前で手を繋ぐのは失策だと今更に気付く。
这么一说,周才反应过来自己牵着真昼的手走到志保子面前这一失策。

% これでは真昼が周の特別だ、と言っているようなものではないか。志保子に勘ぐられて常ににやにや笑いをされるなんて冗談ではない。\\
这岂不是在说,对周来说真昼是特别的存在了么。被志保子误会然后整天被这么笑着,这可一点也不好笑。

% 「……はぐれないようにするためだろ。それに着物だと転びやすいし」
「……是为了不让她走丢啦。而且穿着和服还很容易摔着」

% 「そうだね。着物だと歩きにくいし、エスコートしてあげるべきだろう。私も志保子さんにするし」\\
「是呢。穿着和服很难走路,确实需要一个护花使者呢。我可是保护着志保子呢」\\

% 修斗は理解あるので、真昼の手を取っている事に違和感はないらしい。同じようにするりと志保子の手を握っている。
修斗则是表示了理解,没有对周牵着真昼的手一事感到奇怪。他也和周一样,轻轻地牵着志保子的手。

% あのように父親ほどスマートに手を差しのべて繋ぐなんて出来れば苦労しないが、性格上無理だとも思っていたので、真昼が素直に手を繋いでくれたのはありがたかった。\\
要是能像父亲那样灵巧地伸出手牵起对方的话,那就没这么多累人事了,但周从性格上便做不到,因而真昼愿意接住自己的手周还是很感激的。\\

% 志保子の意識が修斗の方に移った事にほっとしつつ、そっと手を離そうとしたら、真昼の手から力が抜けてくれない。
看着志保子的意识转向了修斗,周松了口气,正想轻轻地松开手,可真昼却没有放松手上的力气。

% きゅっ、と控えめながら離す気がないのは分かるので、どうしたと小声で聞いても返事はない。ただ、か細い指が周を捕らえているだけだ。\\
虽然动作很轻,但周还是理解了真昼不愿松开手的意思,轻声问她「怎么了」却也没有得到回答——只是,用细细的手指抓着周不放。\\

% 「真昼ちゃん真昼ちゃん、温かい飲み物買おうと思うんだけど、おしること甘酒どっちがいい?」
「小真昼小真昼,我打算去买些热饮,汁粉和甘酒你要哪种?」
% 汁粉:一种日本的红豆沙甜品,一般放入麻糬等食用。
% 甘酒:又称醴,是一种甘甜的日本传统浊酒,酒精含量极低或不含酒精。

% 「じゃあ、おしるこでお願いします」\\
「那我就要汁粉吧」\\

% 志保子に遮られて問いかけるタイミングも離すタイミングも失って、そのまま華奢な手を握ったまま。\\
不管是避着志保子提问还是放开手的时机都已错过,周便只好继续握着那娇嫩的手。\\

% 「あなたはどうする?」
「你呢?」

% 「……じゃあ俺甘酒」
「……那我就要甘酒吧」

% 「はいはい」\\
「好好」\\

% ただ、真昼が嫌がっていないならそれでいいか、と胸の中で起こったかすかなざわつきを抑えて飲み込み、志保子に希望を伝えて真昼の手を握り直した。\\
不过,要是真昼不讨厌的话那这样也不错——周抚平那心中泛起的微微瘙痒感让自己接受了,然后告诉志保子自己要哪样,手则再次握紧了真昼。\\

% \\


% ほどなくして出店から帰って来た志保子がそれぞれの注文した品を手渡してくるので、流石にこれは手を離さないとどうしようもないので一度離して一息つく事になった。\\
不多久,从店里回来的志保子把买来的各种东西分了下来,再怎么说这时候除了放手也没别的办法了,于是两人便暂时松开了手稍做休息。\\

% 両親は共に甘酒を飲みながら穏やかに笑い合っている。
父母则一起喝着甜酒放松地笑着。

% 二人の世界というほどではないもののいちゃいちゃしているので、声をかける気にもならず周も渡された甘酒に口をつけた。\\
虽然不及二人世界,但两人还是亲亲热热了起来,不过周也懒得搭话,接过甘酒喝了起来。\\

% 飲む点滴と言われるほどに栄養のあるものだが、米の甘味やコクはほっと染み渡る味で、つい感嘆と安堵の混じった吐息がこぼれる。
虽然甘酒有着滋补胜过输液的传言,但令周享受的还是那米那沁人心脾的甘甜与回韵。一口下去,周不禁叹出一口夹带感叹和安心的吐息。

% おしるこも捨てがたかったのだが、やはり新年という事で気分的にこちらを選んだのだが、個人的には正解だった。\\
虽然汁粉也难以舍弃,但既然是新年,考虑到气氛,周便选择了这边,从个人喜好上看应该是选对了吧。\\

% ちらりと真昼を見てみると、穏やかな顔で紙コップに入ったおしるこを少しずつ飲んでいる。\\
周瞄了一眼真昼,她则是一副安稳的表情一点点地喝着纸杯里的汁粉。\\

% 「おしるこはうまいか?」
「汁粉好喝吗?」

% 「美味しいですよ」
「很好喝哦」

% 「一口頂戴」
「让我尝一口」

% 「どうぞ。私ももらっていいですか」
「请。我也能尝一口吗」

% 「ん」\\
「给」\\

% 折角なので一口味見で交換する事になったので、コップを交換してとろみのついたいかにも小豆といった色合いのそれに口をつける。\\
反正都要尝一口,干脆就直接换着来喝了,周换过了杯子,将那微粘而泛着鲜亮的红豆色的汤汁送到了嘴边。\\

% 柔らかく漂う小豆独特の香りをかぎながら口に含むと、やはり甘くて濃厚な味が広がる。いささか甘さが強いように思えるのは、周があまり甘党ではないからだろう。
嗅着那空气中飘着的红豆独特的香味送入嘴中,一如所料一股甘甜而浓厚的风味从舌尖扩散开来。觉得略微有些过甜,大概是因为周不是甜派的人吧。

% 真昼は甘いものはそれなりに好きらしいので、ちょうどよいのかもしれない。\\
真昼似乎是挺喜欢甜味的东西的,这个甜度对她来说应该是正好吧。\\

% 「おいしい」\\
「好喝」\\

% 真昼の方は甘酒も気に入ったらしく、かすかに目尻が下がった笑みが浮かんでいる。\\
真昼她则是挺中意甘酒似的,微微弯起眼角露出了笑容。\\

% 「……ナチュラルねえほんと」\\
「……还真是一副习惯的样子呢」\\

% 二人の様子を見守っていた志保子が、小さくこぼした。\\
守望着两人的志保子,小声地感叹道。\\

% 「何がだよ」
「咋了啊」

% 「気にしなくてもいいのよ。……今日が寒くてよかったわぁ」
「不用在意哦。……今天是个冷天挺幸运呢」

% 「暖かい方がいいだろ」
「明显是天气暖和好吧」

% 「二人はそうかもしれないけど、私達は……ねぇ?」\\
「你们俩说不定是那样,我们的话……是吧?」\\

% 志保子が同じく見守っていた修斗に同意を求めると、修斗は穏やかな笑み……微妙に苦笑混じりではあるが、柔らかく微笑んで「そうだね」と返す。\\
志保子向同样守望者两人的修斗寻求同意,修斗则以平和……而微妙地混有苦笑的笑容,微笑着「确实呢」这么答道。

% こちらを見る眼差しが妙に生暖かくて居心地悪げに肩を揺らした周を、真昼は不思議そうに眺めていた。
被那微妙的温暖视线看着的周略感不适耸了耸肩,而真昼则以不可思议的眼神望着那样的周。
