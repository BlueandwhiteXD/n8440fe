\subsection{选择礼物的方法}

周在学习上本就很勤奋,上课态度也很认真,所以并没有太费劲便通过了期末考试。\\

他和真昼一起复查试卷,算出来的分数也跟平时差不多,况且平常在学校里的态度也挺好,留级的事情是可以不用担心的了。

树考的分也还行,千岁似乎也避免了挂科,这么一来周熟悉的人就都不用担心留级的问题了。\\

然后就是送别和周没什么关系的高三学生的毕业典礼,再接着就是休业式了……不过在这之间还有一个问题很大的节日。\\

「……该回什么礼呢」\\

这正是,朝着情人节的胜者们逼近的回礼之日。

且先不论周这到底算不算是胜者,但既然从真昼和千岁那收了巧克力,那么自然要有回礼。\\

只不过,周还在头疼该送什么好。

千岁的话去那家买了圣诞蛋糕的店定一个白色情人节款的点心套装,然后再送个她正在收集的角色周边,那就应该问题不大了。\\

问题是真昼。

真昼的话,总觉得无论送什么她都会欣然收下。

周送的东西她都平常地收下了,而且看起来她比较在意心意而对送的东西并不是很关心。讲真这样反而让周最头疼。\\

就算想要选点她喜欢的东西,可周所了解的她的喜好也就只有甜的东西和可爱的东西这种女生差不多都喜欢的东西,因而周一直在头疼到底该送什么好。\\

再怎么说,上回说过的磨刀石这种不但一点意思也没有而且还爆预算的东西肯定是免谈。但即使不考虑那个,周还是打不定主意要选什么。如果可以的话,这回比起实用品来说,周更想要送享受用的东西。\\

周暂且先去了杂货店。他望着里面的白色情人节商品角,但看着这些,周却不太想象得出她真正高兴的样子。

可以的话,周想要送的礼物最好是能让真昼表现出上次收到熊布偶时的那种反应。\\

(不过送两次熊布偶就没意思了啊)\\

虽然可爱的熊布偶货架上倒是摆了一堆,但送两次一样的东西还是欠缺新鲜感吧。

但话说回来,以周那贫乏的想象力,能想到的女生喜欢的东西,除了小饰品以外也没别的了。\\

可两人的关系到底有没有到送小饰品的程度,周仍然摸不大准。\\

如果送的话,估计真昼还是会好好地收下,但问题是她到底会不会高兴。

虽然,周是觉得两人按男女说来算是关系不错了……但是,送小饰品到底能不能让她高兴呢。\\

这要是树给千岁的那肯定没有问题,但周给真昼送就得画一个问号了。\\

周这样一脸烦恼地在特卖角旁边晃来晃去,恐怕看上去就像一个可疑人物吧。

虽然周也换上了外出用装束,但还是改变不了男性在可爱的杂货前晃来晃去很可疑这一事实。\\

周正念叨着这也不行那也不行,突然有人从身后搭了一句「在找什么吗?」。\\

周一回头,便看见一位身着店里围裙的妙龄女性微笑着站在身后。

她大概是看不下去周这实在是苦恼的样子,所以过来帮忙的吧。要不然的话,她也不会特意向这跟个可疑人物一样晃来晃去的周搭话。\\

「啊,那个……白色情人节的回礼,我拿不定主意」

「在这边没有看上的东西吗?别的地方也有常被选去做回礼的商品,我带你去看看吧」

「啊,不是这个意思……只是说和她的关系有些不好形容,所以不知道该送什么不会被讨厌」

「怎么说?」

「她不算是女朋友但关系感觉挺熟的……打个比方吧,小饰品这种东西,从称不上是喜欢的人那儿收到,会不会感到高兴啊,这样」

因为解释起来很害羞,周的说明便有些含糊,但女店员听完后却笑了起来,恐怕是觉得周的烦恼比较逗人吧。\\

「男性烦恼这种东西是很常见的哦」

「那他们是怎么决定的?」

「大部分人虽然犹豫,但最后还是决定购买呢。如果关系亲近的话,就算送了一般也不会被讨厌的哦」\\

不会讨厌——听了这话,周稍稍安心了一点。即使如此,要送真昼饰品还是让他心里有些慌慌的。

真昼她虽然身上打理的很整洁,但并不怎么会佩戴饰品。虽然她偶尔会戴一戴,但那些看起来都是高级的东西。\\

周并没有自信自己选出的东西能达到美感优异的真昼的审美标准。\\

「有需要的话,我给你介绍几个那边在女性间很火的饰品吧」

「……拜托了」\\

听见这求之不得的提议,周不自觉地摆正了姿势点了点头。\\

\vspace{2\baselineskip}

「然后我就买了」\\

周跟树讲了整件事的经过后,树露出了前几天那个店员一样的眼神笑话了周。\\

本来两人是在食堂一角吃着每日套餐的,结果讲到白色情人节的话题,周不小心便把这事讲出来了。\\

「……别多嘴。不过啊,果然明明没有交往却送对方饰品显得有些恶心不是么」

「你怎么这么矫情啊,男子汉做事得靠勇敢和气势懂不。她的话反正只要是周送的收到什么都会开心的哦?」

「……虽然是这么回事啦」\\

以真昼的性格,不论送什么她都会平常地高兴收下吧。

但周的希望是送一件能让她真的高兴而且能用上的东西,因而还在担心这到底达不达得到要求。\\

「结果你买了个啥?」

「……粉金色的、以花为主题的手镯」\\

感觉对真昼来说,比起给人以冷淡感的银色和华贵感的金色,还是这华丽中泛着柔和与可爱感的粉金色比较适合。\\

身为学生,高价的贵金属是肯定买不起的,所以这里只是谈的外观——周自己觉得,在这种颜色的饰品中,他挑选出的是设计精致优美、适合真昼的一款。\\

「咋了,听上去不是挺能让她高兴的嘛」

「……不会被觉得恶心?」

「哎呀你多虑了吧。为什么你这种地方这么消极啊……」

「给女性送礼什么的我可是只给她好好送过啊」\\

首先母亲肯定是成不了这样的对象,而千岁也不算数。不如说给千岁的东西是她自己闹着要的甜品,周甚至不怎么觉得那算是礼物。\\

「你在这种事情上面还真是缺乏自信啊……」

「不如说怎么可能会有自信啊……可是那家伙哦?」

「熊布偶那时候她挺高兴吧」

「虽然是这么回事啦」

「我说啊周,重要的是心意啊心意。既然你钱也花了东西也买了,那剩下的就只有加入你的心意了啊」\\

朝着以轻松的口气说着的树,周嘟哝着「要是真能那么干脆就好了啊」用手扶住了额。\\

看来直到白色情人节当天,周都得要在这个决定到底好不好的纠结中度过了。
