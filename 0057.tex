\subsection{休业式和树的相求}

% 案外呆気ないものだな、と壇上で厳格な面持ちで挨拶をする校長の姿を遠目に見ながら、周はあくびを噛み殺した。
意外地没意思啊。远远看着在台上神色严肃地打招呼的校长,周强行忍住了打哈欠。\\

% 修了式の日がやってきたが、特に感慨もなくこの日を迎えて登壇している校長の話を聞いている。正直寝てしまいたい程度には退屈だ。
虽说终于到了休业式的日子,但周只是一丝感慨也没有地迎来这一天,听着台上的校长讲话。老实说校长的讲话无聊到简直要睡着的地步了。

% それは周囲の生徒ほとんどが同じ気持ちらしく、真面目に聞いている生徒はごく僅かで、大半が適当に流すか眠たそうに壇上を見ていた。
他周围大部分学生也是如此。认真听讲的学生屈指可数,大半都随便听听,昏昏欲睡地看着台上。\\

% 流石に大っぴらに退屈そうな顔をする訳にもいかないので真面目な表情を作りつつも、早く終わらないかなといった気持ちで一杯の周は適当に聞き流していた。
就连坚决不公然做出无聊表情、表现得极为认真的周,也在想着能不能早点结束,置若罔闻了。\\

% これが自分達の卒業式なら感慨はあったのだろうが、修了式なのでこれといった感動やら何やらは湧かない。
要是这是自己的毕业典礼说不定还会感慨不已,但只是休业式的话周根本涌现不出激动还是什么的情绪。\\

% 言ってしまえば悪いがどうでもいいので、周は優等生の振りをしつつ退屈な時間を過ごした。
不过要是就这么结束了确实也不太好。抱着怎么样都无所谓的心情,周一边装出优等生的样子,一边消磨着无聊的时间。\\

\vspace{2\baselineskip}

%「……あー肩凝った」
「……啊肩膀好酸」

%「校長話長いからなあ」
「都怪校长的话太长了啊」\\

% 式が終わって教室に戻れば、みな口々にそんな言葉を漏らしている。
典礼结束后一回到教室里,大家就忍不住吐槽起来。

% それでもやや声音が弾んでいるのは、この後ホームルームさえ終わってしまえば二週間ほどの自由が待っているからだろう。
不过抱怨的声音并无多少消沉,因为等接下来的班会结束,就开始为期两周的自由时间了。\\

%  やっと退屈な授業から解放される、と口許に笑みすら浮かんでいるクラスメイト達を席で眺めながら、周もそっと吐息を落とす。
终于要从无聊的上课解放了。看着班里同学们欣慰的样子,周也悄悄地松了一口气。\\

% 明日から春休みになるが、どう過ごしたものか。
明天开始就是春假了,要怎么度过这段时间呢。\\

% 一応両親にはこの間顔を見せたので、交通費的にも帰らずともよいと思うが、割と暇になってくるのだ。二年の予習はある程度するにしても、時間が余る。
这段时间周姑且也见过父母了,考虑到路费还是不回家比较好。这样的话春假就变得空闲很多了,就算把这段时间拿来预习二年级的课程,也还颇有闲暇。

% 短期バイトしようにも事前にお目当ての職を見付けてないので日数的に足りないし、休みに遊ぶような友人は樹と千歳くらいだ。
因为事先没有找到合适的工作,要做兼职的话时间上恐怕也来不及,而想找朋友放假出去玩的话也只能找到树和千岁而已。\\

%「なあなあ周くんや」
「我说周君呀」\\

% たった今脳内で話題にした樹が、後ろから話しかけてくる。
说曹操曹操就到,周身后的树向他搭话起来。

% 振り返ると実に爽やかな笑顔……周にとっては胡散臭さすら感じる笑顔で、何だか嫌な予感がした。樹がこういう笑顔を浮かべるのは、何か頼み事がある時や厄介事を持ち込む時だ。
周一回头就看见树非常爽朗的笑容……但对他来说这可疑的笑容只让他感到了莫名的嫌弃。每当树想找人帮忙或者带来什么麻烦事的时候,就会露出这样的笑容。\\

%「なんだよ」
「什么啊」

%「お前、明日から暇?」
「你明天有空吗?」

%「まあ暇だな」
「有空啊」

%「うんうん、だと思ってたよ。よかったよかった」
「我就知道会是这样。真是太好了。」

%「……なんだよ」
「……什么啊」\\

% 満面の笑みを浮かべる樹が、自席の横に下げてある鞄を叩く。
树笑容满面,敲了敲挂在自己座位上的书包。

% 昨日大量に荷物を持って帰ってロッカーも机も空にした筈なのに、こんもりと何かがつまっている。今日は授業がないので荷物なんて精々ペンケースやファイル、財布程度な筈だろうに、不自然なもののつまり方をしている。
他应该在昨天就把橱柜和书桌的大部分行李带回家了才是,这么鼓的书包显然还藏着什么。况且今天也没有课要上了,剩下没带的东西至多也该就笔袋、文件或者钱包什么的,也不知道他放了什么奇怪的东西。\\

%「……それは?」
「……那个是?」

%「着替え」
「要换的衣服」

%「何故に」
「为什么要带衣服过来」

%「泊めて」
「因为要在你家借宿」\\

% 語尾にハートマークが付きそうな程に弾みつつ媚びた声音でおねだりされて、周の顔が思い切りしかめっ面になったのは仕方ない事だろう。
树兴奋到了简直可以在话语句尾看到心形符号的程度,用着巴结的语气对周说道。周不自觉地板起脸来。\\

%「あのさ、お前ほうれんそう知ってるか」
「那个,你知道什么是菠(ho)薐(ren)草(sou)吗」

%「知ってる知ってる、訪問連夜騒音だろ」
「当然知道了,就是访(hou)问连(ren)夜噪(sou)音吧」

%「それ単なる夜間の近所迷惑だ馬鹿野郎。騒ぐつもりなのか」
「借宿什么的只是在晚上打扰邻里吧混蛋。你打算吵死人吗」

%「冗談だよ。泊めてってのはほんとだけど」
「开玩笑的啦。真的只是想住在你家」\\

% 基本的に、樹が事前連絡を欠かす事など滅多にない。
树基本上很少不事先告知就牵扯到别人。

% となると急遽泊まらないといけない事情が出来た、となるのだが、そんな事情が思い付かない。
这么说来他是遇到了什么情况才不得不仓促住在外面,但周想不明白会是什么事。\\

%「朝親父と喧嘩した」
「早上和我老爸吵架了」\\

% そんな周の疑問に答えるように、樹はあっさりと事情を暴露した。
树淡淡地把缘由说出来,回答了周的疑问。\\

%「……千歳の事で?」
「……因为千岁的事情?」

%「ん。うちの親父怒ると数日置かないと話聞いてくれなくてさ。ちぃの家に泊まるのは駄目だろ。ちぃの両親は受け入れてくれるとはいえ、流石になあ」
「嗯。我老爸一生气好几天都不会听进别人话的。然后我也不能住在千家里吧。就算千的父母愿意收留我,但还是不能这么做吧。」

%「俺ならいいってか」
「住我这就能了吗」

%「お前なら泊めてくれると思って」
「感觉你会收留我的」\\

% 部屋が片付いていない時も幾度か泊めた事があるので大丈夫だろう、といった考えなのだろう。
毕竟房间还没整理那会也让树住过几回了,周觉得应该没什么问题。\\

% 周としては、別に泊めるのが嫌、という訳ではない。
在周看来,也没有很嫌弃让他住在自己家里。

% ただ、食事を作りに来る真昼が嫌がらないか、という問題なのだ。
只是,过来这边做饭的真昼会不会觉得不情愿,这是一个问题。\\

% 真昼が休むための場所で天使様モードを強制されるというのは中々に辛いものではないだろうか。
强行让真昼在一个休养生息的地方开启天使大人模式,会不会太过辛苦了。

% 彼女は周にだけ素を見せているので、樹の前では隠そうとするだろう。
毕竟她只对周展现过自己的本性,在树面前的话还是会掩藏起来的吧。\\

% もう一つ問題として、最近真昼が妙に可愛らしい仕草をしたり照れたりして異性として意識せざるを得ないのだが、樹がそれを見てあらぬ勘違いをしそうなのが怖い。
另一个问题是,最近真昼常常会不可思议地做出害羞或者可爱的举止,不由得让人恋爱心爆棚。要是让树看到这个浮想联翩就很恐怖了。\\

%「……あいつに一回連絡する」
「……跟那家伙说一说吧」\\

% 真昼の意思も聞いておかないといけないので、メッセージを送っておいた。おそらく帰る前に一度こちらに買い物のメモを送ってくるので、その時に気付く筈だ。
毕竟不能不问一下真昼的意见,周还是把消息发了出去。不过真昼大概会等到回家前要提醒他买东西的时候才看手机,估计等到那时她才会看到推送通知吧。\\

% 手慣れた動作でメッセージを送った周に、樹は何故か感心したように息をこぼした。
看着周把消息熟练地发出去,树不知为何佩服似的啧啧称奇起来。\\

%「なんだ、同棲でもしてるのか?」
「什么啊,居然同居了吗?」

%「お前だけ暖房と布団なしで床に転がすぞ」
「你睡觉的时候别想有暖气和棉被了」

%「泊めてくれる優しさを褒めればいいのか凍死させる冷たさを嘆けばいいのか」
「我可以对你收留我的慈悲和冻死我的冷漠感到唏嘘不已吗」

%「俺はお前のあらぬ妄想について嘆きたい」
「我对你那子虚乌有的妄想感到可悲」\\

% 何言ってんだこいつ、といった眼差しを向けると樹が肩を竦める。
周用着莫名其妙的目光瞥了树一眼,对方只是耸了耸肩。

% 肩を竦めたいのはこちらだ。妙な勘違いをされて真昼の気を煩わせたくない。
想耸肩的是自己好吧——周才不想因为奇怪的误会而让真昼烦恼。\\

% 樹は何だかんだ空気は読めるので真昼をいじったりはしないと思うが、真昼の居ないところで微妙にからかわれそうなのが若干憂鬱である。
虽说周觉得树应该不会不识趣地捉弄真昼,但趁着真昼不在戏弄自己还是让他有些郁闷。\\

% 樹の笑みにため息をついていたら、どうやらたまたまスマホを触っていたらしい真昼から『三人分の材料を買ってきてくれるなら普通に作りますけど』と承諾の旨が届いた。
周对着树的坏笑叹了口气,这时真昼似乎久违地打开了手机,回复过来了『你去买三人份的食材回来的话就照常做吧』的消息。\\

%「いいってよ」
「她说可以」

%「やった、手料理食える」
「太棒了,可以吃到椎名亲手做的东西了」

%「それ目当てじゃないだろな」
「那个不是你的目的吧」

%「若干あった。周が絶賛する料理って一度味わってみたかったんだよな」
「目的有好几个呢。我也想尝试一下周赞不绝口的料理嘛」

%「……あいつに迷惑かけるなよ」
「……会给人家添麻烦的」

%「俺はお前にはかけてもあの人にはかけないから」
「我给你添麻烦了也不会给那个人添麻烦的啦」

%「俺にもかけんな」
「也不要给我添麻烦啊」\\

% へらりと笑った樹の額にでこぴんを叩き込めば「いってえ!」と言いつつも愉快そうに笑ったので、周はこれみよがしに深くため息をつくのであった。
周对着嘿笑着的树的额头就是一发弹指。树一边喊痛一边愉快地笑着,周也得意地哼了一声。
