\subsection{34 天使大人与圣诞礼物}

% 天使様が途中拗ねそうになるという事態はあったものの、その天使様もゲームをしていたらすっかり頭から抜けたらしくまた一生懸命な表情に戻っている。\\


% ゲーム自体には大分慣れたのか、たどたどしいながらもプレイは出来ているし、なんとかついていけている。\\


% 最初にやったゲームとは違い車を操る、というコンセプトのゲームだからだろう。


% 本来のコースから外れてダートに突っ込んだり壁にぼこすかぶつかったりしているものの、それでも前進出来ている。\\


% ゲーム下手な真昼の事なので逆走しっぱなしにならないかとか不安に思っていたものの、思っていたより順調に進んでいてほっとした。\\


% ついでなので周も画面分割して一緒にプレイしているが、真昼が無意識の妨害をしてきてちょっと辛い。\\


% やはり彼女は自然と体を傾ける癖があるらしく、時折ぽすっと二の腕辺りに頭が寄りかかっては離れという事を繰り返している。


% その度にふんわりといい匂いが漂うので、周としては落ち着かなかった。\\


% まあ、それでも最弱CPU相手なので独走するのだが。\\


% 「……なんでそんな早いのですか」


% 「年季と慣れ」\\


% 幾度もプレイすればコースは覚えるしコーナリングも自然と上手くなるものだ。相手からの妨害も、慣れればカメラワークや遮蔽物などを駆使してある程度は防げる。\\


% 納得のいかなそうな顔をしている真昼には苦笑を返して、そっと一人プレイに戻してやる。


% 彼女には経験が足りないので、大きな画面でまず練習させてからだろう。周の腕を見て自分の腕にがっかりするより、CPUに慣れていく方がいい。\\


% 幸い彼女はやる気があるらしく、一人プレイに戻っても熱心に画面を見つめていた。


% この調子なら、まあ何とかCPU相手になら立ち回れるようになるだろう。\\


% 努力家な面がこういう所でも見る事が出来て、やはり微笑ましくてひっそりと笑いをこぼせば気配で分かったらしい真昼がぺちぺちと不服げに膝を叩く。


% それが面白くて余計に笑えば、真昼が眉を寄せた後「周くんのばか」と小さく呟いたのだった。\\


% \\


% 「勝てました」\\


% 苦節二時間強。


% 画面の端に燦然と輝く一位の文字を得たままゴールを果たした真昼は、ほんのり自慢げに周を見た。\\


% 長い間テレビに向かって格闘してようやく得た栄光の一位。


% 何度も何度もビリを経験して、それでも諦めずに走り続けて順位を少しずつのぼり、やっとの思いで一位をとったのだから、感激もひとしおだろう。\\


% やりきったと言わんばかりの達成感のある表情に、周は素直に称賛の拍手を送った。\\


% 「よかったな。頑張ったの見てたぞ」


% 「はいっ」\\


% 褒められて嬉しかったのか、いつもの表情が少し照れ臭そうに緩んでいる。


% にこにこ、といった分かりやすい笑顔ではなくて、ほんのりと嬉しそうに緩んだ淡いはにかみは、いつもの彼女のクールさからは考えられないほど甘い。\\


% 最近普段のクールさの合間に年頃の少女らしい面を見せるようになってきた真昼だが、今日の真昼はいつにもまして年相応の顔を見せていて、無性に可愛らしかった。


% どこか無邪気とも言えるあどけない微笑みは、周の理性の紐を緩めて頭をなでくりまわしたいという欲求を浮かばせるほどだ。\\


% 猫を撫でたいという欲求にも似た、可愛がりたいという衝動はつい腕に指令を出してしまって……思わず手が持ち上がりかけて慌てて下ろした。\\


% 「どうかしましたか?」


% 「ああいや、なんでもない。うまくなったなあと」


% 「上達しましたか?」


% 「したした。最初に比べればすごくよくなった」


% 「ありがとうございます。楽しくて、つい頑張っちゃいました」\\


% ふふ、とまた笑みを浮かべた真昼が見てられなくて、周は誤魔化すように部屋の棚にあったかごにいれておいた小さな箱を取り出した。\\


% 「一位のごほうびにこれをやろう」


% 「え、あの、別にそんな」


% 「ごほうびが嫌なら白い髭をたくわえた恰幅のいいおじさまからの預かりものって事で」\\


% そう、昨日うっかり渡し忘れていた、クリスマスプレゼントである。\\


% 誕生日とクリスマスがそう離れてないので再度プレゼントに困る事になったのだが、今回はまああてがあったので誕生日ほどの苦労はなかった。\\


% クリスマスプレゼント、という言葉に今がクリスマスという事を改めて思い出したらしくぱちりと瞬きをしていた真昼だったが、おずおずと受け取っている。


% 開けてもいいぞ、と声をかければ、また丁寧に梱包をほどいていった。\\


% (まあ、大したもんじゃないんだけどな)\\


% 箱を開いてゆっくりと取り出したのは、レザー製のキーケースだ。\\


% あまり高価なものでも気後れするだろうから、ブランド物という訳でなく、純粋にデザインとして真昼に合いそうなものを選んできた。


% 花と蔦の模様が刻印されたシンプルなもので、普段使いには困らない程度のデザイン。あまり花には詳しくないので何が刻まれているのかは分からないが、繊細な形のそれはきっと真昼に似合うだろうという事で選んだ。\\


% 「ま、合鍵渡したしな。まあ使わないならそれでもいいから」


% 「いえ、ありがたく使わせていただきますよ。……周くんって思ったよりもセンスいいですよね」


% 「思ったよりもって何だよ」


% 「いえ、普段スウェットとかジャージばかりですし……服装だけならセンス以前の問題ですし……」


% 「こんな機能性ある服は他にないぞ」\\


% 真昼には着飾った姿なんてまず見せる機会がないし、そういうのは面倒でなるべく避けたいので、学生服か緩い部屋着しか見せていない。


% なのでセンス云々の前にだらしないとかそういう印象がついてるのだろう。まあだらしないのは事実なので、払拭なんて出来そうにないが。\\


% 「……ちゃんとしたら、かっこよくなるかもしれませんよ? 中学生の周くん、ちゃんとしてたじゃないですか」


% 「あれは母さんが無理矢理……待て何で知ってる」


% 「志保子さんが『ちゃんとしたらこんな風になるのにねぇ』と写真を……」


% 「あんにゃろ」\\


% まさか母親の仕事に付き合わされていかにも外行き用の格好をさせられた時の写真が流出してるとは思ってもいなくて、周はここには居ない母親に内心で大量の文句を送りつけておいた。\\


% 「……俺はああいうの似合わないから」


% 「そうですかね。周くん、他人とあまり目を合わせないようにしたり髪形で隠してるだけで、別に目鼻立ちは整ってると思いますけど……」\\


% 小さな手が、周の顔に伸びた。


% 伸びた前髪をかき上げるように白い手のひらが額に触れ、視界がいつもより広くなる。\\


% 風呂以外では久々に開けた視界で真昼を見れば、少しだけ驚いたような表情を浮かべる真昼が居た。


% 別に驚く事はない、不細工でも美形でもない普通の顔だろうに、こちらをじっと見る真昼が不思議でならない。\\


% 「……なんだよ」


% 「いえ。前より瞳が生き生きとしてるな、と」\\


% 数ヵ月前は目が死んでましたからね、と非常に失礼ながら否定出来ない言葉を送った真昼は、じいっと周を見上げている。


% そんなに見ても楽しいものではないだろうに、静かにこちらを見つめていた。\\


% なんだかこうして異性に、それもとんでもない美少女に凝視されるのは、気恥ずかしい。\\


% ただ、やられてばかりなのはつまらないのでお返しとばかりに頬にかかる真昼の髪に触れて綺麗な顔を露出させる。


% 触るのには躊躇いがあったが、真昼が何の気なしに周の髪に触れたので、これくらいなら許されるだろう。頭を撫でている訳ではないのでセーフだと思いたい。\\


% (しかしまあ、ほんと美人なんだよなあ)\\


% 改めて見れば分かる、真昼の美貌の凄まじさだ。


% かつて周の部屋に落ちていた雑誌に載った美女とやらより余程彼女の方が綺麗で、魅力的だろう。\\


% そもそも、写真というのはあまり信用にならない。


% 一瞬を切り抜いて加工出来るそれは、ありのままを写す事も、美しさを際立たせる事も、偽る事も出来るのだから。\\


% 目の前に居る真昼は無加工でも可愛らしく、綺麗だ。\\


% 飽きそうにない端整な顔立ちをじいっと見つめていれば、真昼が次第に視線をさ迷わせ始める。\\


% 何なんだ、と思った瞬間には真昼が周の髪から手を離し、瞳を伏せる。


% もぞ、と居心地悪そうにしている真昼は、コントローラーを完全に手放して側にあったクッションを抱き締めていた。\\


% 「あの。その……そうだ。私からもクリスマスプレゼントがあります」 


% 「お、おう、ありがとう」\\


% 一体何なんだと問いかけようとして、真昼が話を遮るように側に置いてあった鞄からラッピングされた袋を取り出して周に押し付ける。\\


% 「じゃあ、私夕ご飯の支度しますので」


% 「え? そ、そうか……?」\\


% それだけ言い残してさっさと席を立った真昼に、周はあまりに早い展開で困惑するしかなかった。

