% 31 嵐が過ぎ去って


% 「本当にごめんな」\\


%  夕方になって樹と千歳が帰ったあと、周はほんのり疲れた様子を見せている真昼に謝った。\\


%  いきなり知らない人間に絡まれて秘密を知られて、真昼も困惑しただろうし疲れただろう。


%  こんなやり取りを志保子の時もやった気がする。\\


% 「いえ、私が迂闊だったのが原因ですので」


% 「騒がしかっただろ」


% 「……賑やかな人でした」


% 「素直にうるさいって言ってもいいんだぞ」


% 「ちょっと勢いはありましたが面白い人でしたよ」


% 「ちょっとどころじゃねえ……。……まあ気にしてないならいいんだけどさ」\\


%  あれは確実にうるさいの領域に入ると思うのだが、控えめな真昼は実にマイルドな表現で彼女を評していた。\\


%  そこまで嫌がっていないのは幸いだったが、あれと友達になってよかったのかは分からない。


%  かなり真昼とは違うタイプなのだが……新鮮さ、という意味ではいいのかも、しれない。\\


%  もちろんあまりに真昼が困った時は注意するつもりなので、気を付けて見守りたいところである。\\


% 「私の周りにはああいった人は居ませんので、少し楽しかったです」


% 「まあ千歳みたいなやつはそうそう居ないだろうな……しつこくしてきたらはたいていいからな?」


% 「ぼ、暴力はしませんので、頑張って言葉でとめます」\\


%  二人して彼女の暴走が前提な気がしなくもなかったが、実際千歳は勢い余ってよく変な方向に熱意を走らせるので、注意が必要だった。\\


%  あとで千歳に直接注意しておこう、という誓いを心に留めつつ、周は窓の方を向いてはらはらと落ちている雪を眺める。\\


%  この天候でなければ、あのカップルにはばれなかったのだが……恋人達の祝福に降ってくれているのかもしれないので、あまり文句を言えなかった。


%  真昼も、雪自体は見るのは好きらしく、周の視線の先に気付いて同じように眺めている。\\


%  冬だから早く日も落ちて、辺りは暗くなっている。


%  もう夜と言ってもいい暗さになっており、雪も淡いものなので家の照明でぎりぎり視認出来るといったところだ。\\


% 「ホワイトクリスマスですね」


% 「そうだな。まあ、俺らにはあんま関係ないんだけど」


% 「綺麗だからいいんじゃないでしょうか」\\


%  交際関係では全くないので、ホワイトクリスマスとか正直関係はないのだが……真昼が喜んでいるので、雪も悪くないだろう。


%  小さく舞い落ちる雪が暮れた世界をうっすらと白く化粧していく。このまま降り続いたとしても、さほど積もらなさそうだった。\\


% 「まああまり降られると公共交通機関が麻痺するので、ほどほどにしてほしいですが」


% 「そこは現実的なんだな」


% 「人はロマンだけでは食べていけませんから」


% 「ごもっとも」\\


%  こんなやり取りが出来るのも、雪ならではなのかもしれない。\\


%  お互いに小さく笑い合って、真昼は立ち上がる。\\


% 「じゃあ私ご飯持ってきますね」


% 「え、持ってくる?」


% 「先に向こうでビーフシチュー作ってたので。流石に七面鳥丸々一羽ローストしても二人じゃ食べきれないかなって……」


% 「丸々一羽使って丸焼き作ろうという発想が俺にはないわ」


% 「周くんが料理下手なだけです。……明日の昼食はオムライスにビーフシチューかけましょうね」


% 「なんてうまそうなものを……」\\


%  そんなもの食べる前から美味しいと決まってるので、今日の夕食を通り越して、明日の昼食が楽しみになってしまうではないか。\\


% 「俺卵は堅焼きがいい」


% 「奇遇ですね、私も昔ながらの方が好きです。じゃ、鍋持ってきますね」\\


%  ぱたぱたと周宅から一時的に帰宅していく真昼の背中をぼんやりと見ながら、周は騒がしかった日中を思い出す。\\


%  本当に、バレるのは想定外だった。


%  元々疑われていたし疑念が深まる程度なら想定内だったのだが……まさか、あのタイミングで真昼が顔を出すとは思っていなかったのだ。\\


%  結果的に事情は説明出来たし、理解者を得られたのはよかった、のだが……少しだけ、複雑な気持ちでもあった。\\


%  もう少しだけ、二人だけの秘密でよかったのではないか、と。\\


% (何考えてるんだ俺)\\


%  二人に一々隠さなくてもよくなったので、生活が格段と楽になるだろうに、微妙にもやもやとしたものを感じてしまっていて、自分でも訳が分からず困惑した。\\


%  結果的に見れば悪くはなかったのに、どこか引っ掛かってすっきりしない。\\


% 「どうかしましたか?」


% 「……なんでもない」\\


%  鍋を抱えて帰って来た真昼が周の様子に不思議そうに首をかしげたが、さすがにこんな言葉では表しきれない気持ちを彼女に漏らす訳にもいかないだろう。\\


%  取り繕うように普段の表情を浮かべる周に、真昼は訳が分かっていなさそうに終始きょとんとした顔をしていた。


