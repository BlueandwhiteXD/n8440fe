\subsection{36 天使様とお正月準備}

% 十二月三十一日、大晦日。


% その年最後の一日であり、年の締めくくりの日である。


% 基本的には来年に向けての準備や大掃除をして慌ただしく過ごす一日なのだが――。\\


% 「あの、真昼さんや」


% 「なんですか?」


% 「……俺何もしなくて良いのか?」\\


% リビングのソファにゆったりと座った周は、エプロン装着で朝からキッチンに立っている真昼の背を眺めていた。\\


% 朝から来ているのは、おせち作りのためである。


% 二人で年越しをすると決めたので、当然おせちも二人前要る。\\


% てっきり市販のおせちを買うのかと思いきや、なんと手作りするらしい。主婦でも大変な作業を華の女子高生が一人でこなすのだから驚きだ。\\


% すごいなと感心しきりなのだが、真昼いわく、\\


% 「そもそもそういうのって事前予約要るから無理です」\\


% との事。


% そう言われると確かにと納得してしまったのだが、それでもわざわざおせちを作ろうとしている真昼には脱帽である。\\


% もちろん手を抜けるところは手を抜くらしく、黒豆とかは煮るのに時間がかかるしコンロひとつ使えなくなる、との事で市販品を買ってきていた。\\


% 「周くん、なにもしなくていいのとか不安そうにしてますけど、お手伝い出来るので?」


% 「出来ません」


% 「でしょうね。邪魔されるよりは大人しくしていてもらった方が楽です」\\


% 実にシビアな観点の真昼に諭されて大人しくソファに座っているものの、やはりというかなにもしないというのは落ち着かない。\\


% 周とて、仕事を全くしなかった、という事はなかった。


% 大掃除は昨日終わらせているし、 しばらく出掛けなくてもいいようにとおせちの材料を含めた大量の食材の買い出しをしてきた。


% 完全になにもしなかった、という訳ではないのだが、今の真昼に比べたら労力はかかっていないだろう。\\


% 「昨日は家具家電動かしてしっかり掃除しましたしお疲れでしょうから、ゆっくりしていてください」\\


% 力仕事を担当した周を気遣うような言葉を向けた真昼は、相変わらずこちらには振り返らず調理を続けている。\\


% ちなみに真昼は自宅の大掃除は既に終わらせていたらしい。そもそも定期的に掃除をしっかりとしていたらしく、そう手間もかからなかったそうだ。\\


% 「いやー、それでもなんというか……悪いなあ、と」


% 「別に料理好きですから苦ではありませんよ」


% 「それでもさあ」


% 「いいんですよ、楽しいですから」\\


% なんて事のないように告げて作業に集中し出した真昼に、周はどうしたものかと頭を抱えた。\\


% \\


% 「真昼、昼ご飯買ってきたぞ」\\


% 流石におせちで手一杯の彼女に 昼ご飯を用意させるのも酷なので、コンビニに行って適当なご飯を買ってきた。元々さほど量は食べない真昼なので、サンドイッチ一袋で問題ないだろう。


% そろそろ休憩に入ろうとしていたらしい真昼もエプロンを一旦脱いでいたので、タイミング的にも丁度よかったようだ。\\


% 「わさわざありがとうございます。そこまで手が回ってなくて申し訳ないですね」


% 「いやもうおせちつくってもらってる時点で圧倒的にこっちの方が申し訳ないっつーか……ほら、食べようか」\\


% 休憩も兼ねての昼食であり、真昼は素直にリビングに戻ってきた。\\


% 「サンドイッチとコーヒーでよかったか?」


% 「ええ、ありがとうございます」\\


% 周から渡されたご飯に小さく頭を下げて受け取り、周の隣に腰かける。\\


% 「ちなみにどんくらい出来た?」


% 「ある程度は既製品で賄ってますし、品目も抑えてますからほとんど終わってますよ。あとは冷まして詰めるのを待つものが多いです。周くんは伊達巻好きそうですからそちらは手作りしましたよ」


% 「なぜ分かった」


% 「卵料理好きって言ってたでしょう」\\


% 些細な言葉だったのだがきっちり覚えていたらしく、わざわざオーブンで焼いてくれていたようだ。オーブンの稼働音がしたので何を作っているのかと思えば、伊達巻だったらしい。\\


% 「ほんのりと甘い感じがお好きですよね?」


% 「よく分かってらっしゃる」


% 「流石に数ヵ月もすれば好みくらい覚えます」\\


% なんとも嬉しい事を言ってくれた真昼は、ハムレタスサンドを口にする。\\


% 周も買ってきたおにぎりをかじりながらキッチンの方を見れば、目につくところに真昼が持参した小さめの重箱が置かれている。


% あの重箱に詰めるのだろう。\\


% まさか一人暮らしの身で重箱が出てくるとは思っておらず、漆塗りに金箔のあしらわれた高級そうな重箱が出てきた時はちょっとびびった。\\


% 「ほんと、ありがたい限りっつーか。……なんつーか、一人暮らし始めた時には想像出来ないぐらいに、今年の後半は充実した食生活だったなあ」


% 「私としてはあなたが今までよく生きてこれたなと思ってますよ」


% 「ひでえ。案外コンビニとか市販品でなんとかなるんだぞ?」


% 「健康的ではありませんね。まったくもう」\\


% 呆れたようにため息をついている真昼だが、表情は仕方ないなと言わんばかりの苦笑混じりのもので、すこしどきりとしてしまう。\\


% 「私が居るからには、不健康な食生活は許しませんよ?」


% 「おかんか」


% 「周くんが不摂生だったのが悪いのです。来年はもっとしっかりとした食生活をしてもらいますからね」\\


% 微妙に気合いの入った真昼の姿を見て、来年も一緒にいるつもりで一杯なんだな、と思うと妙に気恥ずかしさを覚えて、目をそらす。\\


% ただ、その態度を怠惰に過ごしたいという意味だと見なした真昼が少し不服そうに周を見たので、周は違うと言い訳するのに少し時間を費やす羽目になったのだった。

