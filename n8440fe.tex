% !TeX program = LuaLaTeX
% !TeX encoding = UTF-8

\documentclass{article}

\usepackage{luatexja}
\usepackage{luatexja-ruby}
\usepackage[no-math]{luatexja-fontspec}
% \usepackage[ZhongYi]{luatexja-zhfonts}
\usepackage[a5paper]{geometry}
\usepackage{float}
\usepackage{titlesec}
\usepackage{setspace}
% \usepackage{newunicodechar}
\usepackage[yyyymmdd]{datetime}
\usepackage{indentfirst}
% \usepackage{tipa}  % for \textsubdot
\usepackage[perpage]{footmisc}
\usepackage[unicode,hidelinks]{hyperref}
\usepackage[numbered]{bookmark}

% 思源黑体
\newjfontfamily{\jpfont}[
  BoldFont=SourceHanSansJP-Bold,
  YokoFeatures={JFM=prop}
]{SourceHanSansJP-Normal}
% 方正兰亭黑系列
\setmainjfont[
  BoldFont=FZLTZHK--GBK1-0,
  YokoFeatures={JFM=prop}
]{FZLTXIHK--GBK1-0}
\renewcommand{\familydefault}{\sfdefault}

\ltjsetparameter{prebreakpenalty={`—, 10000}}

\onehalfspacing
\titlespacing*{\subsection}{0pt}{8.9ex}{3.4ex}

\def\two@digits#1{\ifnum#1<10 0\fi\number#1}
\renewcommand{\thesubsection}{\two@digits{\value{subsection}}}
\renewcommand{\abstractname}{简介}
\renewcommand{\dateseparator}{.}

\newcommand*\sectotoc[1]{\section*{#1}\addcontentsline{toc}{section}{#1}}

\title{关于邻家的天使大人不知不觉把我惯成了废人这档子事}
\author{作者 {\jpfont 佐伯さん} \and 翻译 taroxd, tongyuantongyu}

\begin{document}

\maketitle

\begin{abstract}
藤宫\ruby{周}{\jpfont あまね}住的公寓邻家住着一位学校人气第一的可爱天使。

拥有被唤作天使的美貌,优秀少女——椎名真昼。周是并无特别出彩之处的普通学生,他曾经以为,尽管她是邻居,自己在过去和未来都不会和她扯上关系。

直到他遇见雨中湿透的天使。

「人情我会还的。说起来,房间最好整理一下。简直看不下去」

「要你多管」

天使大人说话有些严格,两人的关系从把伞硬塞给她之后开始。

周感冒时前来照料,指责周不爱护身体而来帮忙做饭,两人进行共同作业(打扫房间),一起出门……

最初冷淡而后逐渐变得开始撒娇的真昼,和一开始是怕麻烦的消极主义却不知何时敞开了心扉的周——

这是不坦率的两人逐渐走近的故事。
\end{abstract}

\tableofcontents
\newpage

\sectotoc{第一章}

\subsection[天使大人是娇嫩欲滴的女人]{天使大人是娇嫩欲滴的女人\footnote{原文为 {\jpfont 水も滴るいい女},有因为下雨而脸上滴着水的双关含义。}}

「你在干啥」\\

藤宫\ruby{周}{\jpfont あまね}和她——椎名真昼第一次说话,是在连绵的雨中看到她坐在公园的秋千上的时候。\\

\vspace{2\baselineskip}

周今年升到高一,开始了独居生活。在他公寓的右邻住着一个天使。\\

天使当然是一个比喻。然而椎名真昼美丽可爱,使得这个比喻仿佛是真的一样。\\

亚麻色的直发一直都顺滑有光泽,透明般的乳白色皮肤光滑得没有一丝粗糙。端正的鼻梁,还有长睫毛下面的一双大眼,加在一起体现出了娃娃般纤细的美。\\

周和她在同一所高中,并且是同年级,经常听到别人对真昼的评价。其中大半都是「文武双全的美少女」。\\

事实上,她在定期考试中始终保持着第一名。体育课上也总是首屈一指的活跃。\\

因为班级不同,周对她的了解并不详细。如果传闻没有错的话,简直就好像完美超人一样。

没有什么像是缺点的缺点,眉清目秀、成绩优秀,同时性格谦虚老实,会受欢迎也是难怪的。\\

有这样的美少女住在边上,这个环境应该让一部分男生垂涎欲滴吧。\\

但就算这么说,周也不打算对她做什么,也不准备和她走得太近。\\

当然,周的心里也认为椎名真昼这个少女非常有魅力。\\

然而,两人的立场不过是邻居。周既没有机会和她说话,也没有打算和她扯上关系。

要是这么做了,恐怕会受男生的嫉妒。说到底,如果只是住在旁边就能关系亲近的话,爱上她的男生们也不会那么辛苦了吧。\\

再提一句的话,异性魅力和恋爱感情并不总是等同的。周的认识里,真昼是最适合远观的鉴赏用美少女。\\

因为这个,周完全没有对酸酸甜甜的关系有所期待,也没有和她扯上关系,只是住在旁边,甚至连接触都没有。\\

于是,看到她不撑伞独自呆在雨中的样子时,周心里想着「在干啥啊」并且露出了看待可疑人物一样的眼神。\\

\vspace{2\baselineskip}

雨大到让所有人都径直奔回自己家里,然而她,在学校和公寓之间的一处公园里,一个人坐在秋千上。\\

(在雨中干啥呢)\\

环境有些暗,雨也让视野模糊了一些。然而,那显眼的亚麻色头发和校服,一眼就能看出是真昼。

只是,却看不出她是为什么要不撑伞,淋湿着呆在那里。\\

看样子,她似乎也不是在等着别人。对于淋湿她也没有抗拒,只是心不在焉地朝着某个方向望着。

稍稍上仰的,那原本就缺少色素的脸,现在气色很差甚至显得苍白。\\

这是运气不好的话没一会儿就能感冒的状态。然而即使如此,真昼还是静静坐在那里。\\

既然连回家的打算都没有,这应该是本人自愿这么做的吧。或许不该由其他人来插嘴操心。\\

周这么心想着,正准备横穿公园——但最后看到真昼的表情扭曲得仿佛要哭一样,周挠了挠自己的头。\\

他并没有想要和她扯上关系这类的动机。\\

只是,要放着露出这种表情的人不管,会让他良心感到疼痛。仅此而已。\\

「……你在干什么呢」\\

为了表示自己没有别的意思,周尽可能用冷淡的声音向她搭话。她摇着吸收了水分仿佛变沉的一头长发看向周这里。\\

脸蛋还是一如既往的漂亮。

即使被雨打湿了,也无法掩盖住她的光芒,反而像是小物件一样更加衬托出了她的美丽。正所谓娇艳「雨」滴吧。\\

水灵的一双眼睛看着这里。\\

姑且真昼也知道周是邻居吧。两人早上偶尔会相互遇到。\\

只是,因为突然被搭话,因为至今完全不认识的人的接触,暗褐色的眼睛渗出了薄薄的警戒。\\

「藤宫。找我有什么事?」\\

周心里产生了「啊原来记住了我的姓啊」这种微妙的感慨,同时也看出了,这个警戒相比是不会放松的吧。

虽说不是完全没见过,但两人是陌生人。搭话后会增强防备也是理所当然的。\\

说起来,时常有各个年级的男生向她告白或者接近她。或许是她不太想和异性有什么接触。大概她是觉得那些男生动机不纯吧。\\

「没什么事。只是雨中一个人呆在这里,肯定会在意的吧」

「是吗。谢谢你的关心,不过我自己想呆在这里的,请不用管我」\\

语气上不是明显的警戒,而是柔和中却表示出完全不让别人深入内侧的意思。\\

(嘛,果然是这个结果啊)\\

明显她是有什么隐情。对于她这一「不要管我」这一拒绝的态度,周也没有深挖的打算。\\

周原本就是心血来潮去搭话的。询问原因只是自然的发展,并不是周有多么在意。\\

如果她想要呆在这里的话,那样也没什么毛病吧。

倒是真昼,应该是产生了「为什么要来搭话」这种感情吧。\\

柔弱的美貌猜疑地看着周。于是,周仅仅回答了一句「是这样啊」。\\

继续搭话下去的话必然会被厌恶,是时候撤退了吧。

幸好,不管真昼对周的印象是好是坏都没什么关系。于是,周做出了决断,爽快地不管她自己回家。\\

不过,知道有个少女独自一人在这里淋成落汤鸡,心情上也是不太舒服的。\\

「要感冒的,打着伞回去吧。不用你还了」\\

所以,最后稍微多管了一点闲事。

要是她感冒了,我也没法睡得香。抱着这种想法,周把打在自己头上的伞递给了她。\\

周让她接过了伞,准确地说是把伞硬塞给了她。趁她还没说话,周转过身去。\\

周快步离开之后,背后传来了真昼的声音。

然而,声音小到几乎被雨声盖了过去。周没有管,迅速穿过了公园。\\

周的心思只是希望她别感冒,才把伞硬塞给了她。或许因为这样,一开始想要无视她走过的这种罪恶感少许减轻了一些。\\

既然她拒绝对话,周也不打算再和她有什么关系。\\

既然无缘,就此别过吧。\\

再次走上归途的周抱着上述的想法。在这时周还是这么想的。

\subsection{天使大人的提案}

「周,你鼻子好吵」

「你才吵」\\

第二天,感冒的是周。\\

赤泽\ruby{树}{\jpfont いつき}这位同班同学,或者说是损友,指出这一点后,周想要发出哼声表达不满却失败了。

相对的,周用鼻子呼吸的时候就会发出鼻涕的声音,这在这个意义上倒是有哼哼的意思。\\

身体非常不舒服。不知是因为鼻塞还是感冒本身,头中一直感到有刺刺的疼。

姑且是喝了药店卖的药水,然而症状并没能完全抑制下来,结果就是这个模样。\\

看到因比赛而扭曲着脸,和纸巾打着交道的周,树的眼神比起担心更多的是吃惊。\\

「昨天你还好好的吧」

「淋了雨」

「没事吧。话说你昨天没带伞吗」

「……给别人了」\\

在学校自然是不可能说是给真昼了,所以只能蒙混过关。\\

顺带一提,在学校瞧见真昼的时候,她脸色不差,表现得也很精神。只有把伞交出去的自己得了感冒,这状况也只能说是可笑了。

不过,原因是没有好好洗个澡暖暖身子,其实是自作自受。\\

「那么大的雨你还把伞借出去,你人也太好了吧?」

「没办法吧,反正就是给别人了」

「感冒的风险都背上了,你是给谁了啊」

「……路过的迷路小孩?」\\

虽然说比起小孩身材要好多了。不如说其实根本是同一个年纪。\\

(……啊这样啊,她的表情就像迷路了一样啊)\\

自己表达出来之后,终于理解了。

当时真昼的表情,简直就和迷路的小孩找父母一模一样。\\

「你可真是好心」\\

树并不知道周现在想起昨天的真昼是怎样的心情,调戏一样地笑了。\\

「不过啊,不管是借了伞还是怎么着的,你之后随便擦擦身体就不管了吧。感觉那才是原因」

「……你怎么知道的啊」

「你那不爱护自己身体的样子,去了你家谁都知道」\\

所以才会感冒啊笨蛋。被这么若无其事损了一句之后,周不得不闭上了嘴。\\

正想树说的一样,周基本上不太把自己的事情放在心上。

再补充一些的话,周不擅长整理收拾,所以房间乱成一团。吃的东西也是便利店的便当、补品或者在外面吃。

树为此还无语地表示「真亏你敢说是独居啊」。\\

从这个角度来看,周生活太过随便,也难怪会得感冒吧。\\

「今天赶紧回家早点休息吧。还有一个周末,赶紧给治好了」

「嗯……」

「至少有个女朋友来照料该多好啊」

「吵死了。有女朋友的人给我闭嘴」\\

树有些自豪地笑着。周不爽地用手指扣进眼前的纸巾盒。\\

\vspace{2\baselineskip}

随着时间过去,身体状况变得越来越差。

原本感冒的症状还只是头痛和鼻涕,现在还加上了喉咙疼痛和倦怠无力,支配住了身体。\\

放学后,尽管是紧盯着前方尽快往前走,然而感冒的影响比想象中还要大,让步履变得非常沉重。\\

即便如此,周还是到达了公寓的入口。拖着沉重的脚步,走进电梯之后,周把身体靠上电梯的墙上。\\

呼吸比往常更急更热。\\

似乎在学校还忍受住了,然而因为快到家就大了意,身体一下子变得难受起来。

电梯里独特的失重感,平时是无所谓的,而现在就化成了痛苦。\\

不过,马上就到家了。\\

电梯停在了自己住的那层。周缓缓走出电梯,朝着自己房间所在的走廊迈出脚步——之前先愣住了。\\

视线前方,是自己以为不会再有什么话可说的,飘扬着亚麻色头发的少女。\\

就外观而言,她可爱的容貌上充满了活力,皮肤上的气色也很不错。

怎么想都是她更可能感冒,而事实上她却活蹦乱跳的。或许是平时她就注意身体,差距如实地体现出来。\\

真昼的手上,握着前些日子硬塞给她的,整整齐齐折好的伞。

这是明明说好不用还,她也依然跑来还了吧。\\

「……明明不用还的」

「有借有还是天经地……?」\\

她话说到一半停了下来。不如说,停下来是因为看到了周的脸。\\

「那个……有发热吧……?」

「……和你没关系吧」\\

在最差的时机撞上了。周皱了皱眉。\\

说极端点,伞爱还不还,都无所谓。

然而,在这个时机遇到并不是好事。她这么聪明,很快就能推断出周感冒的理由吧。\\

「可是,这是我借了伞的原因……」

「那是我自说自话,和你没关系吧」

「有关系。因为我在哪里所以你才会感冒的」

「无所谓了,不用你担心」\\

周心里是觉得,不想要自己为了自我满足而做的事情反而让人担心。\\

然而,真昼却没有就这么老实放过他的样子。端正的美貌上露出了焦急的样子。\\

「……已经够了。再见吧」\\

有问有答也让周不太惬意,于是周强行逃开了真昼的追问和担心。\\

摇摇晃晃的,周随手接过了伞,从口袋里拿出了钥匙……到这里还没有问题。\\

周略有迟缓地打开自家门的瞬间,身体上失去了力气。\\

或许是终于要走进家门而感到安心的原因,周的身体摇摆着倒向了后方的墙壁。\\

虽然心里觉得不妙,不过走廊上护栏很结实,只是撞一下并不怕撞坏。护栏高度也够,所以不可能落到外面去。只是撞疼的话,那也没办法……周已经做好了疼痛的心理准备。\\

然而,周的手臂被猛地一拉,强行让周恢复成了原本的姿势。\\

「……你这样实在不能放着不管」\\

细小的声音传到了有些模糊的意识里。\\

「人情,我会还的」\\

或许发热上头了,脑袋模模糊糊的,没有能够理解她所说的事情。

因为,在理解之前,真昼支撑起周无力的身体,打开了周的家门。\\

「我进去了,这是无奈之举,请原谅」\\

声音静静的,但却不容分说。

感冒的周也没力气抵抗,就被这么拖着,和第一次和同年代的女性一起回家了。\\

生病时虽然没有女朋友来照料,不过似乎有一个天使来照料的样子。

\subsection{天使大人照料病人}

因为发热,周很晚才想起来自家的现状——不如说是亲眼看到实际情况之后,周才后悔把真昼放进来。\\

周住的公寓是一房一厅,有厨房和储藏室。\\

客厅面积挺大,有寝室,还有储藏室,这对独居生活已经相当奢侈了。因为父母还算富裕,考虑到安全和交通最后选择了这里。\\

要求独居住在这里的是父母,周并不打算提什么意见。不过,周心里觉得,不用花那么多钱也没关系。一个人住这么大的家也应付不过来。\\

先不说这些。周这个人是独自居住,并且不擅长整理收拾。\\

当然,别说客厅了,连寝室都是乱糟糟的。\\

「真是看不下去」\\

天使大人,或者说救世主大人,尽管外表可爱却给周开门见山地奉送上了这么一句。

因为事实上确实看不下去,周也无法反驳。如果知道要让别人进来,周多多少少还会移开点东西。但事到如今说这些也晚了。\\

真昼光泽的嘴唇中发出一口叹气。即使如此她也没有回去,而是把周搬到了寝室。

在中途,两人差点摔倒。把屋子弄得这么乱的本人痛切地感受到,再不认真收拾或许就太不妙了。\\

「总之,我先出去一下,在我回来之前换好衣服。没问题吧」

「……你还会回来啊」

「把病人放在床上不管,我会睡不好觉的」\\

真昼似乎和周上次看到淋湿的真昼那时候是一样的想法。周也不好多说什么。

真昼离开房间之后,周就老老实实按真昼说的,换上了室内的便服。\\

「……乱的到处都是,不如说没地方落脚……这种地方你是怎么生活的啊……」\\

在换衣服的时候,传来了小小的困惑的声音,让周感到非常抱歉。\\

\vspace{2\baselineskip}

换好衣服躺下之后,周似乎是不知不觉中睡着了。周使劲睁开沉重的眼睛,首先映入眼帘的是亚麻色的头发。\\

沿着头发往上方看去,只见真昼静静站着看着周。看来刚刚发生的一切都不是做梦。\\

「……现在几点」

「晚上七点了。你睡了有几个小时吧」\\

真昼淡淡地回答之后,配合着周坐起身体,把倒好在杯子里的运动饮料递了过来。

周心怀感激地接过杯子喝了一些,总算能够把视线转向周围了。\\

或许是因为睡了一觉,感觉身体稍微好转了一点点。\\

周注意到了自己脑袋上凉凉的。摸了摸,指尖上传来了像是布一样,有些硬硬的感觉。\\

周的头上贴着家里不可能有的冷敷贴。注意到这一点,周便抬头看向真昼。真昼直接地回答说「从家里带来的」。

周的家里既没有冷敷贴,也没有运动饮料。所以运动饮料也是她拿过来的吧。\\

「……谢谢你特意拿来了」

「不用谢」\\

冷淡的回答让周只得苦笑。

她提出来照料应该只是由于罪恶感,而并不是想和周聊天。说到底,在几乎只是一面之交的男人家里两人独处,这种状态下想来也不可能亲近地说话。\\

「总之,我把桌子上的药拿过来了。这药最好不要空腹食用,你现在有食欲吗」

「嗯,还算有吧」

「这样吗。那我烧了点粥,你先喝吧」

「……诶,椎名亲手烧的?」

「除了我还能有谁。不想喝的话我喝掉好了」

「啊不要我会喝的请给我喝吧」\\

周完全没想到真昼除了照料自己之外,还会为自己准备好粥,所以惊慌失措了一瞬间。

说实话,真昼的料理水平是未知数。不过,周并没有听说过家庭科的课上真昼有失败这种类型的传闻,至少应该不会太糟糕吧。\\

周立刻就低头回答说要喝,让真昼露出了有些无语的眼神。不过,真昼还是点了头,并把床头柜上的温度计递了过来。\\

「我去把粥端来,你先量个体温」

「嗯」\\

周照着真昼说的,敞开了衬衫前面,拿出了温度计。这时,真昼慌忙撇开了脸。\\

「等我不在房间里了再量啊」\\

真昼的声音微微有些慌张。周往她那儿看过去,只见她的脸蛋有淡淡的红晕。\\

周的心里觉得,男人的胸板不像女孩子一样需要藏起来,所以对真昼的反应感到不可思议。不过,真昼可能是对皮肤的颜色没有免疫力,周只是敞开了前面就让她明显慌了起来。\\

真昼白色的脸颊染上了淡淡的蔷薇色,脸依旧朝着别处微微发着抖。不知是否是错觉,感觉她的耳朵也染上了颜色。真昼的害羞程度可见一斑。\\

(……啊,感觉有点明白她周围的男生为什么会说可爱可爱的了)\\

周觉得真昼确实是美少女,但并没有更多的想法。美丽、可爱,这是不假的,但也不过如此。

也许该说是人造品的美——周对真昼的印象就和艺术品差不多。\\

然而,现在的真昼露出了微微的害羞,她那慌张的样子更有人类的感觉,因此有种不可思议的可爱。\\

「……那你赶紧去把粥拿来不就好了?」

「不用你说我也会去的」\\

只是,两人的关系并没要好到周能老老实实夸她可爱的程度。要是说了出来,肯定会让她对我有奇怪的评价的。于是,周把感想咽了下去。\\

周没兴致地那么一说之后,真昼就啪嗒啪嗒地快步往房间外面走去。

她的动作多少有些慢,是因为动摇呢,还是因为房间太乱呢。恐怕是后者吧。\\

迷迷糊糊地目送她离开之后,周再次小小叹息了一声,心里想着为什么会变成这样。\\

(……嘛,大概是因为责任感和罪恶感吧)\\

一般来说,人是不会跑进不怎么熟的男生家里照料病人的吧。要是被侵犯了事就大了。\\

真昼带着这个风险依然做出了这样的选择,看得出她内心非常愧疚吧。再加上周的态度明显对她没有兴趣,这说不定也是让她安心的原因。\\

不管怎么说,真昼是没有其他办法才来照料的,这一点应该不会有错吧。\\

「……我拿来了」\\

有些发热的脑子想着这些事情的时候,真昼有所顾虑地敲了敲门。

真昼似乎是担心着衣服有没有穿好而不打算立刻进来。这时,周才想起来,把衣服弄松是为了量体温啊。\\

「体温还没量」

「趁我不在还不量好啊……」

「抱歉,我走神了」\\

周老老实实道了歉,把温度计夹在腋下。没过多久就听到了有些闷闷的电子声。

周拿起温度计,只见屏幕上显示着38.3°C。虽然不至于去医院,这个体温还是挺高的。\\

周整理好着装,告诉还不打算进来的真昼「已经好了」。真昼这才端着放着一锅粥的托盘小心地进来了。

她这么明显地放下心来,是因为周把衣服整理好了吧。\\

「多少度啊?」

「38.3°C。喝点药睡一觉就会好的」

「……药店的药都只是针对症状,不能消灭病毒本身。要好好休息,激活身体的免疫功能啊」\\

尽管被责备了,不过周知道这是真昼在担心,所以总觉得心里痒痒的。\\

说着「真拿你没办法」,真昼叹了口气,把锅连着托盘一起放到床头柜上,打开了锅盖。\\

锅里是放了梅干的粥。考虑到对胃的负担,粥比较稀,大概一份米七份水吧。

里面放了梅干,应该不是为了味道,而是听说了这样对感冒好吧。\\

锅上没有冒出热气,然而却传来了温暖的感觉。比起现做的粥,还是故意凉下来更好,真昼大概是这个意思吧。\\

真昼无视了盯着粥看的周,而是麻利地把粥盛到碗里。梅干细细地散在粥里面,而里面的籽都被细心地挑掉了,红色的果肉淡淡地混合到了白色里面去。\\

「喝吧。应该不烫了」

「嗯,Thank you」\\

周接过了粥,不过还是握住勺子盯着粥看。真昼也感到了纳闷。\\

「……干什么,是想让我喂吗。那种服务我是不接受的」

「我才没那么说……只是觉得原来你还会做饭啊」

「一个人独居这是肯定得会的」\\

对于还不能好好独立生活的周,刚刚那句话还是挺痛心的。\\

「……你做饭之前,最好先把房间收拾收拾」

「您说得是」\\

真昼好像大概知道了周在想着什么,赶紧打了一剂预防针。周轻轻嘟囔着,舀了一勺粥放到嘴里来把这事糊弄过去。\\

舌头上粘稠的粥味,不出所料地充分体现出了米的原汁原味。盐放得很少。

不过,碎开的梅干带来柔和的酸味和咸味都非常入味,形成了绝妙的平衡。\\

周并不是特别喜欢吃咸梅干,不过却很喜欢这温和的酸味中带有微甜的感觉。如果身体健康的话,他希望直接把这些梅干浇到米饭上,做出这个味道的茶泡饭。\\

「好吃」

「谢谢夸奖。不过只是煮粥,谁煮都差不多的」\\

真昼一副若无其事的表情回答说。不过她脸上露出了微微一笑。

这和学校偶能见到的对外笑容不一样,是流露出安心的微笑。这让周下意识地凝视过去。\\

「……藤宫?」

「啊,没事」\\

柔和的笑容只露出了一瞬就很快消失了。感觉有些可惜。\\

周心里这么想,但没有说出口,而是再次一口一口舀着粥吃试图蒙混过关。

\subsection{天使大人的粥与现况}

「……总之今天你就静养吧。也得记得好好补充水分。另外如果要擦汗用这个。脸盆里水已经准备好了,把毛巾放进去然后拧干来擦就好了」\\

饭后,真昼勤快地拿来了瓶新的运动饮料,倒好水的脸盆,毛巾,还有备用的冷敷贴,一起摆在了床头柜上。\\

再怎么说以一面之交的关系在异性家里留宿也不合适,而且周也觉得这样不自在,便接受了对方的行动。\\

周默默地看着正检查着有没有遗漏的事情的真昼。\\

(……以发自义务感来说这还真是服务周到啊)\\

虽说嘴上有点毒,但做起事来却十分认真卖力。逐渐习惯了这样的真昼的周,无奈地苦笑。\\

(往后就算是两清了,谢谢照顾啦)\\

估计,之后就不会再和她有什么关连了吧。不过是偶然照看了这边一次呐。\\

那么,既然往后就不再接触了,那就趁这个机会问下在意的事情吧。

大概是药起作用了吧,虽说还是一副昏昏欲睡的感觉,但热似乎已经退了一些。头脑也比睡前清晰了不少。\\

「那个,可以问你件事么」

「什么事」\\

安排好必要的东西之后,真昼看向这边。\\

「那时候你为啥淋着雨坐在秋千里啊。跟男朋友吵架了么」\\

周还是对昨天那致使如今事态的开端耿耿于怀。\\

淋着雨坐在秋千上发呆,到底是为什么要那么做呢。

说回来还是看见她那时那如同迷路了的孩子一般的眼神,对此感到在意才把伞硬塞过去的。

可却不知道她为什么会露出那样的表情。\\

看着那像是在等谁一般的样子,周臆测她是跟男朋友吵了一架什么的,但真昼却一脸无奈地看向这边。\\

「很抱歉,我没有男朋友,也没有交男朋友的打算」

「啊,为啥?」

「不如说为什么会觉得我有呢」

「看你那么受欢迎的样子,怎么说也有一个两个吧」\\

对正普通地搭着话的周来说,她更近乎一个性格比较强势的挺有人味的普通少女,但在周围人的眼里却并非如此。\\

清纯可爱,乖巧谦虚的美少女。令人一见钟情的,天使般的美貌。

再加上年级第一,体育全能,还有刚刚才见识的料理手腕。那样的话想必很有人气吧。\\

周瞧见过别人跟真昼套近乎,周也知道自己的同学有不少对真昼有意思的。\\

都这么一个随便挑随便选的状况了,哪能想到她没有和任何一个人交往。\\

周这么想着,便用了「一个两个」这么一个词,可真昼听到这个词,脸上却突然愣了一下,然后露出了稍稍扭曲的表情。

「没有,我也不记得我是那种跟好几个人交往的没节操的人。绝对不可能」\\

眼神突然变得冷淡的真昼,淡淡地否定到。周这才自觉自己是踩到地雷了。

或许是感冒的原因,周突然感到一阵恶寒。不知是不是错觉,感觉连屋子里都冷了三分。\\

「啊,我不是那个意思。抱歉」

「……是我这边头脑发热了,对不起」\\

不过稍稍低头,冰冷的氛围便瞬间散去了。\\

比起头脑发热,感觉更像是空气在暴风雪中一样冷,但周自然不敢说出口。\\

「……总之,那时候不是这个原因,只不过是想让大脑冷静一下罢了。……搞得你担心我还感冒了,实在是抱歉」

「没事的。反正也是我这边自作多情。因此也不希望你那边有什么罪恶感。这么一来,和椎名你的关系也算是到此为止了吧」\\

想着果然是受罪恶感驱使而照顾周的真昼,听了周的后半句话一瞬间却露出了不可思议的表情。

是对关系到此为止有点在意么。\\

「我们也没什么特别的共同点,关系到此为止也是当然的吧。你再怎么年级第一的美少女啊才女啊天使啊什么的,我也没打算想入非非啊。欠我一份人情真是幸运啊嘿嘿什么的,你以为我是这么想的么?」\\

看着尴尬地稍稍移开视线的真昼,周苦笑着想道果然如此啊。\\

看来这不是本人意识过剩,是确有其事啊。\\

给美少女卖人情,借此拉进关系,这也算是可行的手法了。\\

真昼似乎是经历了好几次这种事情,也难怪那个雨天会那么警戒。既然是为了自卫那这边也不好怪罪什么。\\

「对你来说也很麻烦吧。跟不喜欢的男人扯上关系」

「这倒是」

「是吧」\\

听见了本人的肯定,周反而感到一丝有趣。\\

以乖巧的好学生可爱的天使闻名在外的她,也有喜欢、讨厌和烦恼的事情。周稍稍有了一些亲近感。\\

对真昼来说或许是不慎说漏了嘴,她稍带怨恨的看了一眼诱使她失言的周。

真昼也是个有着感情的人类,这就是最好的证明。\\

「其实也没关系的吧?不如说这边倒是安心了。天使也和人一样有这样的困扰呢」

「……能别再那样叫我了么」\\

看起来似乎叫她天使会很害羞的样子,真昼持续露出着不满的眼神。

这样也挺有趣的,周又一次笑了出来。\\

「嘛,也没啥要紧的事情,没理由特意去麻烦你啊」\\

周下了这样的断言,真昼听了则惊讶得睁大了眼睛,然后微微露出苦笑的表情。\\

\vspace{2\baselineskip}

回想起真昼认真地低头行礼之后离开的场景,周躺在床上呆呆地望着天花板。\\

尽管药起了效,但身体本身还很累。一旦放松,睡意便会涌上来吧。\\

周闭上眼睛,回味着今天的事情。\\

被天使(毒舌系)照看了什么的,说给谁听都不可能被相信吧。而且也不是什么值得说的事情。\\

今天的事情,是周和真昼两人间的秘密。\\

秘密啊,用这个词心里有种痒痒的感觉呢。明明只是因为麻烦懒得跟别人说才这么决定的。\\

明天,就是一面之缘的陌生人了。\\

周这么默念着,渐渐地沉入了梦乡。

\subsection{寒空下的相遇}

如周所言,周与真昼的关系,回到了一面之缘的陌生人的程度。\\

受真昼照顾,第二天周便康复了。在便利店里买东西时两人正好遇到,但也没有什么特别的交流。不过,看见精神好的周,真昼稍稍露出了安心的表情。\\

周一上学的日子,也并无二致。不过他人。

非要说变化的话,仅仅只是上学时碰上了会低头示意的程度吧。\\

「哦——周你没事了啦」

「受你照顾呐」\\

上周回家的时候,看见周那半截入土的样子,树也很是担心,今天便早早地在楼梯口等周。树周末还发了条『没死吧』的短信。\\

就算周回了一条说自己没问题的短信,树还是半信半疑,今天看见本人活蹦乱跳的样子才做出夸张的动作舒了一口长气。\\

「哎呀——你当时都那个样子了我也会担心的啦。嘛,虽说这回病好了,你还是放点心思在生活上吧。比如整理屋子什么的」

「你咋跟哪里的谁一样啊」

「嗯?」

「啊没啥。……只是上周知道了该整理了。近日我会整理的」\\

树立马吐槽「给我现在就去整理啊」,而周则无视了这条吐槽。

那堆东西要整理只花个半天根本搞不定。\\

周无所谓地扭开了头,树尽管没追究下去,但是一脸的无奈。\\

「嘛反正是你家,随你便了。下次我去的时候记得整条道出来啊」

「……我看着办」\\

周苦着个脸在门口换好鞋往教室走去,听见旁边的教室的喧闹,不禁扭头看过去。\\

窗户里映着美貌一如既往的真昼不分男女地吸引着同学的场景。

面对搭话,用安静的微笑应对的真昼的身影,与前几日记忆的她,完全是两个人的样子啊。周这么想着,不自觉地露出了苦笑。\\

看见露出这样表情的周,树也把视线移了过去,看见真昼之后,一脸明白了的表情。\\

「啊啊椎名啊。还是老样子人气爆棚啊。毕竟是美少女呢」

「毕竟是天使大人嘛。……树也觉得椎名可爱么」

「你这么问那当然啦。嘛我的话有千在了,感觉不过是鉴赏用的感觉」

「你能别秀了么」\\

树的话,有一个叫\ruby{千}{\jpfont ちぃ},准确来说是叫千岁的女友。

这可是对关系超好的情侣,两人一起的时候可是能秀到这边眼瞎的程度。\\

「要秀滚开别在这秀」周一边说着一边伸手赶着树,但树却不以为然。毕竟已经是经常的事了,「这家伙真没劲」树笑着回道。\\

「说回来啊,倒是周你不觉得椎名很可爱么?」

「挺漂亮的。就这样」

「真是淡泊呢」

「反正是我们高攀不上的高岭之花呗。也搭不上关系,那不就静静欣赏得了」

「是啊」\\

虽说前几天七七八八的发生了照看自己的事情,但本来就是次元不同的存在。\\

周跟真昼好上什么的,这种未来不可能存在。优秀的人之间才会互相吸引吧。

对自己的没用有自觉的周,跟不但可爱还全能的真昼之间,发生些什么有的没的首先就不可能吧。\\

扯上关系什么的,已经不会再有了。周是这么想的。\\

\vspace{2\baselineskip}

「……你在吃什么呢」

周的想法被颠覆,是周在阳台上吸着果冻饮料看着窗外的时候。

\subsection{名为赠物的天降之恩}

连跑便利店都觉得麻烦的周,一边吸着家里常备的果冻饮料,一边靠在栏杆上呼吸着屋外的空气,结果却恰好遇上了走到阳台上的真昼。

看见周样子的真昼跟周一样把头稍微探出栏杆,然后注意到了周正吸着的果冻饮料,稍稍皱了皱眉。\\

周完全没想到自己会被搭话,呆呆地愣住了一会儿。\\

「看了就明白吧。花不了一分钟便能补充能量的果冻——」

「……你晚饭不会就是这个吧?」

「那还能是啥」

「……明明是个食欲旺盛的男高中生就吃这点?」

「多管闲事」\\

平常的话周是靠着便利店便当就着点配菜过活的,不至于就这么简单。不过今天懒得去弄晚饭,又没心情吃杯面,便靠着果冻饮料应付过去。

估摸着这点量也是不够,等会可能还要来点零食补一下量。\\

「……不做饭么」

「不做也不会做。你不也是知道的么」

「……而且还不会打理卫生真亏你能一个人活下来呢……」

「啰嗦。跟你没关系吧」\\

被戳到痛处的周,皱着眉头把剩下的果冻饮料吸完了。\\

扫除什么的前几天已经吃了亏了本就打算处理一下的。天天说来说去反而搞得人不想干了。\\

周反而非常好奇真昼为何总是这么啰啰嗦嗦的。然而真昼盯着周,接着轻轻叹了一口气。\\

「……请稍等一会」\\

话音刚落,真昼就从阳台走回了房间。\\

听着阳台关上窗户的声音,周小声嘟哝了一句「到底怎么回事」。

光说让人等着,是要等什么啊。\\

周疑惑地看向真昼的家里,理所当然地没有回应。\\

(差不多也凉下来了,我想要回屋啊)\\

虽说也照着对方说的正在等着,但秋天的晚上比预想的还要冷。一件汗衫实在是不够。\\

不如说周也不知道为什么自己会就这么乖乖地等着。\\

外面的眼看就要降到呼吸会有白雾的气温。周长吐了一口气,这时从玄关传来了一阵电子音。

听见这来客的门铃声,周扭过了头。\\

心里有数的来客只有一位。\\

周实在不知她为何会来,避开散乱一地的衣服和杂志,走到了玄关。

即便不看猫眼,也知道来的是谁。周用脚把拖鞋撩到门口,解开防盗栓打开门——不出所料,比周眼睛稍低处,有一丛亚麻色的头发。\\

「……你干啥呢」

「你过得太不像样子了,我都看不下去了。……虽然是剩下的,你拿去吧」\\

真昼语气冷淡地说着,把手往前伸了过来。

比周小上一圈的娇小的手,端着一个特百惠的饭盒。半透明的盒盖模糊地透出里边煮食的影子。\\

或许里面的东西还温着,盖子上起了一层水雾,虽然看不清,但里面是煮食没错吧。\\

周连着眨了几次眼睛。真昼似是理解了眼神里询问原因的意思,深深地叹了口气。\\

「还不是因为你不好好吃饭。补品只是辅助,当不了主食」

「你是我妈么」

「我自认为自己的主张算是很普通的。另外你房间改收拾收拾了吧?现在这样落脚都难」\\

真昼瞄了一眼周的身后,一副受不了的样子明显地眯起了眼睛,让周无言以对。\\

「……走还是能走的嘛」

「根本没有啊。一般衣服就不该掉在地上」

「衣服是会掉的」

「洗好晾干叠好收起来就不会掉了。读完了的杂志也请打包收拾好。踩到滑倒了可就是大问题了」\\

虽说话里略微带刺,但周明白真昼不知为何是纯粹在关心他,自然也不能一一回嘴。\\

确实上次来照看他那次,两人就差点因为房间太乱而摔倒了。被说也是理所当然。

一脸苦涩回不了嘴的周,只好从抿着嘴的真昼那接过饭盒。

慢慢扩散到手掌上的温度,在这渐渐转凉的天气里,让人很是心暖。\\

「……那,我可以吃这个么」

「你不要的话我只好倒掉了」

「别别我吃我吃。天使大人亲自做的晚餐一般可是吃不到的啊」

「……能别那么叫我么,真的」\\

抱着报复的念头试着用了下学校里的外号,结果真昼的脸易懂地染上了红晕。\\

对本人来说这个外号也许是太羞耻了。站在她的角度,周也肯定会觉得不舒服,这倒是理所当然的。\\

看见脸上泛起红晕,甚至还有点哭相地瞪着自己的真昼,周不禁笑了出来。\\

「抱歉,以后不这么叫了」\\

再这么叫估计就真的要坏对方心情了,周也不想玩到过分。再说双方的关系也没好到那种能随便开玩笑的程度,过分了也不太好吧。\\

真昼似乎也不想再被这么叫了,清了清嗓子,表示自己重新振作了一下精神。

然而她脸上微微泛红,看上去和刚才并没有多大区别。\\

「嘛,那我就满怀感激地收下了。你也别再想着那事过意不去了」

「那倒没,反正照顾了下生病的你也算相抵了。这个只是我的自我满足……不如说是你过这样的废人生活我实在是看不下去而已」

「是是是」\\

被看见的时候都是一副邋遢样子,她这么看自己某种意义上来说也是理所当然的。

就连现在周身后的走廊也是乱七八糟的,而且照看他那时已经被看了个光,事到如今想瞒也瞒不住了。\\

「……要好好吃饭,规律作息哦?」

「真要当我妈啊」\\

看着一本正经说着的真昼,周一脸疲倦地吐槽道。\\

\vspace{2\baselineskip}

端着分到的赠物回到家里的周,摸出一双超市拿的一次性筷子,坐在了客厅的沙发上。\\

虽说是真昼强塞过来的,但味道究竟如何呢。

给煮的粥倒是很好吃。虽说感冒舌头有些不灵敏,但那个认真从生米煮出来的粥,味道是一点一点温和地渗透到胃里。\\

可以推测真昼的料理手腕应该不错,那实际上是怎样的呢。\\

怀着几分期待,又有几分犹豫的周打开了盒盖。淡淡飘出的无疑是煮菜的香味。

这是几种根菜和鸡肉煮成的。汤的颜色略淡,清楚地映出了鲜艳的胡萝卜的颜色以及旁边点缀的扁豆。\\

全部切成一口吃下的尺寸的各色食物,强烈地刺激着只吃了点果冻饮料的周的食欲。\\

周迅速掰开一次性筷子,首先夹起了一块萝卜。\\

「好吃」\\

口味如何,迅速见了分晓。\\

味道清淡,有高汤的风味,十分有注重健康的真昼的风格。而且还不是买来的那种颗粒调味料,而是拿鲣鱼和海带认真煮出来的汤吧。口味完全不同。\\

周细细地咀嚼,享受着在嘴里渐渐散开的高汤、调味料、以及蔬菜本身的味道。

即便是不爱吃蔬菜的周,面对这不但充分发挥了蔬菜本来鲜味,还彻底入了味的煮菜,也可以大快朵颐。\\

如同说着要好好吃蔬菜一般的少量鸡肉,吃起来也十分鲜嫩,毫无干柴感。除了量以外无可挑剔。\\

作为女高中生的料理,菜式的选择很朴素,但完全体现出了制作者的水准。

可以说,这味道和刚刚学会做菜的人做出的东西有天和地的差别。\\

要是再来点饭啊味噌啊酱油啊啥的那就更棒了——虽然这么想,但是不巧并没有煮饭……不如说家里的米甚至都用完了,这点小愿望也无法实现。

虽说事到如今,但周还是后悔着。早知道就去买两包速食饭包回来了。\\

「天使还真是厉害啊」\\

周用着这本人听了怕是又要不高兴的叫法,称赞着学习运动加上家务样样全能的真昼,一刻不停地享受着这味道理想的煮根菜。

\subsection{天使大人是老妈一说}

「还你这个。很好吃」\\

第二天晚上,周拿着借给自己的饭盒去了真昼家。\\

虽说周是真的不会干家务,但洗点东西还是没问题的。从礼节上讲也该好好洗干净还回去,这么认为的周,认认真真地把东西洗干净了再带去。

虽说洗的时候的大费周章这事肯定没法跟真昼说了。\\

也许是听到门铃声便预料到了是周,真昼并没有问是谁便走出门外。\\

身着一袭酒红色针织连衣裙的真昼看见周,微微眯起了眼。

瞄了一眼饭盒确认一下之后说着「好好洗干净了呢,真了不起」这样夸小孩子的话,让周略不高兴地皱了皱眉。\\

「为了我特意这样,谢谢了。那再见」\\

真昼拿回了饭盒——到此为止都还没有问题,可接着真昼却拿出了另一个饭盒递给了周。\\

该说是果然吧,饭盒摸起来温温的。\\

里面的应该是茄子炒肉吧。温度没让盖子起雾,透过去可以清晰地看清里面的东西:茄子,炒过的猪肉,以及撒在上面的胡椒。

从颜色上看,肉上面的泛着光的酱汁应该是味噌味的吧。稍显焦色的茄子与泛着光泽的猪肉,看着就很有食欲。\\

肯定很美味吧。

想必如此,然而周却不明白为什么又拿来了饭菜。\\

「……啊那个,我是来还饭盒的来着」

「这是今天的晚饭」

「这我知道,但是啊」

「先问一句,没有过敏反应吧?挑食我就不管了」

「那个倒没有?但是再拿你东西就有点」\\

连续两天晚饭都受人照顾怎么说都说不过去了。\\

营养不全面的身体对此求之不得,更重要的是真昼的料理水平远高于同龄女生,味道肯定不会差。\\

这饭盒里的东西肯定也好吃的。\\

不过这要是被学校的同学看见了那可就要变成超级悲剧了。当然是在周学生生活没法安宁的意义上。

这栋公寓虽然设计上是供一人居住,但考虑到设施和地理位置等原因,租金不便宜。虽然这附近没有见到过真昼以外的同校同学,不需要担心被他们目击,但扯上这样的关系还是让周多少有点犹豫。\\

「做的量一个人吃有点多了,你收下我这边也很高兴」

「……这样的话那我就收下好了。不过一般做这种事情可是会被对方误解成对自己有意思的哦」

「你这么觉得?」

「哪有哪有」\\

毕竟对方一副你笨蛋吗的眼神,没理由想到那边去。

再说了,才貌双全的真昼会对最近被看到的净是没用的地方的周有好意,简直就难以想像。\\

确实从可爱的邻居那得到晚饭这种事,简直就像恋爱漫画里的剧情,但两个人之间却毫无恋爱喜剧的要素。恋爱当然是没有的,对话里也难寻喜剧成分。顺带一提周家里也没有米。\footnote{喜剧在日语中为{\jpfont コメ},与米在日语中的读音{\jpfont こめ}相同。}

有的只是天使大人的毒舌和对周的可怜罢了。\\

「那就没问题了。……反正你本来也就打算靠着便利店便当和开胃菜对付过去的吧」

「你怎么知道的」

「厨房怎么看都没有好好使用的痕迹,桌上还放着好多把超市和便利店里的一次性筷子。再说看你这样子不用动脑都猜得到。而且还一脸不健康的样子」\\

只来了家里一次就全都被看光了,周脸上不禁抽搐。可这也是事实,周也只得默默听着。\\

「……那就这样,我回去了」\\

该说的说完了该给的也给了,真昼关上了门回到了家里。\\

喀哒传来一声拴上防盗链的声音,周看向手上的饭盒。

手里端着一盆温温的晚饭,周叹了口气回到了自己家。\\

放了花椒和味噌的茄子炒肉实在是美味,让周强烈地想要吃米饭。\\

\vspace{2\baselineskip}

结果,每天都拿着空饭盒换来装着晚饭的饭盒,周的饮食戏剧性地得到了改善。\\

真昼做的菜虽然调味并不浓郁,但却都很下饭,周于是每晚都准备好速食米饭,和这些菜一起吃了。

真昼的料理每天都不一样,不论和食还是西餐,抑或是中餐,各种菜系都有出现,而且都非常美味,引得周食欲大发,很是难办。\\

每天都给搞得周都有点过分地期待起来了,况且最近周简直像是被饵料驯化了一样,不吃反而觉得难受了。

天使的料理或许有依赖性呢。虽然心里想着不好,但自己还是乖乖收下了饭盒,吃得津津有味。\\

「……最近脸色不错啊。反思了自己的饮食习惯了?」\\

也许是晚饭补充了不少营养吧,周的脸色好了不少。午饭时,树盯着周的脸看。

吃着食堂里乌冬面的周,面对一如往常感觉敏锐的树,不禁流下了冷汗。\\

「树,我觉得你有点吓人啊」

「咋了啊。还真说中啦?」

「不……嗯……该说是不得不反思吧」\\

每次跟真昼在公寓楼里遇见都会被说教几句,晚上还能得到晚饭,自然生活质量本身是有变好的。\\

虽然心里想要对天使表达感谢,不过也有一点觉得她多管闲事的心情。\\

周稍稍含糊地肯定道,树则愉快地呵呵笑了起来。\\

「嘛那还不是。你那一脸病怏怏的样子肯定过得超胡来的吧」

「啰嗦」

「不过你咋就改正了啊?」

「……被迫的?」

「哈哈,被你老妈知道啦?」

「……虽不中亦不远矣」\\

真昼那语调真的跟老妈似的。

虽然叫成老妈又太年轻太可爱了,然而周却并不想拒绝这不知为何爱来照料的真昼。\\

「……我说啊树。我看上去真有那么不健康么?」

「嘛。原本主要是因为气色太白吧。身子很高但是那么瘦弱,脸上也一副没劲的表情,感觉这脸长得就不太健康」

「脸是天生的」

「知道。要的是更加有活力的脸啊」

「那怎么可能做得到。……这样啊,一副死了的脸么……」\\

也不怎么照着镜子盯着自己看的的周不大清楚自己的样子,不过看来在别人眼里是一副病怏怏的样子。

或许是平常就看见周那一副死了一样的表情,真昼才担心起来的呢。\\

「周你也该多注意一下外表了啊。稍稍整一整也不至于这样啊」

「你这讽刺真随意啊」

「是说你不肯穿好点一身寒酸还一脸死相有什么办法啊」\\

树借此机会劝周注意健康同时也打理下自己的身体,周则说着「多管闲事」扭开了头。

\subsection{天使大人既环保又平民}

「啊」\\

背后传来了一声银铃般的嗓音。\\

虽说是最近听惯了的声音,但这里并不是公寓,而是附近超市的零食区。\\

毕竟是有人在的地方,周没有料到真昼会对他有所反应,困惑地回过头去,看到的却是瞪圆了眼的真昼站在那里。\\

她手上提着个超市的购物篮,里面放着萝卜,豆腐,鸡腿肉和牛奶这些晚餐的食材。

大概是在零食区犹豫地看着的时候遇上了周这么一个情况吧。\\

「先说好,不过是偶然啊。我并没有在跟踪你」

「我知道。毕竟是超市就是这里最近了,这种事情我还是明白的」\\

面对周的抢先声明,真昼小声感叹着「倒是你为啥会想到那边去啊」,看向了手里拿着的记事本。

认认真真记下需要的东西还真像做事一丝不苟的真昼的作风。\\

仔细看了一遍写在那有着可爱花纹的记事本上的内容的真昼,没有看向零食区而是朝着另一边的调味料货架望过去。

「酱油和甜酒」,以这样惹人怜爱的声音找着家庭用品的姿态,比起可爱更显露出一种不可思议的气氛。\\

「甜酒在这边,喏」

「不是那个,是甜酒味的调料。甜酒未成年人买不了的」

「这个也算酒啊」

「被认为是甜味的酒呢。料酒因为加了盐没法直接喝,所以未成年人倒是可以买」\\

周拿起甜酒想要递给真昼,真昼却摇了摇头,把甜酒味的调料放进了购物篮。

几乎不做家务的周第一次听到这种事情,「诶」地回应道,不觉地视线便盯着她背后看了。\\

在摆着酱油的货架面前停下仔细看着的真昼,留意到了写着价格的标牌,低声嘟哝着皱起了眉。\\

「……大特价一人仅限一瓶……」\\

似乎是想再买一瓶备用的真昼惋惜地叹了口气,看向了这边。\\

「……那我也买一瓶咯」

「能懂我的意思真的谢谢了」\\

查觉到真昼话中之意的周苦笑着拿起了瓶酱油,而真昼则满足地微微嘴角微微弯起。\\

「……意外的节约啊」

「节约……不如说是能省就省吧。不花冤枉钱咯」

「这算是日本人的气质……嘛,过着拿父母钱的生活的话一般都会这样的吧」\\

周虽说是独自生活,但经济上还是要依靠父母。

生在比较富裕的家庭,能够住在那样整洁安全的公寓里,真的是要好好感谢父母。不仅要交学费,生活费也要花掉不少,所以无谓的花费还是要尽量避免的。\\

「……嗯。毕竟经济上还不独立,还是保持节制为好」\\

真昼冷淡地回答道,整理起了篮子里的东西。那声音里毫无热度。\\

真昼突然声音变得毫无起伏,有点吓人,但等她再抬起头时便回到了原先的表情。

偶然瞥见的那黯淡的眼神,也看不见了。\\

「……话说你为什么要买这个?」\\

如同改变话题一般,真昼看向周提着的篮子里的速食米饭和土豆沙拉问道。\\

从真昼那得到的晚饭确实好吃,但就凭那点量是不够的,于是周就把米当主食,顺便来点沙拉。\\

「晚饭嘛」

「不健康」

「别啰嗦。不是买了沙拉么」

「虽然是土豆沙拉呢。……为什么过着这样的生活都搞不坏身体呢……」

「你太关心别人了」\\

如同说着你该多吃点蔬菜般,真昼眯细了眼睛朝着这边放出了无言的压力,而周则扭开了头当作耳旁风。\\

这样那样地说了一堆话,正好周也结完了账,用塑料袋装好了东西,而真昼却从包里拿出环保袋装起了东西。

真是个爱护环境的平民天使大人啊。\\

可是虽然东西都装进去了,但那东西的量还是让看着的周略微感到担心。

牛奶酱油加上甜酒味调料,这就已经有三升了,虽说和水的密度有差异但少说也有三公斤了。这之上还买了食材,特别是那根大萝卜,想必是很重了。\\

虽说好好地装好扎好了口,但是就这么提回公寓怎么说都还是件体力活。\\

(结果上看因为有我才让她多买了这些调料和食材啊)\\

估计她是做比平时还多的量,然后分给自己的吧。一直以来得到的已经接近一个人的份量了,虽然她说是做多了,但最近应该是特意多做的。\\

因此自己才得到这么多照顾,那么此时再啥都不干就太不像个男人了。\\

看到真昼扎好了袋子,周试着提了提,虽说对自己而言不算太重,但是让女孩子提那就是挺重的负担了。

真昼虽然擅长运动,但是纯粹比力气又是另一回事了。不如说隔着衣服都猜得到那双细手不大可能有多大力气。\\

看见周的行动,真昼眨了眨焦茶色的眼睛。

比起惊讶,更多的是感谢的氛围。\\

「……不是要抢你的啦」

「我不是担心那个。……就这点我还是拿得动的哦?」

「这种情况是直接接受的人比较可爱哦」

「说的跟人家不可爱一样」

「你看看你学校什么样,对我又是什么样」\\

也许是自己也有自觉吧,真昼稍稍退缩了。\\

学校里常常见到的那个所有人都认可的友善、温柔而谦虚的样子,在周面前却见不到。\\

正确来说,对周倒也很温柔,就是说话比较直白罢了。她似乎是对周根本没留委婉的余地,说话一直是有啥说啥。

这总比说谎要好得多,所以周并没有多在意。\\

看见真昼默不作声,周认为机会正好,便提起了装满了食材的环保袋和自己的袋子,快步走向了出口。

背后似乎有些慌乱的气氛,但周并未搭理,也不管两人拉开的距离,径直向前走去。\\

周并没有调整脚步等待真昼。

本来超市里就已经呆在一起了,要是两人并排着一起回家被谁看见了,恐怕要变成麻烦事吧。

互相来说这个距离都是最理想的。\\

装作两人没有关系,提着大袋大袋的东西匆匆往前走着的周,仿佛听见了后面传来的一句「……谢谢你了」的声音。

\subsection{天使大人的扫除大作战}

对周来说家务活全都是不擅长的事,而其中打扫卫生则算得上是最为令他苦恼的了。\\

至于下厨,如果以受伤为前提,并且忽视外观与味道,倒也不是做不出来。\\

热一热塞进胃里不就得了——如果是这种理念下的,不但没有品相,口味也不太行的东西,周并非完全做不出。

当然这样的东西周既不想做也不想吃,也就自然不会去做。\\

洗衣服这事要是不会那生活估计要寸步难行所以没有问题。

实在不行也还有洗衣店,只要普通地把衣服扔进洗衣机,加点洗衣液和水一起转一转就可以,所以洗衣服是可以顺利完成的。\\

不过,只有打扫卫生这一件事,周是是在没有办法。\\

「这怎么办呢」\\

周末,被树和真昼两人催着整理屋子的周,总算是下定决心要来打扫屋子,可面对这一片狼藉却无从下手。\\

心里也清楚是自己的不对,但是东西这么多,周并不知道怎么收拾才好。\\

总之先洗了床单,并把被子晒了晒。

再往下该怎么打扫呢。

衣服跟杂志扔得到处都是,几乎找不到可以落脚的地方。\\

不幸中的万幸是,因为吃的东西味道很大,所以都会及时扔掉,因此没有什么散发着异味或是沾着油污之类的过分的状况。不过是在地上扔得乱七八糟罢了。

不过扔得是真的乱,因而周十分苦恼。\\

正当周叹着气时,玄关处传来了门铃声。\\

周不禁漏出了啊的一声。\\

已经习惯了的来访者,不如说是送完东西就走的上天恩惠的快递员般的存在,如今却如同救世主一般。\\

周快步走向玄关,结果脚下一滑,手直接扑向墙上,打开了门。\\

「打扰了,来拿下昨天的饭盒……你在干什么呢」

「……正准备打扫卫生」\\

周以一副站不稳的姿势看向真昼的脸,真昼的眼神则微妙地透出无语。\\

「刚才好像传来挺大一声」

「……差点滑倒了」

「就知道是这样。打扫还没开始吧?」

「无从下手」

「就知道」\\

「乱得这么过分,确实没法下手」,真昼一如既往地发表着毫无顾虑的言论,令周不禁表情抽搐但却无从否定。

再说要是周非得跟她吵个胜负,那就没法让她帮忙了。\\

不过说回来,该怎么问她呢。\\

周是准备问她扫除的窍门,但说起来她会给建议吗……周以略带犹豫的眼神看向真昼,而真昼则看着周背后散乱的走廊。

看着身后的惨状真昼眼神仿佛说着「呜哇」一般。在真昼看来,走廊乱得真是相当过分吧。\\

「真是的。……这屋子,我来打扫吧」

「哈?」\\

周本来是觉得让真昼来帮忙这样的请求实在是厚脸皮,只打算问一下打扫方法的。

没想到,真昼却直接提出要帮忙。\\

「隔壁屋子这么脏想想也难受」\\

真昼的话一直都很过分,周也已经习惯,没什么生气的了,更何况真昼说的也是事实,也没办法反驳。\\

「连家务都不会做居然一个人生活,你是在闹着玩吗。猜也猜得到是抱着总会习惯的乐观想法混日子,结果至今还是什么都不会,你稍稍反省一下怎么样」\\

周甚至连咕的声音都发不出来了。

妈妈也经常说着勤打扫就很轻松,而周却一直放置不管,结果就成了这样。完全就是自食其果。\\

「本来的话,明明要是平常就勤打扫的话根本就不会变成这样的。这就是平日里怠慢的后果啊,真是的」

「……你说的对」\\

被说到这个地步周还不生气,一方面是真昼经常照顾周,让周没脸面对她,更何况真昼也确实说中了周的心情和过去的行动。\\

就是因为以为放着总能解决,没有重视,结果才变成这个样子。周已经只能对着真昼说的话默默点头了。\\

「这屋子可以让我来打扫么」

「……那就拜托你可以吗」

「既然是我提出来的那当然可以了。还有,我先去做点准备,你要是有什么隐私物品或者是贵重品就放到储藏间锁好门吧」

「这个不用担心」\\

虽说真昼说话很直,但是她那么亲切地来帮忙了,哪有理由去担心她偷东西啊。\\

再说,这么遵守常识还爱管别人事情的真昼,根本不可能加害于人吧。\\

「……你不会担心吗?」

「你也不是会做那种事的人吧」

「不是……是说你不担心我看见你作为男性想要藏起来的那些东西吗?」

「很抱歉没那种东西」

「嘛,那倒是没问题。那我就去换身衣服,拿打扫卫生的工具过来。……这回大扫除,可要彻底扫干净了」\\

真昼耸耸肩,先回了自己屋子一趟,而周苦笑着望着她的背影。

\subsection{天使大人是(垃圾)扫荡大作战总指挥官}

返回周屋子的真昼与刚才相比换了一副打扮,穿着白色的长 T 恤和干草色的宽松长裤。

紧紧贴着身体的T恤,勾勒出窈窕而富有起伏的身体曲线。

真昼把长发盘成丸子扎了起来,露出的雪白后颈让周微妙地有些不自在。\\

平常光看到她穿连衣裙或是短裙了,对周来说这样的真昼有些新鲜。\\

本想着这样男孩子气的衣服会不会不适合真昼,但看来是多虑了。

周反而痛感美人穿啥都好看了。\\

不过,这身衣服看起来确实容易活动,但像是出门穿的打扮。不知道弄脏有没有关系。\\

「这身衣服弄脏了没事么」

「反正过段时间就打算扔了的,弄脏了也没关系」\\

边说着边穿好围裙的真昼,再次环视周屋子的惨状,轻轻叹了口气。\\

「先说好,要彻底打扫哦?」

「……知道啦」

「知道了的话那就赶紧开始吧。我可是不会放水,不会让你妥协的哦」\\

被真昼「可以吧?」这样不由分说的语气一问,周只得从顺地乖乖答应。\\

如此这般,由天使发起的扫荡大作战拉开帷幕。\\

\vspace{2\baselineskip}

「总之衣服先放到洗衣篮里面吧。原本打扫卫生应该要从上到下的,但要用吸尘器得先解决这些衣服。地板都要被埋起来了。衣服这么多,要洗的话就分成几个部分来吧。还有穿过的和没穿过的分开来了吗,可以全部洗掉吗」

「啊啊都照你说的来就好……」\\

理所当然地,就算有吸尘器,也得老老实实先从清理地面这堆东西开始。\\

「……地上没有扔着内衣吧?」

「再怎么说那些也会收进柜子里啦」

「那就好。总之衣服就一会再来处理,就算洗干净了晒干,等下还是会被扫除弄起的灰尘弄脏。而且地方也不够晒的。不着急的话就等打扫完再去洗吧」

「是」

「……然后,杂志的话基本上扔掉吧。你要是在收集的话另说,扔成这样想必也不是。特别想留存的页面就先剪下来,之后就可以处理了,扎好之后拿给废品回收站吧」\\

迅速开始动手进行打扫的真昼指示着周把衣服都收拾到洗衣篮里,自己则把杂志一本一本叠了起来。\\

虽说让周看看有没有特别要留下的杂志,但其实周并不在意这个,摇了摇头否定。真昼看到了便用自带的塑料绳麻利地扎好了杂志。\\

「衣服收拾好了的话就过来分辨下其他的杂物哪些是要的吧。扔在地下的杂物也一样,区分出需要的和不要的,把不要的扔掉。可以吧?」

「……哦」

「要是有意见就赶紧说」

「呃,这倒没有……只是觉得好有条理啊」

「不这样时间不够啊。也不想想多乱」

「您说的是」\\

虽说是周末,但时间仍然是有限的。考虑吸尘器的噪音对邻居的影响得在白天完事。\\

而光是使用吸尘器之前的工作就十分地费事,真昼明白这点,才会尽可能趁早开始收拾东西。\\

虽然想着劳烦真昼到这个地步真的很过意不去,不过多亏了真昼,眼看着就形成了越来越多的落脚之地,因此周的心里也非常地佩服。\\

「椎名老师……」

「既然叫了老师还不赶紧学。你的私人物品这些我没法判断,你就把必要的东西好好地挑拣出来吧」

「Yes, sir!」

「别搞得跟我是个男的似的」\\

随口吐槽的天使大人,一脸严肃地用灵巧的双手收拾、处理着她能判断的东西。\\

有啥东西都想存着的癖好的周,对真昼的干脆果断十分感谢、羡慕。\\

尽管是别人的房间,真昼还是收拾得毫不客气。她真是一副顾家的样子,动作顺畅得已经好像家庭主妇一样了。

真昼那有条不紊的动作,仿佛自己一个人也能轻松收拾完这间房间一样。\\

不过,真昼大概是赶得急,所以没有注意脚下吧。\\

接下来这事毫无疑问是周的原因。真昼似乎是踩着了地上的衣服,结果就失去了平衡。\\

正在真昼嘴里发出「啊」的声音的那一瞬间,周下意识地滑铲到了真昼眼看就要摔倒在的地板上。\\

柔软的触感,与香甜的气味。其中稍稍混有的灰尘味,大概是由于周的慌乱而扑腾起的尘埃所致吧。\\

周屁股着地,有一阵钝痛,不过还在忍受范围内。周感受着靠在自己身上的真昼的重量,只是轻轻地因疼而叫了一声。

情急之中能把她接住,算是很幸运了吧。\\

「藤宫……」\\

真昼抬起头,以微微发愣的视线看向周。尽管她似乎没有生气,不过看样子似乎是有很多话想说。\\

「摔倒了是我的不对,但就是因为会发生这种事情才要整理屋子的啊」

「真的十分抱歉,我在反省了。……没受伤吧」

「没事。特意来接住我谢谢了。我才是该说对不起」

「原因都在我啦……」\\

本来就已经从对方那得到了晚饭,现在甚至连打扫卫生都过来帮忙,而且因为这个还受伤了,实在是难说过去。

不如说实在是抱歉,连脸都不敢对上了。\\

如果真昼愿意的话,周甚至考虑了下跪,但是真昼似乎没有因为摔倒而责怪周的意思。

「真是的」,真昼以可爱的声音抱怨道。至于这一点就是没办法的了。\\

「收拾的目的可就是为了防止这种事情哦?」

「我知道。真的太抱歉了」

「……呃其实没必要说那么重了。毕竟也是我擅自要过来帮忙的」\\

真昼有些慌张的样子,抬头看着周这边。

虽非故意但真昼以这样紧贴的姿势在极近距离以微微不安的眼神仰头看着,令周根本冷静不下来。\\

对与女性没什么缘分的周来说,只是这个距离就够对心脏不好的了,还加上正与美少女紧密接触着。

再怎么说双方没有恋爱感情,周也觉得这样并不合适。\\

真昼似乎是没有意识到这个姿势一样,于是周轻轻抓住她的肩膀把两人分开,在羞耻心泛上脸上之前站起了身。\\

「……那就,继续吧」

「是呢」\\

幸运的是真昼似乎并未注意到周的动摇,抓住了周伸出来的手站了起来。

真昼似乎完全没有对两人的身体接触有什么意识,脸上的表情和平时一样。

周的话则是以「真昼这样被许多男性报以好意的少女应该不会因此就动摇」这样的认识接受了现状。\\

周一脸苦笑地看着平静的真昼,觉得全让真昼帮忙很不好意思,于是打起劲来重新开始了打扫卫生。\\

「……真是吓了一跳」\\

周应付这不熟悉的打扫卫生的工作也是十分头疼的吧。\\

因此,真昼那句小声的感叹,以及隐藏在淡色的秀发下那微微发红的耳朵,并没有被周注意到。

\subsection{天使大人,第一次的}

「……呼,总算是变干净了」\\

结果,搞干净周的屋子把一整天都搭进去了。\\

整理地上的私人物品花了几个小时,然后还有洗衣服清理灯具擦窗户弄吸尘器这一大把事情,等到全部弄完,已经是太阳落山的时间了。\\

真昼过来的时候还能见到的太阳现在已经完全落了下去,可见两人到底忙活了多久。\\

不过,也正是因此周的屋子才变得焕然一新。\\

地上十分干净,没有扔的到处都是的东西;窗户和窗沿也没有污渍;灯也清理了灰尘,比以前更加亮堂了。\\

「居然花掉了一整天啊」

「毕竟乱成那样嘛……」

「那是你搞成那样的」

「您说的是」\\

面对天使大人兼救世主大人,周简直没法抬起头来,只得以毕恭毕敬的态度(下跪磕头被拒绝了)看向帮自己到这个地步的真昼。\\

特意花费了自己的一个宝贵周末的真昼则感叹了句“真是的”,扎好了垃圾袋。

虽然嘴上很毒,但她并没有显露出不悦,反而有些成就感。但她也稍稍露出了些疲劳。毕竟白让她忙活了一天,会疲劳也是肯定的吧。\\

还让她再去做晚餐就有些说不过去了。

且先不论有没有自己的那份,这种状态再让对方工作就真的很抱歉了。\\

「连出门买晚饭的劲都没了啊,干脆就点个披萨吧。今天就让我出钱吧,毕竟平常拿了你那么多东西」

「可是」

「不想和我一起吃的话那你就把你那块带回去也可以」\\

要是不想和周一起吃的话,那也没办法,只好让她自己带一块回去吃了。

比起一起吃更多是出于想要慰劳和感谢对方,所以自己一个人吃也没关系。\\

「……虽然不是这个意思啦。只是,披萨以前没有点过,有点吃惊」

「诶,没有点过么」

「……因为毕竟是一个人住所以没有点过……虽然有做过」

「居然能想到去做,太厉害了吧」\\

普通来说想吃披萨了都是出去吃和点外卖二选一的吧。

特意从头做这种费功夫的事情的人除了真昼应该没多少吧。\\

真是擅长料理的人才会有的想法啊,周产生了这样的感想。\\

「点外卖什么的很平常的吧,我就经常点。你是连家庭餐馆都不会一个人去的吗」

「根本没有去过」

「这可就很少见了啊。我的话就算一个人也会去,爸妈懒得做饭的时候也会一起去。你爸妈是不爱出去吃饭的吗」

「……我家的保姆会给我们做饭」

「还请保姆啊,挺有钱呐」\\

有钱人家的话那倒是可以理解。\\

平常举止也很优雅,衣服和东西看着也很高级。

从这有品位的氛围和有教养的举止来看的话,不如说那样倒不奇怪。\\

本人听到周的话露出了薄薄的微笑。\\

「是呢,应该算是比较的富裕吧」\\

真昼脸上的笑容,既非高兴,也非自豪,而更似于自虐,流露出几分自嘲的心境。看见这笑容,周开始后悔自己的多嘴了。\\

以前提及两亲的事情的时候她的回答也很冷淡,也许是跟父母关系不大好吧。

也许是不大想被提起的事情,所以周不打算刨根问底。\\

既然是人的话总归是有那么一两件不想被知道、提起的事情的。不多过问,这也是面对不那么亲近的人的一种礼仪吧。\\

「嘛,也能当作一次经验嘛。喏,挑你喜欢的」\\

周就此结束关于两亲的话题,把披萨的广告拿给真昼看。\\

这家店即是周常点的店,也是周所知道的范围内,有外卖服务的店里面味道最好的一家。\\

虽然肯定是比不上用专门的石炉烤制的,但可选的配料从普通的到小孩子喜欢的各种各样的都有,想必肯定会有能对上真昼口味的吧。\\

顺着话题转换,真昼接过了菜单,迅速地扫了过去。

带有通透感的焦茶色的眼睛,钉在了各种各样披萨的照片上。\\

平时不怎么浮现出感情的双眼,现在看起来却仿佛闪耀着生气。\\

(……莫非,她其实挺期待的)\\

不知是否多心,真昼看起来似乎有点兴奋,在看了一会儿菜单后,指着一种一般聚会时点的可以体验四种味道的披萨,告诉周「那就这个吧」。\\

周同意了像是在偷瞄这边一般的真昼的请求,然后真昼眼里便微微泛出光亮。\\

即便表情变化很微小,周还是察觉了其中露出的喜悦,无奈地微微笑着,一只手拿起了电话拨通了广告上写着的电话号码。\\

\vspace{2\baselineskip}

大约等了一个小时后,披萨送到了,真昼便迅速开动了。\\

因为有四种口味,真昼似乎是烦恼了一小会到底要从哪种口味开始下手,最后似乎是决定先从培根和香肠的味道开始下手。

不意外地很有大小姐范的真昼,小口地咬起了披萨。\\

虽然是用手抓,但进食的动作还是莫名地流露出一股上品感,这大概是受到的教育的缘故吧。

但与此同时,周却又感觉真昼的举止透出一股小动物般的可爱感。\\

细细地眯着眼看着扯出丝的奶酪,微微松下脸来的样子,看起来格外的可爱。\\

平时真昼看上去很成熟,氛围上也很稳重,而现在的真昼是符合年龄的氛围。

看着哈呣哈呣地小口吃着披萨的真昼,周产生了一股想要摸她头的强烈冲动。\\

「……怎么了?」

「呃,只是看你吃的津津有味」

「……别老盯着我看啊」\\

不过,不满地皱着眉头的表情并不可爱。\\

「……怎么说呢,你还真是不可爱」

「不可爱也无所谓吧,不如说要我现在还一副学校里那样子你也只会不舒服吧」

「嘛那倒是。比起学校里那样子还是更习惯现在这样」\\

周跟真昼在学校既没什么接触,也没说过一句话。

不过是偶尔能够看见那对每个人都同样和蔼的,天衣无缝的美丽笑容罢了。

而相对的,现在看见的她却不那么顾虑他人。\\

估计本来的真昼是这边的样子,而在学校则是进入了外出模式吧。\\

「对我来说倒是这边的样子更不容易累呐」

「不可爱的样子么」

「别记仇啊你。……怎么说呢,学校里的你不知道在想着什么」

「主要是晚饭做什么和上课的内容吧」

「你还会装傻说相声啊」\\

周是想表达真昼像是有什么隐情一样这个意思,但真昼却按着字面回答了。

本人似乎没有装傻的意思,以微妙的不悦的眼神看向这边。\\

「不是那个意思,是说看不到你的内心啦。所以说是这样,比起不知道你想着什么的样子,还是这边这样即便有些不大友善,但是能直率地表达自己的感情的样子比较容易相处」

「……学校的举止是不行吗?」

「也许是处世方法啥的吧不算是不行。只不过感觉你会不会很累啊」

「没有。反正从小就这样了」

「根深蒂固啊」\\

若是从小的习惯的话作出那样的举止也算是能够理解,然而这也表示她是故意要作出『理想的好孩子』的样子,并且别无选择。\\

只是,这些隐隐约约透露出来的家庭环境,周哪好厚着脸皮去追问。\\

「……嘛,有个能松口气的地方不也不错嘛?结果上看就当我帮你纾解压力了吧」

「……看着你那让人放不下心的样子我还真没办法放松」

「那真是抱歉」\\

周动作夸张地耸耸肩,真昼则是有些开心地微微笑了出来。

\subsection{友人的来访}

自那次打扫卫生以来,周与真昼间隔着的障壁似乎略微变薄,但两人互相也并没有特别去拉近距离。\\

在学校还是毫无关系,即便学校外顶多也就拿晚饭的时候的几句寒暄而已。

前几天还被提醒了要好好维持家中的整洁。虽说言语有点直接,但周还是深刻地感受到真昼这位少女是有多么地爱照顾人。\\

也多亏了真昼的提醒以及顺便给出的打扫建议,周的家里保持住了扫除之后的整洁。\\

\vspace{2\baselineskip}

「变得干净多了啊」\\

听说周屋子变干净了,于是周末跑来的树,看见这焕然一新的屋子,发出了感叹的声音。\\

「居然变得这么干净了啊。以前明明那么脏。以前也帮你收拾过结果没两天又脏了」

「啰嗦」

「不是我说啊。你最久保持过地上多久不扔东西啊」

「放心,我可是达成了新纪录。已经保持了两周了」

「新纪录才两周你能有点羞耻感不」\\

听着树说着些一般东西不会丢地上什么的大道理,周微妙地皱起了眉头,但碍于树的亲切心和告知常识的本意,不好拒绝。

再说,真昼帮忙之前,树也帮过周,因此在这个方面周也没法强硬起来。\\

看着憋着说不出话的周,树愉快地笑了起来。\\

「不过啊,变这么干净了有点想把千她也带来啊」

「别啊,为啥我得在自家看你俩撒狗粮啊」

「别在意嘛」

「别把我家当约会地」\\

为啥自己非得看着朋友一对情侣秀恩爱啊。

看着那俩公认的笨蛋情侣秀恩爱的样子,搞得自己也想要女朋友了啊。\\

虽然知道树是在开玩笑,但自己天天看着那俩放闪,还真是笑不出来。

这种事情还是希望他们在自己家里做做算了。\\

「嘛开个玩笑。都搞得这么干净了那就别再弄脏咯?」

「我会考虑的」

「所以说你啊……嘛随你了。但最好还是养成把拿出来的东西放回去的习惯比较好哦」

「你是我妈么……」

「周你也真是的,屋子不经常打扫可不行哦?」

「不但恶心口气还挺像我妈,你好可怕啊」\\

树故意捏着嗓子发出的假声让周直起鸡皮疙瘩。\\

树明明没见过周的妈妈但却演得挺像,简直吓人。

况且树个大男人一副娘娘腔真的恶心得希望他快点住手。\\

周吐出舌头装做要呕,树则笑个没停。\\

「周你妈原来是这样的啊。我妈是真的不管事的那种啊」

「要说我还羡慕你嘞。我妈是事情一件一件没个完的那种」

「关心儿子的妈不好么」

「那样只会让孩子没法独立……」

「但你的情况是你太懒了,妈没法不照顾你吧……」

「啰嗦。就算这样老妈她也太爱管我闲事了」\\

大概是独生子的缘故吧,周的母亲十分关心周。

与溺爱不同,是那种事事都要插手,啥都要操心的类型,虽然谈不上讨厌,但是应付起来比较头疼。\\

为了上高中方便一个人住在学校旁边的时候也是这个那个地说了一堆,不时还跑来突击检查,实在是头疼。\\

「嘛,不也正说明周被看得多重要吗?」

「好沉重的爱」

「你就放弃吧。到时候你就知道这有多么珍贵了」

「明明自己就是个标准的反抗期孩子,亏你能摆出一副经验之谈的口气啊」

「哈哈哈。千的事那就没办法啦」\\

因为女朋友的事情跟父亲有不少争执的树来说这些实在是缺乏说服力,但话本身也算是有几分道理,那就姑且听一听吧。

这家伙自己也有自己的问题啊,周这么想着长叹一口气,而树本人则一脸乐观,表情上完全没显出辛劳。虽说树也曾出言不逊「敢妨碍我跟千的都被马踢死吧」。\\

「总之,老爸那边我会想办法的啦。所以周你要好好过哦?」\\

看着爽朗地笑着的树,周几分烦躁地回答「你不说我也知道啊」,心想着树说的内容跟那个谁一模一样,轻轻苦笑起来。\\

\vspace{2\baselineskip}

看看周的生活状况——其实并非树跑来的真正的目的,而真正的目的只是单纯地跑来玩罢了。关于房间的话题很快结束,两人开始玩起了游戏。

当初的目的是为了一周后的考试学习,但不知不觉就变成了玩游戏。\\

「喂,你瞎浪费回复道具到时候可要没得用了啊」

「没事没事总归有办法的嘛」

「不是,说什么总归有办法啊你等级又没上去,到时候要出问题的……」\\

周烦恼着怎么吐槽追求刺激玩法的树,就在这时门铃突然响了起来,顿时令周生发出别的烦恼。\\

「嗯?客人?」\\

树也把游戏调进菜单画面之后抬起了头。\\

树知道周很少告诉别人自己家在这里,因此会来的朋友也几乎没有。再说就算有客人也会被入口的门禁拦住,应该会拨对讲门铃的。\\

「我也不清楚,大概是邻居吧。应该是有什么消息要传阅之类的」

「这样啊」

「我稍微出去下」\\

周好歹抑制住了自己抽动的脸部肌肉,瞒过了周快步走向玄关。\\

她按完门铃之后没有出声算是万幸了。\\

周也不做确认,直接伸手开门,为了避免让树看见开开个小缝钻了出去,然后顺手关上了门。\\

真昼一如预想站在门外,看着周行动一反常态而连连眨着眼睛。周竖起食指做出「嘘」的动作。\\

「……拜托小声点。树来了我家」

「树?」

「我朋友。过来玩的」

「啊,原来是这样」\\

明白了周一副鬼鬼祟祟的样子是怎么回事的真昼点了点头,不再追究这个问题,而是和往常一样把饭盒递给了周。

看样子是从早上就开始准备了的。里面装着的关东煮,在这个天气渐渐转凉的季节是再适合不过的了。\\

感谢地收下了的周,看着理所当然地递来饭盒的真昼,叹了一口气。\\

「……呃那个,一直都想着很感谢你的照顾,不过一直都没找到时间说。抱歉」

「我也不是为了被你感谢才这么做的。……不错嘛,屋子还能收拾到能招待朋友的地步」

「要我下跪磕头以表感谢吗」

「不是不是。千万别」\\

真昼一副像是在说别弄得我像个坏女人啊的眼神,令周不禁苦笑。

毕竟面对她是真的没法抬起头,周说这话也是有几分当真的。受那么多照顾下跪磕头也不为过。\\

从她那拿的晚餐量很可观,再这么免费拿也过意不去,下次就谈谈晚饭钱的问题吧。\\

「……那就这样吧。朋友来了你也不好说太久吧。那我就先走了」

「……一直都谢谢了。树那边我不会说是你的」

「请务必」

「嘛,就算我说实话他也不会信的吧」

「也是呢」\\

直接被肯定周也感到心情复杂,不过要是周站在树的立场也不会相信真昼会给周做饭这种事。八成会当作妄想。

毕竟天使大人就是这个程度的高岭之花。\\

如果是高富帅姑且不论,自己这种又挫又懒的人能让天使亲自下厨招待,一般来说算太阳从西边出来都不可能吧。\\

「……可以问你一件事么?」

「什么?」

「每天这样给我做晚饭你图啥」\\

一般来说劳力也是要算钱的,免费给晚饭什么的根本不可能。要是立场反过来周肯定也不会做的吧。

对自己抱有好意什么的,周并没有期待这万分之一都不到的概率,却感到好奇得无法自已。\\

听到周的问题,真昼少许抬起头做出思考的样子,然后表情不变地回答道「是我的自我满足罢了」。\\

「并没有什么特别的哦。对我来说做两个人的份比做一个人的更轻松,另外我也许是单纯地喜欢招待他人吧」

「你喜欢做饭?」

「嘛也是一个原因吧。你能不想入非非地表达好吃的话我也很轻松。而且你那饮食习惯我看着就难受,所以果然还是自我满足吧」

「……是这样吗?」

「就是这样。所以说,不必感到不好意思,就当作天上掉馅饼的幸运就好了」

「好吧好吧」\\

真昼看起来也不想再说下去了,彬彬有礼地致意之后留下一句「那我就先走了」便回到了自己家中。\\

(……真的是这样吗)\\

感觉这理由不至于免费提供晚饭啊,周嘟哝了这么一句,回到了自己的屋子。\\

\vspace{2\baselineskip}

「谁?」

「认识的邻居,分了点吃的过来。我去放冰箱里,游戏别往下打啊」

「啊,抱歉 BOSS 我干掉了」

「喂你过分了啊」

\subsection{天使大人与老套展开}

周和真昼初次发生对话的公园在家和学校之间。

周所住的公寓比起家庭用更适合少人数的居住,因此孩子很少,附近的公寓也是大同小异。

而离此不远处建起的这个小小的公园,酝酿出一股凄清的氛围。\\

就是在这么一个没有小孩子会来玩的,少有人烟的地方——周看见了大概是放学回来路上的真昼。\\

「你在这干什么呢」

「……没什么」\\

在长椅上一动不动端坐的的真昼,看见周微微眯细了眼。\\

这回与以前不同,因为互相认识,周也好搭话,但这么做之后真昼的回答却很僵硬。

并不像是警戒的口气,而更近乎于像是在憋着什么事情不说出来一样。\\

「我说,没啥事的话别一副走投无路的表情坐着啊。有啥事么」

「……没什么……」\\

虽然很在意真昼那如同身处困境的表情,但真昼却没有说出理由。\\

虽说和真昼约定好了,出了家门便互相无关,但现在看着真昼一副困扰的样子不自觉地便搭话了。

或许真昼是不大想周来多管的。\\

嘛,不想说就算了吧——正这么想的周,看向一脸僵硬的表情的真昼,突然发现她上衣上沾着几根白线——准确来说,是几根白毛。\\

「话说校服上有毛啊。是跟狗还是猫玩了么」

「不是玩啦。不过是把树上进退两难的猫救下来了」

「什么嘛就这事。……啊原来是这样啊」

「诶?」

「在那等着,绝对不要动啊」\\

听完了真昼的解释总算搞清楚她为什么要坐在长凳上的周,深深叹了口气,暂时离开了那里。\\

真昼一定会乖乖地停在那不动吧。

不如说她是动不了才准确。\\

这家伙总是在奇怪的地方逞强——周一边感叹着一边去附近的药店买来了湿布,绷带,去便利店买来了咖啡用的冰块,返回真昼那里后她果然还在那里。\\

「椎名,把裤袜脱了」

「哈?」\\

周直接地说到,真昼的回答则凉到了极点。\\

「呃就算你发出那种声音……这样吧,你拿我衣服盖着脱,我转过去不看你。总之先冷却下伤处再贴上湿布」\\

周摇了摇手上提着的购物袋,顺带表明自己没有看人脱裤袜的奇怪癖好,而真昼的表情则明显僵住了。

看来是猜中了。\\

「……为什么会知道啊」

「只有一只脚鞋子半脱着,而且两边的脚踝大小还有微妙的区别。另外还一直没打算站起来。跑去帮助猫却扭到自己脚,真是老套」

「啰嗦」

「好好好。行啦把裤袜脱了,脚伸出来」\\

虽然是一看便知的事情,但看起来真昼没料到周会发现,现在是一脸苦涩。

不过,她老实地接过了衣服盖在了膝盖上,应该是听从了这边吧。\\

周转了个身不去看真昼,把从便利店买来的冰块放进塑料袋里把水倒进去。

周扎上了口子不让水漏出来,然后从包里掏出毛巾包上,现场做出了冰袋,然后慢慢地转过身。\\

真昼则照着周说的脱下了裤袜露出了脚。

没有多余脂肪,感觉上紧致而柔软的光滑脚部曲线,以及脚踝不自然的肿胀,都一览无遗。\\

「嘛,看起来肿得不算严重,但乱动的话估计要恶化啊。总之,可能会有点冷,先把温度降下来。等不那么痛了给你贴上湿布,你好好静养」

「……谢谢你了」

「下次的话一开始就老实地拜托别人嘛。我也不是想着卖人情才帮你」\\

不如说是这边想帮忙解决几件事,稍微还一点那一直累积的人情。

把脚放上长凳,冷敷着脚踝的真昼虽然脸上的表情没变,但没了拒绝周的意思,老实地坐着。\\

「不那么痛了吗?」

「……嘛,好了一点」

「那就给你盖上湿布吧,……别把我当成变态啊痴汉啥的生气啊?」

「我才不会对恩人说那么失礼的话呢」

「那就好」\\

周再次强调了自己没有不好的想法之后,蹲到真昼脚的位置,把湿布贴在红肿的脚踝上。

姑且问了一下有多痛,她说能站起来也能走路,但为了伤情不恶化老实地坐着。总之还算是在轻伤的范畴吧。\\

周贴上湿布,用一起买来的胶带固定上之后,真昼俯视着周发现了这一点。

「意外挺能干嘛」

「嘛,处理受伤还是可以的。虽然做饭不行」\\

周稍带幽默地耸耸肩,真昼则微微地笑出了声。

刚才开始就一直是一脸僵硬,能稍稍放松下也很不错。\\

看着态度稍稍缓和的真昼,周松了口气,从包里掏出一条校服裤子。\\

「给」

「嗯?」

「别用那种表情啊,能看见你脚诶。又不能贴着湿布穿裤袜。这条我没穿过,放心吧」\\

缠上了胶布的脚踝大了一圈,就这么让她穿裤袜也不好。再说为了防止冷到和内裤走光,还是穿上裤子为好。

真昼似乎是明白了周没有别的意思,接过了裤子。\\

确认真昼穿好了裤子之后,周拿过刚才借出去的校服上衣,脱下现在穿在衬衫外面的大衣递给真昼,然后把校服上衣穿上。\\

「给,穿上这个」

「所以为什么啊」

「你想被人看见你被我背着吗」\\

再怎么说也不能让受了伤的人走回去,周一开始便是这个打算。

反正要回的地方也基本没差,周来带真昼回去既效率也对伤情有益。\\

「啊,抱歉能背着我的包么。毕竟背着包就没法再背你了」

「就没有不背我的选项吗?」

「我说,扭着脚了就老实点咯。要是没人就算了,这不有双好脚就好好利用下啊」

「脚吗」

「怎么,喜欢用手么。要我把你横抱着回去?」

「你有抱着我回家的力气吗」

「你当我笨蛋么。……嘛虽说确实没有自信」\\

虽说横着抱真昼还是没问题的,但抱回公寓那就实在有点吃不消。再说太容易引人注目能不做还是不做为好。

周也明白真昼只是开个小玩笑,所以被当成小笨蛋也没有生气,而是笑着觉得真昼还有精神开这种玩笑,这就很够了。\\

「穿好了就戴上帽子背上包。还有,你的包等我背起你来之后再提上,我要背着你没法拿」

「……抱歉」

「没事啦。我作为男人还没丢脸到扔着受伤了的人不管回家」\\

周弯下腰把背对着真昼,真昼便一点点移动身子靠在周背上。

即便套上了大衣,穿了那么多层衣服,碰触到的真昼身体感觉还是弱不禁风。\\

周确认了真昼已经把两手在不至于勒着自己的程度握紧后,慢慢地背着真昼站了起来。\\

该说是果然吧,很轻。

尽管真昼对周说这个说那个的,身体却娇小得让人担心有没有好好吃饭。不过或许也可能是她原本个子就小吧。\\

微微传来的甘甜香味,和着真昼不安地抱紧自己的状况,令周不禁浮想联翩,但终究周还是憋在心里开始往家里走去。\\

背着人这事本身多少还是会吸引路人的目光,但多亏真昼埋着头遮起了自己的脸,并没有被太过注目,算是得救了吧。\\

「好嘞,就这样」\\

把真昼背到了家门口放了下来后,周打算就此收手,走开了。

真昼扶着墙自己能站起来,受的伤应该不算太重吧。万幸明天开始都是休息日,静养几天的话应该能恢复到走路不成问题的程度吧。\\

「今天就不用管我的晚饭了好好静养吧。要不你今天也靠营养补品对付过去?」

「不用了。还有剩下的」

「那就好。再见」\\

不用担心饭的问题真是幸运。能够不需要走动就再好不过了。

看着真昼掏出钥匙打开房门,周自己也摸出了自家的钥匙。\\

「……那个」

「嗯?」\\

突然被搭话,周看向了真昼,真昼则抱着自己的包稍稍害臊地看着自己。

周对真昼微微摇晃的眼瞳感到疑惑。真昼视线彷徨着好像有什么烦恼一样,但最终还是好像下定了决心一样直直注视着周。\\

「……今天真的谢谢你了。帮了我个大忙」

「嘛没事啦,反正是我自己想做的。那么,照看好自己啊」\\

对周来说太过介意自己这边也不舒服便轻松地带了过去,看着真昼行礼之后,打开了自家的门。\\

周突然想起自己的大衣和裤子还在真昼那,但又转念一想反正过几天就会还回来吧,便没有回头走进了玄关关上了门。

\subsection{天使大人与班上的王子大人}

「喂,你小子咋成了这种全年短裤的元气系啊」\\

周一的体育课上,周感到了忧郁。原因在于周不擅长运动,以及落得不得不在这冻人的天气里穿着膝盖长的运动衫的下场一事。

到了这个季节,主流已经是长袖运动衫了。膝盖下面都露出来的周在周围人中间有些显眼。\\

「才不是嘞。忘带了而已」

「你还真是个笨——」

「啰嗦」\\

周末没有遇上真昼,因而没有拿回自己的长裤,才成了这样,但周没法跟树这么说,只能说是忘了。

被笑话周还算能忍着,被树嘿嘿笑着啪啪地拍背时还是还手了。\\

听着树没新意的喊疼,周无奈地轻叹一口气,看向了别处。\\

刚刚他们正在操场上助跑跳高,不过似乎女生也是在用到操场的体育课上,操场上也有女生的身影。再者还是两班合上,操场上的人相当多。

那边是在进行田径类的竞技,她们在等待时间会看我们这边上体育课这种感觉。\\

「门胁君加油——!」\\

体育课男女上课地点一般是分开的,现在女生在使得男生这边嘈杂起来……而女生们看着的是周的同班同学,有名的帅哥门胁优太。\\

周没怎么跟他说过话,但周知道他待人和气、学习也好,而且一年级就成了田径社的王牌选手,在女生间十分有人气。

对周来说不过是想着上天也会造完人啊的想法,但是对其他男生来说这就不那么有趣,有不少男生露出微妙的苦脸。\\

「哦哦——那边好厉害耶」

「是呢」

「没兴趣么」

「嘛,反正实际上没啥关系不是么。就算是同学但也没怎么说过话。怎样都好」\\

反正也没有伤到自己,既然互相没关系,老实说无所谓。\\

周尽管理解自己的想法是少数派,不过还是不至于跟其他男生一样到嫉妒的程度。

不如说他啥都那么完美反而觉得连嫉妒都没了意义。\\

「周你一直是这样嫉妒心很淡呐」

「怎么,要我说『如此受欢迎真是羡煞我也』么」

「你不是那种设定吧」\\

周斜看着咯咯笑着的树,望向一脸沐浴在女生声援里一脸爽朗笑容的门胁。\\

以男性的视点看也是体型均整,相貌帅气,简直就像是王子大人。因为一眼望去找不到可以算作缺点的缺点,实际上他的外号就叫王子。\\

面对女生们热情的眼神和高亢的声音,优太微笑着摇摇手回应,真是个善于社交的人啊。\\

「哎呀,王子大人这么熟练真是厉害啊」

「是呐。那笑法我可做不出来」

「女生们也很兴奋啊」\\

树的话有爱得不行的女朋友千岁在,对其他女生基本就是毫无兴趣的样子。
千岁看起来也是对门胁毫无兴趣的样子,想必树对门胁也是没什么想法的吧。\\

(王子啦天使啦,这学校还真是不少外号超羞耻的人在啊)\\

说起来天使大人,也即是真昼,她有好好静养么。

周末似乎没有出门了的样子,应该是安心养伤了,就是不知道伤好得怎样了。\\

正好另一个班就是真昼的班,周朝那边扫视过去,在操场边上望见了即便人很多但还是容貌很显眼的少女。\\

她没换上体操服,也不在上课的人群里,大概是在观摩吧。

静静地站着的真昼吸引了许多男生的目光。\\

虽然距离很远,但周和真昼对上了眼神,周尴尬地偏开了眼神,而真昼则微微露出了笑容。\\

而因真昼这笑容对着周——不如说是对着男生们的集团,同学们说着「刚才她对我笑了?」「才怪是对我吧」骚乱了起来。\\

「这可是好机会啊,得向椎名同学展现自己吸引她」

「哪能让王子把好处全占了」\\

微微一笑便引起这么多反应,该说是厉害还是说只是他们太单纯了呢。\\

「……真是单纯啊」\\

树似乎也想着同样的事情,嘟囔了一句。周也不禁笑了出来。\\

「嘛既然关系到学分我们也差不多努力下吧」

「咋了,周你被天使大人看了也要展现展现么」

「才不是啦。不是说了没兴趣么」

「嘛,也是。你还真是对啥都没兴趣啊」\\

周随意回答着又开始嘚瑟地说起女朋友的好的树,再次看向真昼那边,露出苦笑。

\subsection{天使大人的慈悲?}

「前几天真的谢谢你了。这是你借给我的大衣和运动衫」\\

这一天,一如往常送来晚饭的真昼,除了饭盒以外还带来了一个纸袋。

纸袋里隐约能看见的是上周五借给真昼的大衣和运动衫吧,是好好叠整齐放进去的。\\

「嗯。伤怎么样了?」

「已经基本上不痛了。现在痊愈之前暂时不会做剧烈运动」

「那就好。体育课记得也是在旁边看着的吧」

「是的」\\

以防万一,真昼体育课在旁边观摩,这是正确的选择吧。虽然看上去不怎么痛了,但她走路的姿势还是微微看得出在照顾那只脚,应该还并没有完全痊愈。\\

周认同地点了点头,突然回想起体育课的场景笑出了声。\\

「嘛不过说起来啊,天使大人的人气还真是高啊。一个微笑就能让全体男生干劲高涨呢」

「所以不是说了不要那样叫我啊……。我是很困扰啦,至于这么高兴么」

「嘛,美人朝着自己露出笑容当然是这样了啊。你想门胁朝女生招手的时候,她们不也一样咿咿呀呀叫起来了么」

「……门胁……啊啊,那个很受欢迎的人吗」\\

真昼一脸无趣的样子——不如说可能确实不感兴趣,光听到名字还记不起来,听周的解释才想起这个人来。\\

虽然不及天使大人,但门胁在年级里也算是十分有名的男生,因此真昼光听到名字不知道是谁让周很是意外。\\

「你没有兴趣吗?」

「没啊。毕竟班级不同,也没有什么能扯上关系的事情」

「诶。其他女生倒是挺感兴趣的来着,天天说着什么好帅啊」

「毕竟长的好看嘛。我没跟他说过话也没有跟他有什么关系,所以无所谓了」

「你对这种事情倒是很无所谓嘛」

「要是就因为美丑而对对方产生好感的话,你怎么对我没点好感?」

「哦哦,原来知道自己长的很可爱啊」\\

真昼的话确实在理。

因为长的好看而对对方抱有好感确实可以是个理由,但并非是只因为好看便会喜欢上对方。\\

周同意这话,也承认真昼是个美少女。虽说她本人居然也承认让周有点意外。\\

「身边那么不得安宁的,我再怎么说也知道了。而且,客观看我也明白自己长的还算可以,打理自己也没有偷懒」\\

那是当然——真昼这么说道,但她完全没有表现出自大的态度。\\

实际上真昼也是为了美貌用尽了手段的吧。\\

真昼的容貌本身便很端整,但她也没有就此满足。

头发似乎真若天使这一外号般几乎能看见光环。肌肤也十分完美,毫无褶皱黑斑。尽管在做家务,双手也并未因此变得粗糙,而且指甲也好好修剪过。

当出则出,当隐则隐,能把身体打理的如此匀称,想必不是一朝一夕的功夫就练成的。\\

「确实啊。你这样毫不嘚瑟地说着事实,那就算我夸奖你你也不会高兴吧」

「别人太过奉承的话,我只会先觉得烦人哦」

「还真是麻烦啊,当个美人」

「也有相应的回报所以也并非都是不好的事情呢」

「……真是一副事不关己的语气啊」

「怎么,难道要我害羞地回答『没那回事』更好么」

「别别别,这跟你平常画风不一样违和感太强了」

「就是说嘛。我也觉得,对你摆出那副样子也没什么意义」

「是呐」\\

真昼这毫不掩饰的说话风格现在就算改掉周也会觉得奇怪,要是真昼照着在学校里的样子对待周周也会微妙地起鸡皮疙瘩,所以还是保持这样的为好。\\

习惯可真是恐怖啊。要是学校的天使大人举止像天使一样反而会感到奇怪。

周意识中的真昼已经完全是现在面前的这个真昼而非学校里的那个了。\\

结果两人得出的结论是现在这样就好,周于是看向了递给自己的饭盒。

比平时更大的饭盒里装着好几样的各式菜品。比起分赠已经更像是送来一份便当了。\\

「今天挺豪华的啊」

「毕竟受了你的关照」

「不是说了不用在意的么……哦哦,居然还有可乐饼啊」\\

可别小看这可乐饼。

虽说作为下饭菜可乐饼是十分常见,但自己做起来却十分麻烦,但自己做起来,在家庭料理里面算是最麻烦的。\\

把土豆蒸完炒好,配上牛肉啦洋葱啦做出形状,冷却之后再裹上面衣下锅炸……如此这般一堆普通却很麻烦的工序。

基本不做饭的周只是看见母亲做这个的工序便心想着这绝对麻烦死了。

因此,让母亲做的时候也经常不情不愿的。\\

「嘛虽然只是把做好的东西冷冻下然后拿去炸就好了」

「所以顺带做了炸鸡吗」

「是呢」\\

一人独居炸物基本上就是吃些零食一类,能得到这样的手制品简直是感激不尽。

要是再贪心一点的话,周还想要在刚炸出来面皮还酥脆的时候和饭一起吃。\\

「……偶尔也想吃点刚出锅的菜呢」\\

真昼考虑到卫生上的原因,都是把东西先凉一凉再放进饭盒里,因此吃之前还是要再次加热的。虽说用烤箱可以还原那种感觉,但还是比不上刚出锅的味道。

当然这样也十分美味,但刚出锅的感觉还是要好上不少吧。\\

周不过是把愿望说漏嘴了,没什么别的意思,但可能是自言自语说得太清楚,真昼听见之后稍微皱起了眉头。\\

「你意思是让我去你家?」

「我可没那么说啊,都已经分我饭菜了,再那样也太没分寸了」\\

周为了摆脱莫须有的怀疑耸耸肩明白地表达否定,真昼则用手撑着下巴低着头往下看。

似乎是在考虑着什么,没有和周对上眼。\\

「……一半」

「嗯?」

「伙食费各出一半,我可以考虑在你家做饭」\\

终于开口的真昼,说出的话威力大得让周不住目瞪口呆。\\

虽然只是玩笑或者说是不小心说漏嘴的念头,结果真昼认真考虑之后还答应了,不禁让周不知所措。\\

一般来说,会有人想跑去关系不算亲密的男性家里做饭吗。

就算这样更有效率,对方毕竟是异性,而且关系并不至于亲密无间。按理肯定会觉得不安的。\\

「各出一半与其说正和我愿不如说我这边得到的太多了所以完全没问题……你不觉得危险么?」

「要是被做了什么锤烂就好了。在物理上,烂到再起不能」

「哎哟好怕,瑟瑟发抖」

「再说,就算我不那么做,从你的风险上考虑你也不会做的吧。你很清楚我在学校里的立场吧?」

「如果做了什么我肯定完蛋了啊」\\

周与真昼压倒性的人望差距加上真昼作为柔弱女性的事实,要是真昼说自己要被周做些什么那周就真的没法去学校了。\\

周还没有蠢和没有节操到即使知道社会性死亡的结局还去做什么。

不如说周自己本身就没这个打算。\\

「再说了」

「嗯?」

「你看上去对我这种类型也没啥兴趣」\\

真昼一脸认真的断言令周不禁苦笑。\\

「要是我就喜欢你这样的呢?」

「那你估计就会不厌其烦地跟我搭话,然后我就会跟你拉开距离吧」

「那我算是被认可了咯」

「嘛,至少觉得你很安全」

「那还真是谢谢了」\\

这样就行了么——尽管周这么想着,但他完全没有打算对真昼做什么,所以没有否定。\\

于是乎,自然不会放过这千载难逢的能够享受新鲜的极上美味的机会的周,在收下「无害的男人」称号的同时,获得了共进晚餐的权利。

\subsection{围裙与手制料理是男人的浪漫}

真昼同意在周家里做饭的同时,提出了如下条件。\\

\begin{itemize}
    \item 周出材料费的半数加上若干人工费。
    \item 如果有事不能一起吃饭至少提前一天通知对方。
    \item 食材的采购和饭后的处理由两人分担。
\end{itemize}

关于人工费,算是真昼拗不过坚持表示要补偿为自己花费时间的周做出的让步,而其它部分则十分顺畅地决定了下来。\\

关于由真昼来做饭这事已经是理所当然了,并没有什么大问题。\\

于是乎这么决定后,次日,真昼便早早地拎着——准确来说是两只手抱着购物袋过来做起了下厨的准备。\\

「……还真是新得几乎没有好好使用过的痕迹啊……」

「啰嗦」\\

明明正处于家中有穿着围裙的女性这一男人的浪漫具现一样的状况,可周却不知为何却感到如坐针毡。\\

将头发扎成一束的真昼带来的新鲜感也是原因之一,但果然主要还是因为厨房基本就没使用过,这点被真昼再次指出造成的尴尬吧。\\

「明明有这么多好东西却被放着吃灰」

「你能用上的话那不就不吃灰了么」

「那是结果论吧。多好的厨具啊,怀才不遇的都要哭了」

「那就用你擅长的料理让它们破涕为笑吧」\\

周干脆地表达自己不行,真昼则一脸无语地看着他,但也许是料到如此,只是叹了口气而并没有继续啰嗦。\\

「那么,做饭用的调味料有么」

「有啊,你当我傻吗。保存方法和保质期我都好好注意了的」

「哎呀真是意外」

「因为都没开封呐」\\

大部分调味品都是没有开封,放在阴凉避光的地方,应该不必担心吧。

这些买来的好东西得不到展现自己的机会,不如说因为周基本没下过厨房,结果都没使用过。对调味料来说,能被真昼这般的厨师使用,也算是物尽其用吧。\\

「这可不是什么好嘚瑟的事情。嘛,要是不够的话我回家拿一趟来用就好」

「那就好」

「总之既然有基本的调味料的话多少是能做出点东西的。啊,今天的菜单就我一个人定下来可以吧」

「我也不太清楚这些东西,能吃的话什么都可以的。我不怎么挑食」

「这样啊。那我就动手了……调味料放在哪了告诉我一下」

「都放在这个篮子里」

「……还真是都没开封呢……」\\

真昼瞄了一眼塞满了调味料的篮子,无语地皱了皱眉,但因为周事先说过,马上便恢复了表情,在水龙头旁边开始洗手。\\

「那我就开始做饭了。你就在客厅或者房间里等着就好」

「行。反正我也帮不上忙」

「你这还真是,该是干脆呢……嘛算了。要是你不会料理还晃悠来晃悠去的我也很难办」

「你也很直接啊」

「反正是事实。跟你说也没必要拐弯抹角吧」\\

正如真昼所说,自己显然是个累赘,周便老实地走回客厅观察真昼的背影。\\

洗完手的真昼迅速投入到调理工作中。\\

虽然不知道要做啥,但从材料看应该是日式的。

如此美味的料理竟是在自己家里做出来的,不禁让人感到不可思议,怀疑是不是在做梦,但真昼正摇晃着背后扎成一束的秀发处理着食材却是不争的事实。\\

(……咋说呢,感觉跟有了老婆一样)\\

尽管两个人互相并没有这样的感情,然而目前的状态看上去实在像是成了家一样,让周不由得心生想象。\\

周自然是对真昼没有一丝一毫的非分之想,不过有个美少女在自家厨房这个状况本身就足够让人浮想联翩了。\\

果然,就算不论好感,仅仅是可爱的少女正在做饭这一场景,便足够让周有种莫名的感动。\\

「……你不会在想些乱七八糟的事吧?」

「别瞎猜啊」\\

真昼没有回头的突然发问让周脸抽筋了一下,但也万幸真昼没有回头让周不致败露。\\

这家伙可真是敏锐啊——周既佩服又感到发凉,收起了微微涌起而不至于邪念的男心,继续观察着真昼的背影。

\subsection{天使大人与至高的料理}

大约过了一个小时,饭桌上开始排起了一盘盘料理。

由于是真昼定下的菜单,故而桌上的都是符合真昼健康追求的和食。\\

「这边的厨具和调料也算挺够用,看来是不用我回家取了。明天开始就能做精致一些的菜了」

「哎呀只要有东西吃我就感激不尽啦」\\

因为不清楚有多少厨具和调料能用,比起精致的菜肴更多是简单的东西,但即便如此,色彩和摆盘也可称得上是完美了。\\

青菜煮鱼,味噌煎蛋等等,各种对周来说连想都不敢想要去做的和式菜色并排摆在桌上。\\

在不怎么挑食,但基本上还是喜欢和食的周心里,甚至都想对稍稍抱有歉意的真昼说自己想要的就是这个了。\\

「……看上去超好吃的」

「这么夸我我是很高兴的啦。趁热开动吧」\\

真昼这么说着坐上了椅子,而周则坐在了正对方向的椅子上。\\

单人生活准备的餐桌尺寸偏小,不论怎么坐两人靠得都很近。

还好备了两把给客人用的椅子,但面前便坐着一位美少女还是给人一种难以名状的感觉。\\

不过,一旦开动起来,真昼的美貌什么的也都无所谓了。\\

例行示意开动之后,周首先尝了一口味噌汁。

周享受着在嘴唇碰上碗沿那一刻,味噌与高汤的香气。慢慢地将其含在嘴里,与那香气相称的味噌与高汤的风味便在舌尖散开。

这种与速食味噌汤完全不同的柔和口味,肯定是经过了反复计算调整的吧。\\

味噌不太浓,咸淡上也保留住了高汤的风味。\\

略显清淡的第一口,应该是考虑到了要和其他料理一同食用,这样的味道在喝完的时候恰好会觉得浓淡适中吧。\\

与其说是有什么不足,不如说是让人安心的,引起品尝米饭和其他菜品欲望的味道。\\

「好吃」

「谢谢夸奖」\\

周直白地表达自己的感想,而真昼则放下了心,微微眯细了眼。

尽管平常一直在夸她做的菜好吃,但是当面说还是会紧张的吧。\\

看着刚刚一直在关心着这边反应的真昼也开始动起了筷子,周也向着菜品伸出了筷子。\\

周甚至想要把摆着的吃的一齐包进嘴里,可见真昼的料理实在是美味。\\

煮鱼好好地入了味以及保持了肉中的水分。

为了做到入味而长时间加热的话,水分就会流失,肉质就会变干巴巴的,但真昼做的煮鱼肉质却十分鲜嫩,口感很好。\\

煎蛋卷的调味则是正中周的喜好。

被表面鲜艳的金黄色引诱,周尝了一口,舌尖传来的果然是高汤那柔和的风味。

煎蛋卷有加糖和仅盐这些各种各样的派系,真昼做的则是加入了高汤略带甜味的蛋卷。\\

隐约而柔和的甜味应该是蜂蜜吧。

放的量应该并不是很多,留有余韵的甜味增加了味道的深度。\\

当然不论是甜味的还是咸味的煎蛋卷周都不讨厌。

不过,周最喜欢的还是这种加入了高汤略带甜味的调味精致的煎蛋卷,如今吃到这理想中的蛋卷,周甚至有些感动。\\

好吃啊——周以谁都听不见的音量轻轻感叹了一句,然后继续吃着菜。

火候的调整也是绝佳。周咀嚼着这饱含汤汁的口感鲜嫩的煎蛋卷,静静地享受着这美味。\\

周一边默默想着「确实比我妈做的还要好吃啊」这对不在现场的母亲有些失礼的事情,一边幸福地大快朵颐。接着,周突然注意到真昼正盯着自己在看。\\

「……看起来吃得很香呢」

「实际上也很好吃嘛。面对美味应该要抱有敬意不是么」

「嗯,这倒是」

「而且,比起板着个脸吃,还是这样坦率地表达好吃,两边都开心吧?」\\

就算料理十分好吃,不从表情上表达出来的话,制作者也会感到不安和在意。板着个脸的话就算说好吃也会让人怀疑到底是不是真话。\\

比起那样,不如坦率地把自己的感受表现在脸上,对双方都有好处。毕竟不管是感谢还是被感谢的人,都喜欢有个好心情。\\

「……是呢」\\

真昼似乎是接受了周的解释,微微露出了笑容。

如同松了口气般的,表达着安心的柔和笑容,其可爱程度,甚至让周的大脑有一瞬间一片空白的程度。\\

「……藤宫?」

「啊,……呃没什么」\\

看到入迷了——不可能如此坦白的周,为了掩饰自己渐渐涌起的羞耻感,继续吃起了晚饭。\\

\vspace{2\baselineskip}

「……我吃饱了」

「招待不周」\\

将摆在桌上的饭菜一扫而空的周满足地表示自己吃饱了,而真昼则淡淡地回应道。

不过,真昼温和的表情,似乎是在表达看到周这样将饭一点不剩的吃完感到的喜悦吧。\\

「很好吃啊」

「看你的样子就知道了哦」

「比我妈做的还好吃耶」

「把女孩子亲手做的料理跟老妈做的比较好像是禁忌哦」

「那不是贬低的时候的说法么?话说你很在意?」

「我倒是不在意呢」

「那不就得了。反正好吃的事实也不会变」\\

真昼的厨艺可不是光靠一点点下厨的经验便可以达到的程度。\\

周母亲虽然有常年的下厨经验,但在调味上和周的喜好不同,而且还很随便,自然比不过经过仔细地计算调整的真昼的调味了。\\

不如说在做饭上连父亲都比母亲更加擅长,更不用论跟真昼比了。\\

「……哎呀感觉我是不是太幸福了啊。每天都能吃到耶」

「虽然只有在我们都没事的时候才行呢」

「……话说,每天一起吃饭真的好吗」

「不好的话我也不会这么提议了」

「嘛虽然话是这么说啦」\\

周也十分清楚像真昼这种直率的人要是不喜欢的话一开始就不会这么提议,但还是不禁烦恼这样到底好不好。

虽说周付了一半的材料费加上人工费,但还是不禁担心真昼的负担会不会太大。\\

「……我说,一般来讲,你会给谈不上喜欢的男的做饭吗?」

「还不是因为你生活太不健康了吗。再说,我很享受做饭这件事本身,看着你吃得津津有味我看着也不会不高兴」

「但是啊」

「……你要是实在在意,我其实不给你做也无所谓的哦?」

「别别别还请你务必做上我的份」\\

周反射性的回答,也表现出对周来说真昼的料理是如此地必要和享受。\\

事到如今,要是把真昼的料理给停了,对周来说那可就真的算得上是死活问题了。

周对自己的胃已经被抓住一事早有自觉,不过问题是真昼的料理实在美味。照这样下去眼看就要回到每天小菜就饭的无味日子,想想都可怕。\\

听到周那易懂的回答,真昼一张实在无奈的脸上露出了似是苦笑的表情。\\

「那就请你好好收下吧」

「……哦」\\

想到与这大慈大悲的天使大人共进晚餐的日子还将继续,喜悦、罪恶和期待感在心中交织的周不住地叹了一口气。

\subsection{天使大人成绩也很完美}

「周~,考得怎么样?」\\

期末考试终于结束,总算熬过了地狱考试的学生们,比平常更加兴奋地在教室里聚成了几团。

周和树也是一样因为考试结束而松了一口气,评判着自己这次的发挥。\\

「嗯?一般吧。差不多还行」\\

听到发问周虽然做出了回答,但其实并没有什么可说的。题目都在考试范围内,只要平时做好复习的话这场考试并不算难。

这次写起题来的手感跟以前并没有什么不同,所以周也没有什么特别的感想。\\

周虽然是个怕麻烦的人,但基本上还是会好好地做好复习。

上课学的内容大致都懂了,考试也发挥正常。虽然满分还是有些难,但八九十分还是没问题的。\\

「然后你年级前三十稳了是吧……你个学霸」

「靠平时习惯啦」

「就你那平时习惯你还有脸吹?」

「再怎么样也轮不到你这个天天秀恩爱不读书的家伙讲」\\

论起造成周和树成绩差距的原因,比起聪不聪明,主要还是树在女朋友千岁身上花了太多时间。

树脑袋也不笨,要是认真起来的话应该也能拿个挺不错的名次,只可惜树把时间都花在了女朋友千岁身上,结果成绩就比不上周了。\\

「……女友可是个好东西哦?」

「对对对对」

「我说啊周,你也去找个咯」

「想有就能有那这世上男儿们也不会流下血泪了啊」\\

这世上想要女朋友而求之不得的人比比皆是,对某些人来说树这句无心之语听上去想必是十分地扎心。\\

不过周倒是并没打算对树生气,说到底现在根本也没有想要个恋人的欲望,于是只管听过便算了。\\

「再说,女朋友咋找啊」

「来个双重约会——」

「然后我和那个幻想的女朋友就会被你俩秀到闪瞎吧」

「那你们也秀啊!」

「你觉得我这性格能干出那种事」

「……看样子不行」

「嗯哼」\\

周也对自己这淡泊的性格有所自觉。

周的性格怕麻烦而且说话直来直去,有些人可能会觉得冷淡,因而给人的印象不算很好。这性格根本不是能找到女朋友的那种。\\

就算万一有了女朋友,关系想必也会很平淡。至少不会像树那样大庭广众狂撒狗粮。\\

「不是我说周你至少找个喜欢的咯。话说啊,周你要是剪掉点前发,弄清爽点,整整发型,背挺直了,女生们绝对会刮目相看的」\\

周自认为对自己有正确的评价,即便达不到门胁那种帅哥等级或是树那种稍显轻薄的端整外表,周也觉得自己的外表绝对谈不上丑。

要是周好好打理打理自己的仪表和形象的话,也是有不输同龄高中男生的水准的。\\

不过,周没有在穿衣打扮上考虑到别人感受的能耐。\\

「光凭外表就来套近乎的可都不是什么好货色哦」

「说是这么说,可要是对方对你没兴趣,你也没法了解对方的性格吧?」

「……就算是那样,我现在也没找女朋友的心思」\\

就算这么着找到了女朋友,看见周平常的样子肯定也会幻想破灭吧。

周这人生活不能自立,日子过得邋遢,而且对人还不友好。不如说要是有女孩子对自己感兴趣我倒还真想看看——甚至周自己也如此苦笑道。\\

毕竟周是抱着天天考虑着别人多烦啊的想法,性格上就不适合和人交往,因而并没有想要女朋友的想法。\\

毕竟周心目中的优先度是真昼的料理>还没找着的女朋友,而且这个优先度恐怕没法轻易改变吧。\\

「真是个没欲求的家伙……要不让千给你介绍几个朋友也行哦?」

「你可别瞎操这闲心。千岁她朋友都是群吵闹的家伙吧,光是当朋友怕就够让我头疼的了」

「毕竟周你是个阴暗角色嘛」

「是啦咋地」

「嘛,你要这么说那暂且就算了吧。不过啊,美妙的高中生活,连女朋友都没有,一个人空虚度日,不难受么?」

「没意思还很麻烦啦」\\

虽然周并没有「你把学校生活当什么了」这种较真的思考,但反正女朋友这东西不是非要不可,所以周并没有想着去找一个。

再说了,喜欢的人不是那么好找的,也不容易产生结果。\\

「……可惜了啊」

「随你便吧」

「不过啊,周你要是有了喜欢的人是会变的哦?」

「你哪来的自信啊」

「就是你这样的家伙,宠起女朋友才会不要不要的」

「好好好你说啥都对」\\

觉得自己变成那种甜得发腻的人什么的无法想象也绝不可能的周,把树的话当成耳旁风就这么吹过去了。\\

正一脸悯然地看着周的树……忽地移开了视线,表情也舒缓了下来。\\

「\ruby{树~}{\jpfont いっくーん},回家吧?」

「哦,\ruby{小千千}{\jpfont ちぃ}啊」\\

正好,树他女朋友千岁过来了。因为两人似乎约好了一起回家,周就在树等着的时候陪他聊天了。\\

周回过头,便看见一位一头亮茶色短发,带着男孩子气的少女,正以满面笑容朝着这边——准确来说,是朝着树招手。

那活泼的气氛和明快的笑容,甚至让看着的周感到有些耀眼。她的性格也正如外表,为人友善,活泼明快,好也好坏也好,她都是身为热闹气氛负责人的少女。\\

与真昼风格不同的美少女千岁满脸笑容地跑向这边。

周希望着她能就那样别说话,因为,千岁一说起来基本上周都会被欺负。\\

「千千你说是不,周这样的家伙,肯定会宠女朋友的」

「别多嘴」

「唉?什么?周有女朋友!?」

「有个毛啊」

「哎,什么嘛。有的话我还想打好关系呢」\\

「切」的一声,千岁瘪着嘴一脸失望。\\

「我那幻想的女朋友要遭一波你那美其名曰打好关系的过分身体接触的罪还真挺可怜」

「唉,原来你有虚拟女友吗?」

「我是说假如有的话好吧!?」

「玩笑啦玩笑」

「应付你可真够累人的……」

「只是周你体力不足吧」

「是体力连着精神力全被你消耗掉了啦……」\\

比起体力,感觉累的还是精神。\\

本来平常周过着的就是除了熟悉的人以外基本上不说话不起眼没精神的学生生活,要被迫跟千岁这种全天精神高涨的生物对话,实在是艰难。\\

即便周的回应有些刻薄,千岁也毫不在意,对着一脸疲劳的周说着「真是不像样呢」,十分愉快地笑着。\\

树也同样笑着说着「你赶紧习惯啦」这样没啥意思的建议,因此周除了累得叹口长气以外毫无办法。\\

\vspace{2\baselineskip}

「……在干啥呢?」\\

周回到家吃完真昼亲手做的晚饭之后,洗碗回来就看到真昼在客厅摊开了试卷。\\

洗碗这事是轮班,但周为了不给真昼添负担抢先去洗了,因而这段时间真昼便在客厅里待着。本人云就这样把事情全部扔给周自己回去,有些过意不去。\\

「给卷子算分」

「嘛看也知道」\\

大概是在检查答案,真昼看上去是在对着课本确认有没有写错。\\

「话说结果咋样」

「如果答题纸上我没有写错的话就是满分了呢」

「只能说不愧是你啊」\\

真昼满分的回答太过平淡,让周的也没有什么太大的反应。\\

毕竟已经好多次在月考排名上看见真昼那雷打不动的年级第一,周也不吃惊了。\\

周本来就觉得真昼的话说不定能做到,因而听见满分也只有果然啊之类的想法。\\

「学习我不讨厌啊。再说实际上我已经提前把明年要学的东西全部学过一遍所以现在只要复习就好了」

「呜哇,恐怖如斯。不愧是学神……」

「藤宫你学习不也挺上心的么」

「你还知道我成绩啊」

「名次靠前,有贴出来的那些人基本都记得哦」\\

看来在搭话之前她就已经知道一部分周这么一个人了。

本以为排不到个位数的人根本就进不了她的眼,不过真昼却不假思索地说出了周上次的排名,看来她还挺关心成绩表的。\\

周会用上一定的功夫学习,其原因,并非是学习是学生的本分……这种较真的脑回路,而只不过是家里给出的条件罢了。\\

「嘛,毕竟是让我独居的条件嘛,保持成绩这事」\\

在家里同意周一个人住的时候,提出了要保证成绩不下滑的要求。

另外还有半年回家一次这个条件,不过这边只要趁放长假回一趟就行,所以基本上只要保持住成绩家里就不会多指手画脚。\\

「我也就保持不出麻烦程度的成绩罢了,比不上你。你是超努力的吧」

「……我的话,是因为不努力不行啊」\\

真昼低着头,轻声嘟哝到。

虽然她的表情被垂下的前发遮住而看不太清,但肯定并不算开心吧。\\

不过,真昼很快便抬起了头恢复了平常的表情,周错过了指出这事的机会。

就算有机会,周也不会去问的吧。毕竟那氛围,就是在忍耐着什么痛苦的事情一般。\\

时不时地,真昼就会露出这样的表情。

虽然真昼从来不会把自己痛苦和厌恶的原因说出口,但给人的印象便是被什么束缚,挣扎着的样子。\\

不难想象,变成这样的原因是家庭环境。\\

因此,周来插嘴干预是不合适的。

周十分明白这是自己这个局外人不应踏入的区域,因而一直保持着作为邻居的距离感,避免触及那个部分。\\

周也同样有不想他人提及的东西。\\

周也常常切身体会到,干涉他人私事十分失礼,反而是装做浑然不知更加令人感谢。\\

「我差不多也该告辞了」收起刚才的氛围的真昼,以平日里清爽的声音说道,同时开始把课本和试卷收进包里。\\

周也不打算挽留,「噢」地简单应着,望向收拾东西的真昼。

在真昼把拿出来的东西全部收拾好,从座位上站起来的时候,周突然注意到,在空杯子的阴影处,放着一件不属于周的东西。\\

周伸手拿起来,看清了这是每个学生都有的装着学生证的塑料套。

估计是把课本拿出来的时候一起拿了出来,整理的时候却忘记了吧。\\

周看着这印着正面照、姓名、学号、出生日期和血型这些简单信息的学生证,喊住了正在玄关穿着鞋子的打算回去的真昼。\\

「落下了哦」

「啊,抱歉让你特意送过来。那么,晚安」

「晚安」\\

周目送着礼貌地弯腰行礼离开自己家的真昼,静静地叹了一口气。\\

回忆起刚才看见的学生证上写着的出生年月日——特别是月和日的部份,周扶住了额。\\

「……这不就在四天后嘛」\\

念着那要是没看到学生证的话恐怕一直都不会知道的真昼的生日,周想着要是早些知道的话就好了,再次深深地叹了一口气。

天使大人想要的东西

「话说,你有什么想要的东西没」

次日,想着事不宜迟的周在晚饭时向真昼发问了。

其实说是生日礼物,但周也没有什么特别的意思。周的打算是兼作平常受了那么她多照顾的回礼,打算挑选个适合这个目的的礼物。

不过,周的问法听起来怕是十分的奇怪。
连周自己也感觉这问法既不委婉还很粗神经开始有点后悔,而真昼则一脸惊讶地看了过来。

「为什么突然问这个」
「总觉得你一副无欲无求的样子,好奇所以问问」
「还是很突然……」

虽然周自己也想要是再稍稍糊弄一点更好,但说出去的话已经没法收回来了。

不知算不算幸运,真昼并没有注意到周在想着生日的事情。
退一步讲真昼应该也没有料到周会知道自己的生日,所以根本就没往那边想吧。

「嗯。要说需要的东西嘛。现在想要的的话」
「想要的的话」
「磨刀石呢」
「……磨刀石?」

因为得到的回答完全超出了预想,周下意识地追问道。

不如说,恐怕谁也料不到问女高中生想要的东西会得到这样的回答吧。

一般来说想要的应该都是化妆品啊装饰品啊包啊一类的东西才是。居然冒出想要研磨金属的工具什么的,周实在是没有预料到。

「嗯。磨刀石。虽然我已经有几块了,但果然还是想要目数细一些的,结尾时用的磨刀石呢」
「喂我说现役女高中生」
「请你不要在我身上寻求一般女高中生好嘛」

被这么说,周一时语塞。

就算是往好了说,真昼也称不上是个一般女高中生。
光是被称作天使就已经可见一斑,他不但文武双全,甚至连做饭和家务都不在话下。

这个那个地忙活着照顾生活邋遢的周的时候那个努力少女的样子,看上去已经可以被误认为是主妇了。虽说嘴上有点不饶人。

(所以说就算这样也想不到会是磨刀石啊)

想要的东西居然是磨刀石的女高中生怕是除了真昼就没有了吧。

「……你自己不能买吗」
「其实不是不能买啦。不过,基本用不大上,而且价格还不便宜,所以就没有去买了。再说论想要其实我已经有一个睁一只眼闭一只眼的话也算是能完成收尾的了,所以也并不是很必要吧」

随意地说着自己手上有好几块的事情,真是令人担心将来。

「……自己磨刀的女高中生啊这实在是」
「其实也是有的哦」
「就算有我认识的人里也就只有你一个,而且想要的磨刀石的也就只有你了」
「听起来很稀有,挺不错的嘛」
「到底哪不错了啊……」

实在太稀有以至于对她的喜好周实在是摸不着头脑。

原本的目的泡汤了的周已是黔驴技穷,而真昼则是一脸不可思议地歪着脑袋。


「话说啊,树」

就连对真昼想要的东西的倾向都一无所知的周,只好出此下策,跑去问问树来当参考。

既然千岁是他女朋友那么他也应该多少搞得明白女孩子的心思,像是一般女孩子喜欢的东西也该有个大概的把握吧。
虽说周不知道能不能把真昼当普通的女孩子对待,但他推测女孩子会喜欢的东西真昼应该也不至于讨厌。

「咋啦」
「树你给千岁送礼物的时候都送了些啥」

周想着从树给女朋友送了啥开始问应该可以,便这么发问,可却遭了树的白眼。

「诶,你小子对谁有意思想要送礼啊」
「你看我像会做那种事么」
「不像」
「那不就得了」
「那你干嘛问啊」
「认识的人过生日,参考下」

别说是参考,周甚至都想照着就去买了,不过周也没打算明说。

「哼~嗯。要说的话还是对方想要的东西最好啊。话说你平常就该调查下啊这事情,这可是拉进关系的秘诀啊」
「不是说了不是女朋友么」

假想下真昼成了自己的女朋友,周的身体就本能的觉得很危险(主要是身边的杀气)。再说这事本身也是癞蛤蟆想吃天鹅肉。

确实真昼在身边感觉很自在,但那只是两个无欲无求的人感到志同道合罢了,谈不上恋爱感情。
当然周也觉得她可爱,但凭心而论周也没有跟她发展到有些这样那样的关系的打算。

「想要的东西啊……要是不大清楚呢?」
「那就看关系了。要是关系好,送些饰品也不错,如果关系没那么亲近送些小物件或者消耗品也大致OK。花什么的经常对方收到也很难办」
「……你还真了解啊」
「嘛多少学习过啊」

树和千岁并不是那种一见钟情然后就相思相爱的关系,好象是初中的时候慢慢拉进关系的。上别的初中的周不清楚详细,但据说是克服了不少事情最后才互相交往的的样子。现在周还经常听到树他一边回忆一边炫耀。

「另外,护手霜应该也还不错」
「护手霜?」

听见意外的选项,周不禁重复到,而树则一脸得意地笑着解释。

「不管哪个年龄层都用得到啦。学生的话上课天天碰课本手容易干,工作了的打字吹空调手干也是常事,家庭主妇的话手泡在水里干活也容易变粗糙。总之作为礼物总能派上用场」
「额……你太秀了吧都冷场了」
「还不是你来问我的」

啪―地,周的背被树拍了下,不过两人都知道这不过是玩笑,互相笑了起来。

(护手霜么)

确实,这东西的话应该没什么问题的吧。
虽说晚饭后洗碗的工作周全部包了,不过真昼在自己家肯定还会洗东西,难说手会不会变粗糙。
要说的话看她那滑嫩的双手,想必平常也都会买这种东西保养,那么作为礼物送给她应该也不坏。

「嘛。那我会参考的」
「对了一会你也去问问千千她吧。同性的话应该会有些主意」
「……呃―」
「好啦你差不多也该习惯啦」

当然谈不上讨厌但千岁这种性格的人周实在是应付不来,因而被叫去找千岁周的态度有些犹豫不决,而看着这些的树则是一脸愉悦地笑着轻轻拍着周的背。

\subsection{天使大人与生日}

向树和千岁寻求完建议后,总算选好了礼物的周,在生日当天以一副易懂的紧张表情看着真昼的身后。\\

周以车站前的可丽饼屋卖的特制可丽饼(冬日限定莓类特辑)为报酬,说动了千岁帮自己忙买了个东西,并把这东西也加进了礼物……可现在周却在苦恼该什么时候把这礼物送出去。\\

而那过生日的本人,正一如既往地做着晚饭。\\

虽然不清楚菜单,但看上去是在做和食的样子,怎么看都没有什么特别的感觉。完全就是一副平常心的样子。

从当事人身上完全感受不到生日的氛围。不如说那淡定程度,简直让人觉得她是不是根本就不记得这回事。\\

甚至到了晚饭做好,两人上了餐桌吃饭时的对话也是一如往常。\\

真的拿不定主意该什么时候把东西给真昼的周,看向藏在沙发后面放着礼物的纸袋,皱起了眉。\\

过了一会,周收拾好餐桌回到客厅的时候,真昼正坐在那刚好两人位的沙发上看着似乎是自己带来的书。

就连看书的模样也美如画作,实在是不虚天使之名。\\

虽说不知为何周对坐在真昼旁边有些微妙的犹豫……但退缩也不是个办法,周便提起放在那里的纸袋,坐在了真昼旁边。\\

真昼突然抬起了头。

大概是注意到了周的气息和纸袋摩擦的声音,真昼那焦糖色的双眼看向了周,然后又移向了周拿着的纸袋。

依旧一脸不解的表情的真昼,看来是到了这个地步还没有注意到自己生日的事情。\\

「嗯,给你的」\\

周把纸袋推出去放在了真昼膝上,使真昼变得一脸茫然。\\

「这是什么」

「今天不是你的生日吗」

「是倒是……话说为什么你会知道啊。我不记得我有跟谁说过这回事啊」\\

真昼的眼里微微露出警戒的态度,但听到周「你上次把学生证落屋子里了吧」的解释,或许是明白了便恢复了平时的表情。\\

「其实,没必要在意的。反正我也不过生日」\\

那冷淡而透出排斥感的声音,应该不是自己听错了吧。

当周望见真昼那如同对生日这词汇本身便不知为何抱着忌讳感的眼神时,他如此确信。\\

原来如此——周想到。\\

明明是生日,她的态度却毫无变化,其原因,并非是不记得生日的事情。

因为生日很烦人所以故意忘掉的——应该说是这么回事。

若非如此,她也不会用那种语调吧。\\

「啊这样啊。那就当作是平常受你照顾的回礼吧。权当我一厢情愿想要报恩」\\

平日来的回礼是另一回事所以不过生日也没差,这就当作我表达的感谢而不是生日礼物,周这么解释,把礼物塞了过去。

每天都吃着这么好吃的饭,偶尔还来帮忙打扫屋子,虽然都是小事,但也实在是受照顾了。即便只是慢慢来,周也想要回报真昼。\\

周虽然很轻易地就退让了,却还唯独要把礼物送过来,这让真昼有些混乱,但真昼还是对着打算把礼物送上的周困扰地微微皱着眉接过了礼物。

真昼的视线,集中在纸袋里面的,用另一层袋子包装的东西上面。\\

「我可以现在打开吗?」

「嗯」\\

看见周点头,真昼紧张地从纸袋里把箱子拿出来,小心地打开包装纸解开缎带。

怎么说呢,看着礼物在自己面前慢慢被打开格外地令人紧张。\\

里面放着的是树推荐的护手霜。因为是套装一起卖的,这个大盒子里还附带着一点小点心。\\

顺带一提这并不是那种带有香味的时尚品,而是以没有香味、适合家务、亲和肌肤,滋润保湿为卖点的东西。

周也确认过网上的评价,效果应该是不用担心的。\\

「嘛抱歉,不是什么值钱东西。看你干家务手应该会干吧。虽然也有那种有香味的,不过你估计有了所以没买。这种的对皮肤挺好,似乎挺有效的」

「实用品呢」

「你的话要选肯定是选实用的吧」

「是呢。谢谢你了」\\

看着微微露出笑容,似是在说你还挺了解我的嘛的真昼,周也稍稍放松了嘴角。

看来印象不坏。\\

之后虽然还有一件东西……但要当面打开周还是觉得有些害羞,因而可以的话周还是想真昼回到家再发现那个东西。\\

可事不如愿,在把护手霜放回纸袋里的时候,真昼似乎是注意到了纸袋里还有一件东西,眨了眨眼。\\

「……是还有一件东西吗?」

「啊—。呃,那个,怎么说。就是个充满独断和偏见的附赠」

「附赠?」

「……啊,附赠」\\

周撇开视线生硬地回答后,歪着头搞不明白意思的真昼觉得不如直接拿出来看来得快,便从纸袋里把那东西拿了出来。\\

周为了让那东西尽可能的不起眼而用了跟纸袋一个颜色的包装,塞在了最底下,但果然这个大小还是很起眼。不如说真亏得能在打开护手霜的盒子之前都没被发现。\\

那东西的包装并非盒子,而是聚酯塑料袋。其大小,正好够真昼双手抱住。

看着真昼把那深蓝色的丝带慢慢解开的周想着自己能不能站起来离开的时候,真昼正好把里面的东西取了出来。\\

两只手小心地把里面的东西取出来的真昼,看起来是真的吃了一惊,眨巴着她那两颗大眼珠。\\

「……熊?」\\

真昼说着的,便是那东西的原型。\\

不算太大,大概小学生抱着大小正适合的,布偶。

布偶那近于真昼头发的淡色的软毛是其特征,而那附在不知为何透出一股天真气息的脸上的乌黑圆润的眼珠里映着真昼的身姿。\\

都高中生了还玩偶啊——说不定会被这么想。\\

尽管如此,听了千岁「女孩子可是不论什么时候都喜欢可爱的东西的」这样的建议后,周选择了这个。\\

再怎么说男的一个人跑去买这东西实在是太令人害羞了,周便以车站前的可丽饼为报酬让千岁陪着自己去买了。

结果从挑选到打包周一直在被千岁笑嘻嘻地看着,说不定其实一个人去买羞耻感还会少一点。\\

「……觉得女孩子会喜欢这个吧所以」\\

周挠着头,不知是在跟谁解释般嘟哝到。\\

这种事周实在是不擅长。

不如说给异性送礼自从小时候送妈妈的以外就没有干过,周甚至没有料到自己能做到这种事情。\\

从男的那里收到这么可爱的玩偶会不会让她受不了啊……周偷偷瞄了眼真昼,而真昼则紧紧盯着熊的脸不放。

也不知是高兴还是不高兴,真昼只是呆呆地望着布偶熊。\\

「嘛,不喜欢的话扔了也行」\\

如果不喜欢的话那也没办法,周这么想着,玩笑般地说了一句,结果真昼却刷地皱着眉把头扭了过来。\\

「那种事不会做的!」

「啊,呃。看椎名的性格我想应该也不会的」\\

真昼的否定比预想要强烈,令周不禁退缩地点了点头。真昼则再次看向手中的熊布偶。\\

「……我不会做,那么过分的事情的。会好好珍惜」\\

真昼纤细的手腕,像是要将其拥入怀中般,紧紧抱着熊布偶。

那姿态看上去,像是不愿喜欢的玩具被拿走的孩子,又像是慈爱地抱着孩子的母亲。

一句话来形容,便是极为珍重地抱着布偶。\\

以感觉快要蹦出啾~的音效的抱法抱着的真昼,稍稍垂下眼帘向下看着怀里的布偶。\\

而那脸上的表情,既不是平常的那种冷淡的表情,也不是被周的脱线惊呆时的表情,而是心安而柔和,泛着慈爱的,爱惜的表情。

如此这般,甚至还泛着天真感的纯洁的微笑,其美丽以及可爱程度,令周不禁屏息。\\

(——不该看的)

望见这样的表情,周不由自主地有了反应。\\

就算没有恋爱意义上的喜欢,单纯是让数一数二的美少女露出了这样的表情,还被自己看见了这一事实,也足以让周心跳加速了。\\

那珍惜地抱着布偶,露出淡淡微笑的姿态,其可爱已经到了恐怕不论谁看见了都会看入迷的程度吧。就算是自认为无欲无求的周也被迷住了。\\

为了确认自己脸上积蓄了多少热度,周伸手按了下自己的脸,手上则传来了比平时更加明显的热感。

想着自己害羞起来太好懂了的周,以真昼听不到的声音「……靠」地骂了一句。\\

幸好,真昼正紧紧抱着熊布偶,把半张脸埋进里面,因而并未注意到周。

看着那可爱的样子,周好不容易才忍住了发出怪声的冲动。\\

「……这么喜欢的话倒也正合我意」\\

周想着说些什么挤出了这么一句,真昼则瞄了这边几眼。\\

「……我是第一次,收到这种东西」

「唉,以你的人气这算是日常贡品吧……」

「你把我当成什么了……」\\

这带有稍许无奈的声音与表情,反而让周安心了下来。大概是因为不再需要直视那样的表情了吧。\\

「……我没有告诉过别人过我的生日。因为不喜欢生日,所以从来都不说的」\\

不喜欢——如此断言的真昼的视线移向了布偶熊。

真昼看着布偶的眼神与嘴中的话语截然相反,充满了安心感,不知为何却让周觉得不大自在。\\

「一般,不认识的人或者没什么关系的人送我礼物我也觉得可怕所以不会收」

「我送的倒是收了啊」

「……藤宫同学又不是不认识的人」\\

真昼小声地回答着,然后把脸埋进布偶里仰头看向自己。周后悔直视她了。\\

无意中向上看着自己的真昼那放松下来的,流露出的与年龄相应的天真表情,实话说,相当令人怜爱。

那可爱令人不自觉地产生了想要摸摸头的冲动,周慌忙用力收回了自己不经意间便伸向了真昼的头的手。\\

「……怎么了吗」

「没,没啥」\\

不知是注意到了周一瞬间动了的手,还是察觉到了周那几近爆发的心痒感,真昼咚地歪了歪头。

仅是这样便差点抓住了周的眼球,美少女这种生物还真是可怕。\\

但是再怎么说,直接回答因为可爱所以看呆了,周还是会感觉羞耻,而且就算说了周也确信真昼只可能回答「哈?」。\\

再说,如果那样的话周在各种意义上都会死亡,因而决定还是把这个冲动深藏于心。\\

「……谢谢你,藤宫同学」\\

真昼小声说出的话,再一次传进了撇开脸的周的耳里。

%************************************************
% 后记:※娇化输出 20%

\subsection{友人的窥探}

「我说周啊,和送礼那位咋样了?」\\

东西是一起去买的,要说当然也是当然,第二天千岁就来嘿嘿地笑着窥探起周的八卦了。\\

在别的班级的千岁放学后跑到了周的班里来,这还没问题。可是这张笑脸周实在不想应付,巴不得现在就跟他们说句拜拜。\\

「既不是你想象的那种关系也绝对没有那种展开」\\

至少周并没有怀着恋爱感情,也不是因为有什么想法才送的礼物。

真昼收到礼物是很高兴没错,然而根本不存在千岁期待的那种展开。\\

「哎呀你想想,就没什么人能让你那么上心吧。这么看关系肯定不浅,还是女的,八卦八卦咋了啊」

「我们没什么见不得人的关系」\\

树也帮着千岁说话,周没办法只能矢口否认。

真昼开心是开心了,可还有这些麻烦事儿,所以周才想尽量不跟人商量的。\\

周可不愿意填饱这俩人的好奇心,所以回答得很冷淡。树把手撑在嘴边,好像在思索着什么一样。\\

「……嗯。我说啊周」

「咋了啊」

「你是送你邻居了?」\\

虽然树情商高、直觉准,可这种时候还真是觉得麻烦。\\

「……你为啥这么觉得」

「你活动范围里认识的人,还说是受了照顾,那只能是邻居了吧。想想你又不是当地人,又和女生没什么交流。最近人还给你饭吃了,感觉你就是感恩了不是」

「你说是就是咯」

「唔唔……我说周,感觉你最近脸色好的不行啊」

「啊,我也发现了」

「那人给你送饭是不是挺频繁的啊。所以你就送个生日礼物感谢一下?」\\

因为说得太准,周拼了命才稳住自己的表情。

一串推测准得简直就像在现场看到一样,搞得周有时都怕了。树虽然看上去轻浮,实际上很认真而且观察细致,其实还挺受欢迎的。不过真希望他这些优点能只对千岁发挥出来。\\

「你还真敢这么乱猜啊」

「又不知道真相,不就只能脑补了。所以,到底是啥情况?」

「你就猜吧」

「小气的家伙」

「小气——」

「啰嗦」\\

不管他们说什么,周都不准备老实交代。\\

万一说漏嘴了一点点,那最后要是不彻底交代清楚的话——树先不说,现役女高中生这种热爱八卦的生物是不会停止追问的吧。

因为世界上存在着这种没有恋爱都能硬扯上恋爱的神奇生物,所以麻烦麻烦真是麻烦。\\

「简直了」周叹了口气,收拾东西背起包准备回家。

这是战略性撤退,也是为了回避他们的烧心攻击。\\

「拜拜了,你们就甭管别人闲事秀你们的恩爱去吧」

「不用你说也会的哦?」

「……\ruby{树树}{\jpfont いっくん},我们去跟踪看他和那女的见面吧……」

「哪有你那样在人面前说的,再说压根没你们想的那码子事,跟来也顶多跟到大门口」

「切」\\

虽然千岁嘟着嘴唇很可爱,但是眼神却一副认真样。\\

看千岁这样子,不开玩笑她真做得出来。周瑟瑟发抖地丢下这俩人快步离开了教室。\\

\vspace{2\baselineskip}

「……好危险」

「什么危险?」\\

周回到家不由得感叹了一句,接着真昼很好奇地问道。

现在这时间要做晚饭还太早,真昼买完菜来了周家里,所以两人正一起稍微休息着。周的自言自语好像是被真昼听到了。\\

顺带一说,今天的她和以前一样。

昨天那笑容是半点都见不到了。她这平时的表情让人怀疑昨天的事情是不是在做梦。

这样才是普通的,不如说周希望她这样。要是再让她摆出昨天那种表情,周是感觉自己心脏要疼的。\\

「啊,怎么说,就是礼物这事,让树他们八卦了」\\

周补上一句「因为之前找树他们商量的」,叹了口气。大概是记住了树的名字的真昼,像是全都明白了一样吐了口气说「啊原来如此」。\\

「嘛,毕竟是藤宫看上去就不会买的东西」

「虽然说不是这个意思啊」\\

周想送女性礼物,这件事本身似乎就让他们觉得不可能是周会做的事情,所以才会有恋爱云云的怀疑吧。

实际上,双方都没有感受到酸酸甜甜苦苦的这种伴随恋爱的味道和感情。\\

「是我这边的事情。真是的,他们瞎想个什么劲」\\

确实,真昼那么可爱,那时是有想摸摸的欲望。这一点周不否认。\\

然而周觉得是个青少年都会这样,说到底周只是再次体会到真昼是个超级美少女然后心跳了几下而已,哪可能是恋爱感情。

就算喜欢她的人格,周也没想过要和她成为这样那样的关系这种夸张的事情。\\

悄悄瞄上一眼,还是一如既往端整的美貌。

然而,并没有昨晚那样的悸动。周再次确认自己并不是喜欢上了她,轻轻叹了口气。\\

要是让真昼知道了周在看她,不知道她会说什么。于是周把视线移回到手机上,忽然看见聊天 App 的图标上已经攒了几个表示未读消息的数字。\\

心想着这大概是树吧,周打开App,结果新消息那儿的名字在周的预料之外。\\

看到志保子这个名字,周皱起了眉头。

这是周为数不多的三位女性联系人之一。

具体来说,就是千岁、真昼,还有——母亲。\\

有什么事啊,周想着打开了她的私聊界面。上面写着周不擅长对付的兴致高涨的文章,内容大概是考试怎么样、生活有没有什么困难之类的这种事情。\\

周不擅长应付千岁,就是因为家里人有个和千岁差不多的……不如说是感觉千岁年纪大了大概就会变成这样。尽管周不讨厌也恨不起来,但就算是亲生母亲,性格上也有些应付不来的。\\

『你爷爷寄水果来了,也给你分一点。礼拜六给你寄过去,那天下午你就呆在家里啊!要是拒收或者不在家的话饶不了你啊?』

「自说自话就把我日程给安排了……」\\

虽然这周六没什么特别的打算,倒是没什么问题,不过这种事情不该早点联系的吗。\\

「怎么了吗?」\\

自言自语似乎给听到了,真昼用平常的表情看着周这边。\\

「嗯,老妈说礼拜六下午要把爷爷给的水果寄来。大概是苹果之类的吧」

「你会削皮吗」

「……削皮器能削吗」

「削是能削……不过会削掉厚厚一层,有点浪费营养呢」\\

「这话像是咱老妈说出来的」这个感想还是咽到心里吧。\\

「大不了连皮啃就是了」

「真粗野啊」

「毕竟麻烦么」

「真懒啊」\\

真昼意见一如既往的直白,周只能露出苦笑,耸耸肩不管了。

真昼虽然一副无语的样子,不过还是认可地说了「嘛反正到了胃里都差不多了」。\\

「对了,不知道烂掉之前吃不吃得完,椎名你要一点么?」

「那就要吧。毕竟水果那么贵\footnote{日本的水果是真的贵。}」\\

真昼说的事情有些成家之后为生活奔波的感觉,不过要说的话她一直就是这副样子吧。\\

「周六是吧,我那天就先做点午饭顺便当做回礼了」

「明明是我一直受照顾啊」

「没事,反正给你做饭我也不讨厌」\\

真昼轻轻地微笑了。

她的微笑让周想起了昨天的事情,周有些尴尬地错开眼睛,简单回了句「……那就拜托了」。

\subsection{安息之地敌人来袭}

接过礼物之后立刻亲手送上的打算或许是个错误。\\

听到门铃声和「\ruby{周——}{\jpfont あーまね}」这充满俏皮的高声时,周就掌握了所有情况并抱住了头。\\

\vspace{2\baselineskip}

真昼周末来做午饭的提议原本让周求之不得感激涕零。周原本还以为这是上天的恩惠。

事实上,她做的培根意面也很好吃。浓酱和黑胡椒的刺激相得益彰,美味得不得了。\\

并不是真昼有什么过错。是的,不是真昼有什么过错。\\

有错的是被千叮咛万嘱咐要呆在家,结果还没注意到这事的自己——以及这位超爱惊喜,会做出奇葩行为的,和自己有血缘关系的女性。\\

「……那个,藤宫?不是快递……」

「不是。这是老妈拿着钥匙穿过大门直达了……」\\

回想起来,错误还是在于把这千方百计想来视察的母亲说的话当真了。

那母亲,不搞点事那是不可能的。\\

「……诶,你母亲?」

「咱老妈估计是来看我日子有没有好好过吧……不事先说好是因为怕我装模作样」

「哦……」

「你这副赞同的样子让我心情很复杂啊不过现在这不重要」\\

问题是现在在这里的真昼该怎么办。

要是老妈在大门外,叫真昼立刻回家就好。然而,既然已经到家门口了,便没法让真昼回家了。

话虽如此,啥都不做就把母亲领进门的话,碰上真昼肯定会发生莫须有的误会吧。真昼肯定也不期望这样。\\

就在自己烦恼着如何是好的时候,门铃声的间隔越来越短了。\\

(——啊真是的)

「……抱歉啊椎名,先到我房间去一下吧。拜托了」

「诶,嗯,嗯?」

「这个你拿着,我想办法把老妈支到外面去,之后你就回家吧。真的抱歉不过拜托了」\\

真的是迫不得已,周选择了隐蔽的方针。\\

虽然她做了午饭,不过已经收拾干净了,这一点没有问题。

鞋子藏鞋柜里就发现不了,她带到家里来的毯子之类的私物让她拿进房间里就好。\\

真昼在房间里这段时间,只要周在母亲大致粗查一遍之后求着做饭吃,母亲应该是会答应的。房间的视察周是准备全力拒绝来应付过去的。

故意要求冰箱里的东西做不出来的东西,然后一起去买菜,在这期间让真昼逃离——这是预定的计划。\\

周告诉真昼没有其他办法了,递给她多出来的的钥匙并认真恳求后,真昼虽然困惑着还是嗯地点头同意了。\\

另外,不用储藏室是因为,现在这个季节没空调还是很冷的。

周的房间有空调还有软软的座垫,这样就不会搞得她坐在空空荡荡的地板上腰疼身子冷了吧。\\

「……那就拜托你了。我现在去应付老妈……」\\

面还没见,周已经很扫兴了。周到玄关的时候,真昼也静静走进了周的房间。\\

确认真昼进去之后,周不情不愿地开了门。\\

「哎呀周可真是慢。精神就好,还以为你睡着呢」\\

很快映入眼帘的是暑假以来就没见过的母亲。

明明是自己母亲,容貌上却没体现出年龄,还挂着家里一直见到的那副笑嘻嘻的表情。体现不出年龄的不只是外表,还有行为也是一样。\\

「行了行了精神着呢您就回去吧?」

「你就这么对你妈的吗……花了几小时才过来的,连点慰劳都没吗?」

「远道而来诚为感激,请回吧」

「还说这种话啊。这么不可爱,和修斗一点都不像呢」

「男人要什么可爱啊」\\

虽然周发出了呕吐的声音,母亲——志保子并没有心情不好的样子,呵呵地笑着就当成是叛逆期接受了。\\

「那我进来了?」

「等等,没让人进吧」

「这边可是拿我和修斗的钱租的?」\\

「我说啊妈,要来就先说一声。都这么大了」

「哎呀,要是不突然袭击的话,不就看不到儿子有没有好好过日子了么?」

「唔……你看没问题吧,都收拾好了」

「还真是,吓着了。周你在家啥都不会,其实意外地挺能干嘛。没想到啊」\\

志保子到了客厅环视了一圈,好像赞赏着一样感慨地点着头。\\

当然,收拾好是多亏了和真昼的共同作业,能保持清洁时多亏了真昼的建议和提醒。以上基本都是真昼的功劳,然而现在这没法和志保子说。\\

「皮肤也挺好的,营养也有好好摄取呢」

「……嗯」\\

周稍微移开了点视线,因为这也是托了真昼的福。\\

「菜也有好好做呢……咦,看上去是两人份的?」\\

涂着指甲油的手指指着餐具的部分。

午饭是两个人吃的,盘子当然也是两人份的。这里是周粗心没注意到,不过志保子眼神也真是好。\\

「因为朋友来了」\\

周并没有说谎。

虽然周不那么确定,不过两人已经建立了接近于朋友关系的交情,应该不算有错吧。虽然说性别没讲出来。\\

周尽量憋着动摇淡淡地回答着说。志保子回了一声「哦~」好像没接受这套说法一样,又把目光放回了客厅。

总算是勉强糊弄过去了,然而差点都要出冷汗了。\\

「……嘛,合格了……简直好得不像是一个男生自己住嘛」\\

志保子观察了一阵,重复了几次质疑回复之后,阐述了总评。

某种意义上是理所当然的吧,大部分事情都有真昼掺和进来。\\

「没啥妈要担心的事情吧」

「是啊,真是吓了一跳。家里你还啥都不会的,看来是成长了啊」

「……我也是会成长的」\\

周心里自嘲着「哪来的脸这么说」回答之后,志保子也笑嘻嘻地称赞了「你也努力了呢」。

因为不是自己的功劳,心里默默地难受。\\

然而实情是不可能说出来的,周只能忍着求她回去。\\

生活检查姑且算是做完了吧。

或许不用求做饭也能回去了——周甚至产生了这样的想法。\\

「最后就是检查房间了呢」\\

然而志保子最后投下的炸弹让周不禁瞪大了眼睛。\\

检查房间,也就是私室……寝室的检查。

里面当然有真昼。要是被发现的话,容易预测得到下场比当初设想的会面还要惨。\\

「喂搞啥啊,自己房间就算是妈也不能进啊」

「哎哟,是有什么见不得人的东西么」

「正常来说男高中生的房间总有一两个见不得人的东西吧」

「还承认了啊」

「是啊承认了所以别进来」\\

这里必须尽全力阻止。就算面子丢了,真昼的存在也必须隐藏到底。

现在,要是周房间里的真昼给看到了,志保子毫无疑问会以自己方便的形式朝着愉快的方向想入非非。无论如何都得避免。\\

就好像固执地不让通行一样,周挡在门和志保子中间说着 NO 拒绝着。志保子很快就看出这是藏着什么东西,说着「有秘密不告诉妈了,你还挺像个样子嘛——」笑嘻嘻地逼近过来。\\

要是到了关键时刻,哪怕有些内疚,周也是抱着不惜来硬的也要拒绝的打算在和志保子对峙的。\\

然而,房间里发出了哐镗的一声。\\

「周」

「嗯」

「藏着什么啊」

「……和妈没关系」

「这样啊,懂了懂了」\\

嘿嘿的笑容变得更浓了。

这种笑带着不准拒绝的压力。每次看到这种笑容,周都非常不适并且反抗的精力会减掉不少。

这已经成了习惯,改不掉了。\\

趁着周支吾时的破绽,志保子把手摆到了门把手上。\\

现在再后悔已经为时已晚了。\\

为了确认这声声响,志保子穿过周的旁边打开了房间门。\\

门前看到的是——背靠床边,膝盖上抱着垫子的美少女。

而且还眼睛闭着,重复着规律的呼吸……简单来说,看到的就是正在打盹的真昼。

\subsection{天使大人被镇住}

打盹本身这事是常有的。

位置上是开着空调的暖和房间,时间上是刚吃饱了午饭,光这两条对打盹来说已经是充足的环境了。

虽然会涌现「正常来说会在男人房间睡着吗」的疑问,不过她姑且把周认识成了无害生物,说不定是不小心睡着了吧。\\

这也怪不得她吧。不出声傻傻待着也挺无聊的,而且总有些事是没办法的。\\

周抱头烦恼的原因,是母亲志保子过来的时候遭到目击,而且还是在这个状态下。\\

百分之百会被误会的。

要是站在别人的角度,周自己也肯定会误会,觉得两人关系已经好到能进房间还大意到打盹的程度。\\

周脸上抽着筋瞄了母亲一眼。她看真昼的眼神十分灿烂。听到了「哎呀哎呀可以可以」这样的心声大概是错觉吧。\\

「哎呀真是的,找了个这么可爱的女朋友啊!周还真是不能小瞧啊」\\

志保子「呀」地发出了一道不符合这年龄的高亢声音,让周的头开始疼了起来。\\

完全被误会了,而且还进入了兴奋状态。

就算假设儿子带了女朋友来,一般也不会那么高兴的吧。

然而现在志保子就是这么高兴,理由肯定是因为志保子喜欢可爱的东西没错了。\\

虽然周已经见得习惯了,然而每次看过去真昼都还是那位拥有极品美貌富有魅力的少女。\\

天真无邪的睡脸没有防备,可爱到让人不禁想去摸摸。

抱着周的垫子睡得香甜的那个样子,强烈勾引起周不太想大大方方说出来的那类欲望。\\

像那样的,连已经看惯的周都赏识的美少女,在志保子眼里是儿子的女朋友(暂定)。

恐怕这让她情不自禁兴奋起来了吧。\\

「莫非不让妈进来是因为女朋友在里面?不知不觉你也成了个男的啊」

「才不是咧!从头到尾都不是!才不是什么女朋友!」

「哎,不用找借口的哦?只要你挑的,妈都不反对」

「哎所以说不是这个问题!不是交往关系啦!压根就不是!」

「说啥不是的,房间都进来了哇」

「还不是您老人家来的这么突然啊!就算只在客厅您不也得误会嘛!」

「最根本的问题是,周要是没这意思也不会把女孩子领进家里来,女孩子没这意思也不会跑到对方家里去的哦?」\\

被这么一说,周使劲思考着反驳的论据却难以寻得。\\

正如志保子所说,周基本上把家当成自己的领域,不怎么愿意别人进来。

虽然一开始让真昼进来是因为没挡住她那架势,不过在那之后,就算不考虑做菜这事,说到底也是因为周中意真昼的性格才像这样放进家里来的。\\

(要说喜欢的话,确实是喜欢没错啦)\\

对周而言,真昼这个少女就算不考虑外表也是很中意的。\\

她有着学校里不表现出的辛辣耿直,同时还不坦率的这种矛盾的性格;她看似冷淡无情实则爱照顾人;她说话的措辞仿佛达观一样;她被出其不意的时候会慌乱地露出与年龄相符的样子;她极为偶尔地会露出天真无邪的笑容。周觉得以上这些全部都是真昼的魅力。\\

虽然说要问是不是恋爱感情的话周会予以否认,但至少她是很有魅力的少女。\\

「虽然有作为友人的好意,但可别把对异性的好意全当成恋爱了。再说,这家伙也没这意思」\\

他们的感情并没有甜到能让周老实赞同志保子的说法。说到底,真昼也不愿意被误会成对周有意思吧。\\

「这可说不准哦?是不是觉得自己能理解女孩子复杂的心情,自以为是了?」

「要怎么说才明白我们不是这种关系啊……椎名,求求你起来吧……」\\

就算说了千言万语,志保子也是不停把话题引向恋爱的方向。周烦恼地按着额头。

希望能快点起来,认真的。\\

「嗯……」\\

或许许愿起了效,又或者是被闹醒了。\\

真昼缓缓抬起合着的眼皮子,发出甜甜的声音抬起脸来。\\

亚麻色的头发顺着肩膀滑落下去。

焦糖色的眼睛朦朦胧胧水汪汪的,那副样子怎么说呢没有戒备到了不好意思直视的程度。\\

可能是意识还微妙地没有完全觉醒,真昼睡眼惺忪地仰视着周,让周微妙地错开了视线。\\

「椎名,睡着这事先不说,现在我被误会了,来帮忙解释一下」

「误会……?」

「我说咱家女朋友啊,你名字叫什么?」\\

真昼还是那副水灵灵的样子思考着那句话,志保子则毫不客气地靠近过去露出了老好人一样笑嘻嘻的表情。

面对这无忧无虑的笑容和亲善的眼神,真昼好像还在刚醒来的混乱中,肉眼可见的惊慌失措着。\\

「诶,那,那个」

「初次见面的时候,互相报出名字是很重要的呢!」

「诶,椎,椎名真昼……」

「哎呀小真昼,名字好可爱呀!我叫志保子,别客气直接叫名字就行」\\

在气势所迫之下不禁报上名字的真昼往周那边看过去,好像在求救一样,而周自己还巴不得别人来救他,实在是帮不上忙所以摇头拒绝了。\\

因为是自家的老妈,所以周很清楚,一旦失控起来就停不下来的。\\

看她对真昼兴致勃勃的,大概是想和真昼彻底交流到最后吧。

也不知道她有没有注意到真昼这个最要紧的人正在困惑着。\\

「那,那个,母亲\footnote{日语原文为 {\jpfont お母様},是一个可以称呼对方母亲的敬称}」

「噢!已经认我当妈了啊!」

「藤宫!」

「我和周都姓藤宫,对吧周」

「妈啊椎名头疼着呢」

「周,女朋友不用名字喊可不行哦?」\\

因为志保子实在不听人说话,周皱起了眉头,但志保子没有介意的样子。看志保子这嘿嘿笑着的样子,该说她是胆子大呢还是脸皮厚呢。\\

「那,那个,志保子」

「啥啊?」

「我,我和藤宫——」

「你说哪个藤宫?」

「……周、周君不是那种关系」\\

在志保子这假惺惺的话下,真昼明显一副狼狈样但还是努力否定了。

因为志保子的催逼,真昼犹犹豫豫地喊了周的名字并窥视了几眼,而成功让真昼喊出名字的志保子露出了满脸的笑容。\\

「噢,那就是今后会变成这种关系吗」

「呃,那,那个,不是」

「哎呀我真是的,是不是当电灯泡了」

「那,那个,让我好好说清楚!我和周、君,不是那种关系,只是在一起吃饭,那个,就是因为周君不会做饭」

「你能当个好老婆呢,小真昼。咱家这周啊家务活不会干还得一个人过日子。如果是那样的话务必要支持他啊」

「啊,那个」\\

我觉得真昼尽力了。

然而,要顶住志保子这势头把事情说清楚,这种事情大概是做不到的吧。\\

定期来家里,亲手下厨做菜,一起在桌前吃饭,听到这些的时候志保子眼神变得更灿烂有气势了。

事已至此,周是阻止不了志保子了。能阻止的大概只有父亲修斗了吧。\\

「……椎名,放弃吧。妈一兴奋就不听人说话的」

「怎么这样……」\\

已经到达大彻大悟的领域的周早早放弃了解释,选择了围观母亲的失控。

\subsection{天使大人感到羡慕}

「话说周还真找得到这么漂亮一女朋友啊,妈都吓到了」\\

疲于否定的周和不知所措的真昼同时闭口不言。

志保子将这一沉默视为肯定——不如说是不管两人怎么说都会视为掩饰难为情的肯定——以毫不掩饰好奇的眼神盯着真昼。\\

「怎么样,小真昼你看周有在好好过日子吗?」

「嗯……那个……至少死不了人吧……」

「你倒是说点好话啊」

「可是一开始房间那么脏」

「要不要那么严格,现在有保持干净吧」

「那不是我帮忙打扫的吗」

「那个,嗯,我是真的有谢谢你啦,饭啊打扫啊这些的」\\

在这些方面周对真昼是抬不起头的。

正因为有她在,才形成了现在舒适的生活,所以如果是磕头感谢的话周是能毫不犹豫地执行的。虽说真昼不喜欢所以不会真磕头,不过周是打算在平日里尽可能努力以慰劳真昼的。\\

只是,志保子把这段发言往不太好的方向理解了。\\

「嘛,周不只是这次,平时也一直让真昼帮忙啊,真是个没办法的孩子……那个说法是在同居吗」

「不是!怎么样才能想成那样啊!只是住在隔壁啦!」

「哎呀,那就是命运的邂逅呢!真好啊周,能有个这么漂亮还能干的姑娘照顾你」

「漂亮能干是没错啦但是命运的邂逅什么的我有意见」

「多浪漫啊不挺好么」

「我说的不是这意思!是说我们完全没在交往!」

「哎哟哎哟」\\

志保子肯定是以为周是在难为情了,而周的脸是真的快要抽筋了。\\

母亲总是做出方便的解释,或者说总是将事情解释成能作为自己美好妄想的食粮的那种东西,而不知道被这样的母亲烦恼了多少次的儿子,发出了这几个月里最沉重的叹息。\\

至于被这惊人的气势压倒的真昼,则是交替看着周和志保子,明显一副不知所措的样子。\\

「小真昼小真昼,虽说这可能是咱做家长的偏袒自家孩子吧,周啊他虽然嘴巴毒而且不坦率,但是其实很诚实很绅士的,你可以当作买到了件好货哦。虽然说周没有女性经验吧,所以这一方面就需要小真昼好好把握了」

「说啥呢妈赶紧闭嘴」\\

后半部分是相当的多管闲事。\\

「我说的不都是事实嘛。倒是周之前为什么不找个女朋友啊,明明长得和修斗一样外表也还不错的。还是因为太土了吗」

「要你多管」

「给小真昼看看你帅的那一面?」

「不给而且这家伙也没想看」

「又来了又来了。啊,要不小真昼把他打扮成你喜欢的样子?只要打扮起来的话周也挺好看的」\\

看着笑嘻嘻地推销周的志保子,真昼或许是束手无策了,笑得很含糊。
能让那个沉着冷静的天使大人畏缩到如此地步,志保子的存在或许在某种意义上非常厉害。\\

「妈,椎名真的很困扰啦。话说你回去吧」

「让母亲回去还真是长大了啊。嘛不过打扰了你和女朋友甜蜜时间也是事实,我也差不多该告辞了吧」

「真的赶紧回去吧」\\

周坚决否定得也已经累了,真昼也被志保子这兴致搞得很辛苦吧。\\

往真昼那边一看,她也微妙地一脸疲乏。

这也是当然的吧。她基本上是个文静的姑娘,却被迫参与到了这情绪高涨得连亲儿子都会累的对话当中来了。\\

周心里做下决定,之后再慰劳她,同时对着志保子往门外甩着手。志保子回了一个微妙的不服气的表情。

即便如此,志保子也没说要留下,应该是姑且顾虑到了这边的意思。虽说明显是在错误的方向上。\\

「啊,小真昼也交换下联系方式吧。咱家周的生活方式之类的 各种事情,之后都跟我说说」

「诶,嗯,好的……?」\\

在最后志保子还建立起了这「求放过」的联系,让周捂住了额头。\\

真昼无可奈何地顺着这势头用手机交换了联系方式。

这样一来,志保子无疑也会开始多管起真昼的闲事的吧。\\

(真的抱歉了)\\

看着志保子满面笑容地握住真昼的手嘱咐着「周就拜托你了」,周便决定了一会儿给父亲发条消息说「求求你把妈控制一下吧」。\\

\vspace{2\baselineskip}

「好累啊……」

「抱歉来了阵台风」\\

尽管滞留时间不长,两个人已经精疲力尽,并排坐在了沙发上。

扑通坐下的周捂着脸长叹了一口气。真昼虽然坐得有些拘谨,平时挺得直直的背也比往常要弯曲了。\\

「真的抱歉,误会没解开就让妈回去了」

「没事,毕竟没有什么损失……」

「啊损失还是……看那样子是中意起椎名了……估计之后得管你不少闲事……」\\

在这一点上,周给真昼添了麻烦,所以真的很对不起她。

首先是儿子的女朋友(误会)再加上志保子喜欢可爱的东西,恐怕志保子是中意得不得了,估计会有不少事情想来照料吧——以至于会到多管闲事的等级。\\

「志保子真的很重视藤宫呢」

「说好听点是这样说难听点就是缠人……」\\

虽然这和溺爱还是不一样的,然而志保子对周的疼爱也并非周的所愿。

因为周太邋遢也有责任,周并不太好提太多意见,但即使如此周也觉得她管得太多了。\\

周对母亲很感恩也很重视,但同时也直率地觉得很麻烦并且希望保持距离。\\

「……真好啊」\\

周看着低声细语的真昼。\\

「好个啥啊」

「你母亲,虽然这么热闹但是挺温柔的」

「那该叫又吵又过度干涉吧」

「……那样也好啊」\\

她的表情不是在客套,而是真的体现着羡慕。她用几乎要消失的轻轻淡淡的声音嘟哝之后把眼睛俯下去了。

容易看出她的表情忧郁而昏暗。仿佛一碰就会垮的那副样子,谁看了都会觉得孱弱吧。\\

渗透着看上去绝不仅仅是疲劳的柔弱和虚幻,这样的她似乎感受到了周的视线,忽然抬起头轻轻地微笑了。

仿佛在表达什么事都没有一样,真昼恢复了往常的表情,罕见地把身子靠在了沙发的靠背上。\\

「小真昼,啊」

「……咋了突然这么来一句」

「不是……只是感觉好久没人叫我名字了。一般都是叫姓的」\\

没人用名字称呼那个人气爆棚的天使大人,这一点很是意外,不过应该是周围人都觉得用名字称呼真昼太过惶恐、过意不去吧。

因为她在学校是天衣无缝的天使大人,周围人不敢那么随便叫她。
还有,用外号称呼她的人倒是不少。虽然她本人讨厌得要死。\\

「嘛,要是没有好朋友的话,也就爸妈会叫了吧」

「爸妈才不会呢,绝对的」\\

冷淡的声音秒答了。\\

周不由得往真昼的脸看过去。她表情上没有浮出任何颜色。

她面无表情,仿佛一切都脱落下来了,甚至可以视作无机物一样。或许缘于其端整的美貌,这甚至还带来了对方是人偶的错觉。\\

然而这也只是一瞬。真昼注意到周的视线后,收起了这张无表情的脸,像是困扰着什么一样眉毛低了几分。\\

「……总之,就是很少见了」\\

小声嘟哝着,真昼轻轻吐了一口气。\\

周早就已经看出来,真昼和父母相处得不好。\\

触及父母的话题时,她偶尔会露出冰冷的表情。而且她还说过,没和父母出去吃过饭,讨厌生日。从这些发言就很容易想象她的家庭环境有问题——然而,哪里能想象出,父母连她的名字都不叫呢。\\

『……真好啊』\\

方才的那句细语,究竟是以何种心情织出的呢。\\

「真昼」\\

自然地,周说出了这未曾叫过名字。\\

焦糖色的眼睛啪地眨了一下。

或许是因为出其不意,好像发呆一样的,平时的态度和表情里藏着的某种稚嫩显露在了外头。茫然自失这个说法应该形容得正是贴切吧。\\

「叫个名字谁都可以吧」

「……说的也是」\\

生硬地补了一句之后,真昼隔了一会儿露出了淡淡的笑容。

那微微安心的笑容,在周的心里泛起了涟漪。\\

「……周君」\\

自己的名字被小声叫了出来,让周心里的涟漪更大了。\\

到刚才为止,或许是因为只对着母亲才这么叫,所以周没怎么挂在心上……像这样面对面叫出名字的话,周就心里痒痒的,有什么感觉焦躁的东西在胸中翻滚着。\\

「在外面别这么叫啊」

「……这种事情知道的啦。倒是你别在外面说漏嘴了啊」

「知道的。这是秘密嘛」\\

周无法直视一脸微笑的真昼。\\

于是,周简单地回了一句「噢」,假装改变姿势朝旁边看去以逃开她的笑脸了。

\subsection{钥匙的去向}

周六母亲的突然来袭之后,周和真昼互相的称呼发生了改变,但除此以外并没有什么大的变化。

并不是说关系突然变好了。不过,随着称呼变得亲密了一些,真昼的态度也多少软化了一点。\\

「……那个,周君」\\

周日傍晚,比平常来得更早的真昼,脸上一副微妙的尴尬——或者说是困扰的表情。\\

虽然让她进来了,但真昼这不明不白的态度还是让周一脸困惑。

周想过是不是对直呼名字有所抗拒,不过真昼叫周名字的时候倒是毫不犹豫,所以原因大概另有所在吧。\\

总之两人都先坐在了沙发上。周看向真昼,真昼则从裙子的口袋里掏出一块手帕。\\

正当周想着突然怎么了的时候,真昼将叠得仔细整齐的手帕展开,把在其中包着的亚光色的钥匙拿给周看。

周对这把钥匙有印象,是因为这就是昨天交给她的那一把吧。\\

「……把钥匙还给你。昨晚结果还是没有还成。那个,不小心忘记了错过了还你的机会……很抱歉,这样」

「这样啊」\\

看起来是因为就这么把钥匙拿回了家心里过不去的样子。\\

搞明白真昼这奇怪的样子是怎么回事的周,看向放在手帕上的钥匙。\\

仔细想想,真昼差不多是每天都会来这边做晚饭。虽然周会去给开门,不过有时周绕了路所以不在家,有时手上忙着停不下来,就需要让真昼在门外等一小会儿。\\

现在这个天气让人站在门口等着,对女性来说是不是太苛刻了呢。

听说对女性来说,身体受凉是大敌,而且设身处地想想,周自己杵在门外也不怎么舒服。\\

反正差不多每天都会来,那让真昼拿着钥匙,对周来说也会更轻松些吧。\\

「虽说你就这么拿着也行呐」

「诶?」

「等什么时候我们没关系了再还我就行」\\

嘛直说的话周要是把钥匙给了真昼,那就表示要受她一段时间的照顾了,可真昼还是不安地看着不接下钥匙的周。\\

「但,但是」

「不如说每次都跑去开门好麻烦」

「真话说漏嘴了哦」

「反正你也不会乱用咯」

「话是这么说……」\\

再怎么说,从她那得到晚饭和来自己家做饭也已经有一个多月了,周觉得真昼的人品自己也算理解了。\\

真昼她首先富有常识和良知,性格上不会做坏事。

就算她拿了钥匙,想必也不会干出把钥匙给别人或是趁着周不在的时候跑来偷偷做些什么这些的事情来吧。她是可以相信的人。\\

「每次都按完门铃等着你不觉得麻烦么」

「就算那么说,感觉你太没有警戒心了」

「我是因为信任你才给你钥匙的来着」\\

听到周这句话真昼瞪大了眼睛,又一时语塞般,皱起了眉。

真昼的表情上浮现出困惑,以及另一种难以名状的感情。\\

不过周这边把钥匙给真昼只是为了省点事,要是真昼不愿意的话周是准备老老实实让步的。\\

而真昼则是来回看了钥匙和周好一会儿——然后轻轻叹了口气。\\

「……那好吧。我暂时借下了」

「嗯」

「……周君真是的,都搞不明白你是心大还是粗心了」\\

对以稍稍带刺的声音抱怨着「真是的」的真昼,周只得苦笑。\\

「这才是我的风格嘛」

「这种话才不是自己说自己的时候用的」\\

哼~地,真昼以冷淡的声音指正,结果却反而让周笑的更灿烂了。\\

看起来真昼已经和周熟悉到能进行这种没营养的对话了。

再说都已经互相称呼名字了,要说还没熟悉反而有些不可思议。\\

虽然被真昼那饱含无奈,似是在说着「拿这人真是没办法啊」的眼神看着,但周却并未感受到冷淡,反而查觉到一丝丝温暖。

真昼也明白周那是在插科打诨吧。\\

「那我就不客气地用了啊。在你家做了什么我可不管哦」

「比如说?」

「……趁你不注意打扫屋子吓你一跳?」

「那还真是感激不尽啊」

「做好一堆吃的把冰箱塞太满?」

「然后早饭也有得享受晚饭菜色也会增加咯」\\

真昼的恶作剧实在是和平——不如说是求之不得,因而对周来说反而是大欢迎。不过恶作剧被轻易化解的真昼则是微妙地有些不服气。\\

只能做得出没有效果的威胁,这如实地反映出了真昼的善良,实在是令人愉快。\\

「总觉得被当成笨蛋了」

「我可没这么干」\\

看来再笑下去真昼就真要闹别扭了,虽然也想看看真昼闹别扭的样子,不过周还是收起了笑容默默地看着真昼。

\subsection{天使大人与奖励}

望着走廊里贴着的写着许多学生名字的纸,「嘛也就这么回事吧」周轻声叹息道。\\

上周考试的名次出来了,于是周便和同年级的同学们一样过来看了。

要论结果嘛第二十一名,和往常差不多,看着还行但不怎么显眼。写题的时候手感上跟以往没有什么变化,现在看到排名一如既往周也稍稍安心了下来。\\

顺带一提,真昼依旧雄踞年级第一之位。\\

尽管她真的是个才女,但周也深知她也没有欠缺努力,于是只得叹服不愧是她。

毕竟也经常看见她晚饭后学习的身影。\\

虽然也有原本脑子好使的原因,但要论将她送上第一宝座的,果然还是坚持不懈的努力吧。\\

「椎名同学又是第一啊……」

「不愧是天使大人,脑子实在好用啊」\\

听见混在喧嚣中的这般论调,周不禁撇起了嘴。\\

「咋啦周,一脸不爽的。排名不妙?」\\

和周一起的树看见周的样子,有点惊讶。

顺带一提名单只有前五十名,所以树其实只是陪着周过来看。\\

「没啥。二十一」

「哦,这不是比上次还好了点嘛」

「差不多吧。这点只是误差」

「哎呀哎呀聪明人说的话感觉就是不一样」\\

周「好好好」随意地应付着故意边笑边说着烦人话的树,再次看向排名表。\\

看起来是真的有好好努力过了啊。

虽然她不怎么愿意把努力给人看见,但对在暗处默默努力的她来说,即便别人看上去理所当然,但这也是她付出了莫大的努力才得到的成果吧。\\

即便周围的人会夸奖她「真厉害」,但这些人却对她付出的努力一无所知,因而也不会犒劳她的努力。

对真昼来说,这应该让她非常苦闷吧。\\

「……至少,我来补足下吧」

「嗯?你刚说了啥」

「没啥。喂,我回教室咯」

「好嘞~」\\

\vspace{2\baselineskip}

「咦,周君,这是什么?」\\

看上去从超市回来直接就进了周家里的真昼,正打算把买来的食材放进冷藏库,结果注意到了这多出来的白箱子。\\

「嗯?啊,是蛋糕」\\

白箱子里面放着的是蛋糕。估计看见箱子的形状真昼多少也料到了,只是姑且问问周确认一下吧。

顺带一提,这是周跑去千岁经常在社交网站上发的喜欢的糕点店买来的。\\

「……你喜欢吃蛋糕吗?」

「倒也不算。这是给你买的」

「怎么又来」

「你不是考了年级第一嘛,稍微庆祝下咯。年级第一,祝贺祝贺」\\

听见是给自己买的的时候,真昼疑惑地眨了眨眼。

看来是真的很出乎意料吧。\\

「其,其实每次都考第一,并没有什么好庆祝的」

「就算是那样,你也一直在努力,偶尔来些奖励不也挺好的嘛。草莓奶油蛋糕不喜欢吗?」

「哎?倒,倒也不会不喜欢……」

「嗯,那就好。吃完饭来吃咯」\\

即便察觉到真昼吃了一惊说不出话的气氛,但周还是就这样结束了会话。\\

要是太过顾虑真昼反而会让她陷入困惑,所以态度还是干脆一点为好。\\

在周看来,真昼这个人在对待他人上算是很尽心尽力的类型,但对待自己来则是十分地严格,没有什么大事就不会让自己放松。

要是没有谁来表扬、犒劳一下她,那她就会一头扎进要做的事情里而不知休息。周甚至觉得,她是不是基本上不知道撒娇这一行为。\\

虽然周和她的相处不算太久,但多少也搞清楚了她的性格。一直都受她的照顾,哪怕只有一点也好,周也想要有所报答。\\

周看着仍呆呆僵在厨房里的真昼,苦笑着轻轻地叹了一口气,在她重启之前一直都看着她。\\

\vspace{2\baselineskip}

饭后,看着一脸微妙地紧张着把蛋糕放在盘子上端过来的真昼,周不禁笑漏了嘴。\\

「为,为什么要笑啊」

「没,没啥」

「感觉就不像是什么也没有」

「别在意」\\

不过是看着真昼紧张得动作发硬的怪样子感觉有些有趣。

但要是笑得太过了会坏了真昼的心情,那原本犒劳她的目的就达不成了,于是周笑得差不多就停了下来。\\

顺带把咖啡也拿了过来和蛋糕一起放在了台子上的真昼,坐在了周的旁边。

这些动作也微妙地显得不自然,令周想要发笑,但毕竟真昼本人就在旁边,现在还是忍住为好。\\

真昼畏畏缩缩地朝上瞄了一眼周。\\

「嗯,祝贺祝贺」

「……谢谢你。不过……」

「好啦好啦你就乖乖收下吧。毕竟你也确实努力过了咯」

「虽然,是这么回事」

「那就行咯,快吃吧。偶尔也让自己放松下嘛」\\

「反正已经买了给你了」周这么补了一句,真昼才以略带抱歉的神情微微点头,拿起了盛着蛋糕的盘子和叉子。\\

「感激不尽」

「请吧」\\

周轻轻地摆摆手,真昼则不知为何以慎重的动作将蛋糕切成一口大小送进嘴里。\\

虽然印象里女孩子对甜食十分挑剔,但既然是千岁都常吃的店那应该就没问题吧。\\

尝了一口的真昼,稍稍睁大了眼,然后微微放松了嘴角,恰是证明了这一点吧。

虽然真昼很少有表情变化,不过最近她也开始变得慢慢会流露出易懂的喜怒哀乐了。\\

真昼慢慢吃着蛋糕,脸上浮现出的柔和表情,让这不过是吃东西的场景变得好似一幅画作。\\

「……?怎么了吗」

「不,没什么」\\

突然发现自己被盯着的真昼,不解地歪起了头。

那比起平常稍显幼稚的表情,令周方才还在盯着看的视线不自觉地迷离。\\

取而代之盯着周看的真昼,则是突然想起了什么一般,用叉子叉起一口蛋糕,朝着周这边伸了过来。\\

变成了所谓的啊~的姿势。\\

「唉,不,我不是想要吃,是说」

「不是这样吗?」

「……呃,嘛,那个……要是送过来了,我还是会要的,啦」\\

这样的场景周实在是没有想过,现在突然发生在眼前搞得周很是狼狈,最后一个不小心同意了下来。\\

毕竟已经是这个年纪,更何况对方还是异性,再加之要被不得了的美少女喂食,某种意义上说不定算是幸运——然而周还没有把自己的羞耻心舍弃到了能老老实实为之高兴的地步。\\

「本来就是周君买来的东西,周君你也有吃的权力啊」\\

如此提案的真昼似乎完全没有意识到这些,仍以平常的表情把蛋糕伸向周的嘴。\\

就算看向真昼也只看得到她不解的回应,周便下定决心一口咬下了蛋糕。\\

在嘴里泛开的,是无比甘甜的味道。\\

「……好甜」

「毕竟是蛋糕嘛」\\

事实上并不只是那样,但真昼似乎没有注意到吧。

呜呣地嚼着,感觉总之就是甜。精神状态的影响相当大。\\

「……看来是什么都没感觉到啊」\\

这边可是甜味害羞味心痒味全部尝了个遍,真昼那边却一脸没事人样。\\

这实在是令人不甘,周于是「稍微给我下」这么说着从真昼手上夺过叉子和刚才一样叉起蛋糕伸向真昼。

有借有还,被搞了怎能不还手。\\

「嗯」

「……那个」

「吃了」\\

或许是因为周语气有些强硬的原因,真昼小心翼翼地,像是被以同样的方式喂食的小鸟一般,一口吃下了蛋糕。

周死死盯着真昼的脸,看见了那脸上微微泛起的红晕。\\

「于是乎,感想如何」

「很,很好吃……」

「不是这个,是问的被喂的感受。如何」

「……感觉非常地害羞」

「是吧。所以说,做这种事可是要被人误解的啊。要做的话女生之间做做算了」\\

对问完「这下明白了我的感觉了吧」后呼地别过头的周,真昼以几乎听不清的声音「嗯……」地答道。\\

应该是把自己当作无害的人看待,真昼才做出那种事情来的吧。

真昼这样无意识地做出这种事情让自己很困扰,不过周感觉也不算坏,因而也没有什么特别要怪罪的。

只不过,那甘甜的滋味依旧在口中回荡。\\

(太过不设防我也很难办啊)\\

被真昼信任本身是挺令人高兴,但那样不自觉而毫无防备地对自己做这种事实在是让人够呛。\\

得出这样结论的周,看着旁边微微害羞地缩着身子的真昼,轻轻地叹了口气。

\subsection{圣诞的过法}

「我说周,在你家开个圣诞 party 不好吗?」

「不好」\\

周一口回绝掉唐突的提议之后,千岁的脸蛋易懂地鼓了起来。\\

平安夜就快要来临了……对与家人分开并且孤单一个人的周来说,这个节日与他并没有多少关系。不过千岁和树似乎是想和周一起过,于是才这样来发出邀请的吧。\\

午休时间特意赶来周和树的教室的千岁,听到周的秒回之后正鼓着脸蛋。\\

「有啥不好的啦周你反正是一个人……啊,莫非是女朋友」

「没有没有不存在的」

「那有什么问题啦。还是你不愿意?」

「周要是不愿意的话那就算了吧」\\

他们也以自己的方式为朋友着想着吧。

虽说估计也有个原因是他们想要一个能自由自在秀恩爱的地方。\\

他们露出这样抱歉的表情周自己也过意不去,而且周也并不是讨厌开圣诞 party。

周之所以不情不愿的,是因为在私人场所看到他们那激烈得不同寻常的身体接触很羞耻,另外还有要跟真昼说明得花上一些功夫。\\

要是说得极端点的话,只要事先跟真昼说好在他们回来之前不要到家里来,再把平时真昼存在的痕迹消灭就好了。\\

「也不是不愿意啦……行了行了,24 号对吧?估计天黑之前就会解散的,那之后你们再去两个人你侬我侬亲亲热热。千万别在我家撒太多狗粮啊」\\

嘛,也不至于非要拒绝。周答应之后,千岁的脸上就变得笑嘻嘻的。\\

「真拿你没办法,就这样凑合吧」

「你以为你谁啊」\\

因为千岁嘴上说得有些嚣张,周没什么顾忌捏了她的脸,接着千岁口齿有些不清地说着「疼疼——树啊~周欺负我~」开始求救了。\\

「喂,周你别欺负小千啊?她的脸只有我能捏」

「行行你就帮我好好捏捏她吧」

「交给我吧」

「别啊——!」\\

周想着这应该也能给他们个借口亲热,便把捏脸的机会让给了树。不出所料两个人果然开始捏脸嬉闹起来了。

被捏的千岁实在是一脸高兴的被捏样儿,周看着这场面耸了耸肩。\\

「……我能回去了吗?」\\

当然说是回去这也是自己的教室,不过在被秀一脸之前周还是想要和他们保持点距离。\\

「不行——怎么能不做好安排啊。蛋糕和午饭都得准备!」

「我可不会做啊」\\

再怎么说周也做不来圣诞用的午饭。

真昼的话估计普普通通就能做出来,然而没办法拜托她帮忙吧。\\

周连连摆手表达自己做不来,不知为何千岁则是凝视着周。\\

「咋了啊」

「就是感觉你不会做饭怎么看着这么健康啊」

「这种事情怎么都无所谓吧」

「嘛嘛小千,周也有他的情况吧」

「诶,树树你不也挺想知道的」

「他说之后会告诉我的」

「我没说过」\\

周使劲瞪了一眼树告诉他别自说自话做下约定,而树好像也是故意的一样高声大笑了起来。

不会缠着不放是他的优点,不过偶尔会像灵光一闪一样开人玩笑也是他不好的地方。\\

「真是的……不过,饭叫个外卖就好了吧。蛋糕倒是得先预约」\\

先不管树对周的窥探,周拿出了一个现实的建议。

要说当然也是当然,自己又做不出蛋糕,而且饭菜也做不来,那么自然就该准备现成的东西吧。\\

「啊,那我要吃披萨!蛋糕我去老地方预约了啊,现在应该还能预约」

「圣诞不该吃烤鸡吗」

「树树不也更喜欢披萨嘛」

「那倒是。还是小千了解我」

「诶嘿嘿」\\

虽然他们自说自话就决定吃披萨了,不过周自己也并不讨厌披萨,而且披萨也挺有 party 气氛的,所以周觉得也没什么不好。

照这节奏,午饭估计就定下来是周和树经常点的那家披萨宅急送了吧。\\

听到披萨,周忽然想起了真昼。

像小动物一样哈姆哈姆嚼着吃的真昼,让周觉得有种神奇的可爱,这应该是因为周平时看到的一直是她优雅吃饭的样子吧。\\

想起前日还喂蛋糕给真昼吃,周的脸上自然就感觉带了点热意。\\

(以后再也不做那种事了)\\

互相喂食这么羞耻的事是在也做不出来了吧。又不是树和千岁那样的亲热情侣,应该也不会再有机会了。\\

「……周,怎么了?」

「啊,没事没事。那预约蛋糕就交给你了」\\

千岁看着一瞬间想了起来然后傻傻不动的周,感到诧异之后又担心地看着周那边。周慌忙把那件事情赶出脑子,恢复到了平日里的面孔。\\

「好嘞!披萨我也去约啦!」\\

听着兴高采烈的千岁的声音,周决定回到家之后向真昼询问圣诞节的安排。\\

\subsection{天使様のクリスマスの約束}

% 「クリスマスの予定ですか? 特にありませんけど」\\


% 洗い物が終わった後にソファに腰かけていた真昼に問いかければ、実にあっさりとした返答があった。\\


% てっきり女子会をしたりするのかと思っていたのだが、予定はなかったらしい。


% 意外だと顔に出ていたらしく、真昼は周を見てほんのりと呆れた顔を浮かべている。\\


% 「基本的に、私と交友のある同性の級友は大体彼氏が居ますよ。男性から誘われてもお断りしていますから、予定はどうしても空きますね」


% 「男子涙目だな」\\


% 外行きの際の真昼は非常にガードが固く、淡い期待を込めて真昼を誘った男子達は堅牢な守備に涙を飲んだ事だろう。


% 周からしてみたらよく誘えたな、という感想である。自分に自信がなければあの天使様を誘えないだろうし、陽キャはすごいなと感心すらしていた。\\


% 「……そんなに私と過ごしたいものなのですかね」


% 「あわよくばお近づきになりたいんだろ」


% 「なんのために?」


% 「そりゃ、お付き合いしたいからでは?」


% 「何故付き合いたいのでしょうか」


% 「……付き合ってあれこれしたいんじゃないのかね」


% 「不純ですね」\\


% ばっさり切り捨てられている男子諸君に内心合掌しつつ「まあでも」と付け足す。\\


% 「そういうやつばっかりじゃないと思うから、あんま疑ってやるな。お前なら、男が向けてくる視線の質くらい分かるだろ」


% 「そうですね。全員が全員不埒な眼差しで見てくる訳でもありませんし。周くんだって違うでしょう?」


% 「俺がお前を不埒な目でいつ見たんだ」\\


% 可愛いとか頭撫で回したいとかその程度の考えならよぎった事はあるが、どうこうしたいとまでは考えていない。


% そもそもそんな事を考えていたら真昼が気付いて離れていただろう。\\


% 無害な男だから彼女の隣に居座れるのであって、仮に牙の一つでも見せたら彼女はすぐに居なくなる。\\


% 特にこれといって彼女ほしいという願望はないし食欲の方が大切なので、今の関係を崩すつもりもなかった。\\


% 「そうでしょうね。周くん最初から私に興味なさそうでしたし」


% 「まあ」


% 「ですので、信頼してます」


% 「そりゃどうも」\\


% 男としてそれでいいのかという信頼のされ方をされている気がしなくもないが、ひとまず安全な男の立ち位置に不満はない。\\


% 「……で、私のクリスマスの予定を聞いてきた周くんは何か予定が?」


% 「ん? ああ、俺、二十四日は昼に樹達がここにくるから、まあいつも通りだが晩ご飯が遅くなるかもしれんっていう連絡しようと思ってな」\\


% ようやく話が本筋に戻ってきたので改めて説明する、真昼は納得したように頷いた。\\


% 「分かりました。そのクリスマスパーティーが終わったら呼んでください、そこからご飯作りますので。用意だけはしておきますね」


% 「おう、なんかすまん」


% 「いえ。楽しんでください」


% 「……寂しくないのか?」


% 「慣れてますし、一人は」\\


% 何て事のないように告げられて、少しだけ胸が痛んだ。


% 彼女の脳内でも親の事が一瞬よぎったのかもしれない、どこか自嘲するような笑みが浮かんでいる。\\


% 「……あー、その」


% 「はい?」


% 「……非常におこがましい申し出ですが、イブは無理でもクリスマス一緒に居る、というのは」\\


% なんというか、こういった提案をするのは、非常に気恥ずかしい。


% 特に他意はないのだが、クリスマスを共に過ごそうと誘うのは、普通特別な意味を持つのだ。\\


% 決して、他意はない。


% ただ、真昼がどこか寂しそうな眼差しで瞳を伏せたのが、嫌だっただけなのだ。\\


% 提案に、真昼は目をぱちりと瞬かせていた。\\


% 「一緒にって、何するのですか?」


% 「え? あー、別にする事ないよな。ごめん」\\


% そこを指摘されると、周としてはそれ以上は強く押せない。\\


% 他人に見かけられた場合の面倒を考えれば一緒に出かける事はまずない。


% では家で過ごすという事になるが、この家には真昼の興味を引くようなものはほとんどないだろう。\\


% となると何もせずに二人で側に居るだけなのだが、恐らくめちゃくちゃ気まずいのではないだろうか。\\


% それくらいなら、別々に過ごした方がいいのかもしれない――そういう考えで撤回しようとしたのだが、真昼は周を静かに見つめていた。\\


% 「……じゃあ、あれ、してみたいです」\\


% 予想外だったのは、真昼が存外乗り気だった事だろうか。\\


% か細い指が、テレビの方を示す。


% 正しくいうと、テレビボードの中に格納されているゲーム機だろう。


% 最近の夕方は真昼が居るのであまり起動しなくなったのだが、それに興味を示したらしい真昼は「ああいうの、した事ないですし……」と小さく希望を示している。\\


% クリスマスに特に交際している訳でもない男女がゲームして過ごすというのは何だかシュールな気もする。\\


% 「いや、まあ、それでもいいけど……いいのか? ゲームとかで」


% 「駄目なのですか?」


% 「駄目ではないが」


% 「じゃあ、それでいいです」


% 「お、おう」\\


% それでいいんだ……とは思ったものの、真昼の希望なので出来うる限り叶えてやろうとも思う。


% ささやかな楽しみくらい、あげたい。特にクリスマスに予定を入れている訳でもないし、真昼と食事が出来るだけでももうけものだろう。\\


% 「ま、クリスマスとか関係なしにまったり過ごせばいいだろ」


% 「そうですね」\\


% 小さく笑った真昼が何だか直視しにくくて、周はうなずいてさりげなく顔を背けた。


\subsection{天使大人与意外的相遇}


% 「メリークリスマス!」\\
「Merry Christmas!」\\

% そしてやってきたクリスマス。
然后,到了圣诞当日。

% 学校は既に冬期休暇に入っており、恐らくみな思い思いの過ごし方をしているであろうこの日、樹と千歳は荷物を抱えて周の家に集合していた。\\
学校已经放了寒假,在这想必大家都按着自己的过法度过的日子,树和千岁抱着行李聚在了周家里。\\

% 時刻は十三時頃。
时间大约十三点。

% テーブルの上には既に宅配を頼んだピザやジュースが並んでいる。こんな時間になったのは、予約していたといえどクリスマスの混雑には敵わなかったので遅れてしまったからである。\\
外卖点的批萨和果汁已经摆在桌上了。拖到这个点,论原因还是圣诞节订单太多,就算预约了也无济于事,送达的点还是迟了。\\

% 昼食にするには問題ない時間であるし、二人も昼を過ぎてからやってきたうえ、そう待ってもいないため、皆気にした様子はなかった。\\
但是这个点吃午饭也不算晚,况且两人也都是过了晌午才过来,没等太久,所以都不怎么在意。\\

% 「はいはいメリークリスマス」
「好好好merrychristmas」

% 「周ノリがわるい! もう一度」
「周说的好应付!再说一遍!」

% 「Merry Christmas」
「Merry Christmas」

% 「発音よく言ってるけどやっぱりノリ悪いね?」\\
「虽然发音很标准但还是觉得好应付哦」\\

% 元々テンションが高い千歳と一緒にしないで欲しかった。\\
周本来就不大想跟活泼喧闹的千岁待一起。\\

% 樹はこれでもテンションはあげている方だと気付いているので千歳をなだめつつ、いつものややチャラいながらも爽やかな笑みを浮かべている。\\
注意到周是在勉强附和着千岁的树,一边哄着千岁,露出了平常那爽朗而略带轻浮的笑容。\\

% 「まあまあ。そんな事はいいだろ、とにかく食って遊んで寝ようぜ」
「嘛嘛。这种事随便了啦,总之吃饱了玩玩够了睡咯」

% 「うちで寝んな馬鹿」
「别睡我家,蠢货」

% 「冗談だよ。寝るならちぃんちで寝るし」
「开玩笑啦。要睡也是去千千家了啦」

% 「親が居ない内にしとけよ」
「记得要趁父母不在啊」

% 「えー、周ったら何すけべな事考えてるのー?」\\
「哎~,周你是不是在想什么色色的事情~?」\\

% にやにや笑いの千歳はスルーしておき、周は食器とコップを取りにキッチンに向かった。\\
周无视掉一脸坏笑的千岁,走向厨房去取餐具和茶杯。\\

% 千歳はつまらなそうに唇を尖らせたものの、手伝うーと言いながら周の後ろをついていく。\\
千岁虽然撇着个嘴一脸不悦,但还是说着「我来帮忙~」跟在了周的后面。\\

% キッチンは、当然ながら綺麗に整理整頓されている。最早真昼のテリトリーであるため、彼女が使いやすいように各種道具や調味料が並べられていた。\\
不必说,厨房自然是被整理的干干净净。毕竟这已经算是真昼的领土了,自然各种道具和调料也都按她方便被排的整整齐齐。\\

% 「意外なまでに綺麗だね」
「还真是意外的干净呢」

% 「そりゃどーも」\\
「谢夸~」\\

% 適当に流して食器棚から小分け用の小皿やらコップやらを取り出して千歳に半分ほど渡していると、千歳は食器棚をじーっと見つめていた。\\
周适当应付着千岁,从餐具柜上取下分批萨用的小盘子和杯子,分了一半递给千岁,却发现千岁正盯着餐具柜看。\\

% 「……なんだよ」
「……咋了啊」

% 「べつにー?」\\
「没啥哦~」\\

% にまー、という笑みに何だかねっとりとしたものを感じたので背筋を震わせつつ、あくまで無視の方向を決め込む。
看见千岁咧开嘴的笑脸,周莫名地感到一阵不妙,决定坚持无视千岁。

% ものすごく彼女の中で多大なる誤解的な何かをされている気がしたが、口にはされていないのでそれが何かまでは分からないのだ。\\
虽然周注意到了千岁心里似乎产生了一个不得了的误解般的什么,但因为她没有说出口,周也没法确定到底是啥。\\

% ほんのり上機嫌になっている千歳に頬をひきつらせつつ、周と千歳は樹が待つリビングに戻った。\\
心情变得有点愉快的千岁,和看着那样的千岁不禁面部抽搐的周返回了树待着的客厅里。\\

% \\


% 「しかしまー綺麗だねえ部屋。広くてゼータク」\\
「不过嘛~这屋子可真整洁啊。还大的吓人」\\

% 部屋に置いたオーディオから流れるクリスマスっぽい音楽を聴きながら食事をあらかた摂り終えたあと、一息ついた千歳は三人しか居ないリビングをぐるりと見回して呟く。\\
听着屋里音响传出的圣诞风的音乐,买来的食物都被吃的差不多的时候,正稍稍休息下的千岁环视着这只有三个人的客厅小声说到。\\

% 広いのはここを借りている両親のお陰であるし、綺麗なのは真昼が掃除を手伝ってくれたからなのであまりコメント出来ず「そりゃどーも」とだけ返すに留めておいた。\\
宽广是多亏了租下这里的父母,而整洁则是仰赖真昼过来帮忙打扫,因而周也没什么好回复的,只好答了一句「嘛谢了」。\\

% 「まあ一時期すごかったよなー。よく綺麗になったもんだ」
「嘛~有一段时间可是超乱的嘞。现在倒是变得挺漂亮了嘛」

% 「うっせ」
「啰嗦」

% 「うんうん、女の匂いがするねー」
「嗯嗯,总觉得有股女性的气息呢~」

% 「何でそうなる」\\
「那什么鬼」\\

% 部屋が綺麗になった、から何故女の存在に繋がるのか、周にはちっとも理解出来なかった。\\
屋子变得干净,怎么就跟有女性扯上关系了,周一时间有点没法理解。\\

% 「んー? 何となくかなあ。周の性格的にちょっと掃除の仕方が違うかなって。本の並べ方とかコードとか傷まないようにまとめてたりとかもあるんだけどさー。周の趣味じゃなさそうな食器幾つかあったんだよねー」
「嗯~?大概是直觉吧。屋子的感觉跟周的性格对不上啊。书摆放的方法啊电线的走法啊不让东西损伤摆在一起啊这些的。另外还有一些跟周的兴趣对不上的餐具这些~」

% 「……母さんのだし」
「……我妈干的」

% 「ふーん?」\\
「呼~嗯?」

% 一応奥の方に仕舞っておいたものの、食器を取り出す時に千歳に見られていたらしい。
尽管那些东西都被放在了最里面,不过看来拿餐具的时候还是被千岁看到了。

% 周の食器だけでは足りないので真昼がいくつか自宅から持ってきていたのだが、まさかそういった細かいところによくもわるくも大雑把な千歳が気付くとは思わなかった。\\
因为光是周这有的餐具还是不够,真昼便从自己家里带来了几样,但周还是没有料到平常那么粗神经的千岁会注意到这样的细节。\\

% 「ま、別にいいんだけどー? ねーいっくーん」\\
「嘛,反正也没差?呐,树树~」\\

% 微妙に反応が遅れた周を意味深に見た千歳は、にっこりと笑って樹にもたれかかる。
意味深长地看着微妙地迟疑了一下的周的千岁,满脸笑容靠向了树。

% こうしてくるのはいつもの事らしく、特に驚いた様子もなく千歳に手を伸ばして膝の間に座らせていた。そのまま彼女の体を包んでいるので、なんというか非常に直視しにくい。\\
似乎这已经是常有的事,树并不怎么惊讶,而是把手伸向千岁让她坐在自己膝间。这样把千岁抱住的姿势,对周来说莫名地难以直视。\\

% 「人んちであんまいちゃつくな」
「在别人家这么秀有点过分啊」

% 「うらやましいー?」
「羡慕啦~?」

% 「別に」\\
「那倒没」\\

% 羨ましいというよりとにかくいたたまれなくなるので止めてほしいのだが、彼らはこれが通常運転なため注意もあまり効果がないだろう。\\
与其说是羡慕,不如说令人感觉待不下去,希望他们能收敛下,但考虑到对他们来说这才是正常运转,就算让他们注意也无济于事吧。\\

% 樹とくっついてご満悦そうな千歳は樹の胸に体を預けつつ、天井と樹の顔を見上げている。\\
跟树黏在一起,看上去心情大好的千岁把身体靠在树胸前,仰头望着天花板和树的脸。\\

% 「……今頃みんなこうしていちゃいちゃしてるのかなあ」
「……现在大家是不是都在这样亲亲热热呢」

% 「血涙を流すやつらが居るのも忘れないでやってくれ」\\
「别忘了那些正留着血泪的家伙们啊」\\

% みんなこうしている、なんて事はありはしないだろう。家族と過ごす人間も居れば友達と過ごす人間も居る。一人の人間だって居るだろう。
大家都是这样——这完全是天方夜谭吧。肯定也有和家里人、和朋友一起过的人。当然也有一个人过的人咯。

% 独り身を屈辱だと捉える人は結構に居るので、千歳の発言は外に出したら危ない気がした。\\
视独身为屈辱的人可是要多少有多少,千岁这话要是在外面说恐怕有些危险。\\

% 「男子ってそんなに恋人欲しいものなの?」
「男生都那么想要恋人的吗?」

% 「そうなんじゃないのか。俺はそうでもないけど」 
「嘛或许是那样吧。我倒没有特别想要就是」

% 「そりゃ周が変わり者だからな気がするけどな」
「我倒觉得那是因为周本身就是个怪人呐」

% 「うっせ」
「闭嘴」

% 「まあクリスマス前ってみんな浮き足立ってるよねえ。特に独身男子。この間天使様のところに押し掛けてクリスマスの予約とろうとしてみんなばっさり切られてて屍の山が築かれてたよ。なんでも約束してる人が居るから無理です、だって」
「嘛圣诞节前大家倒是一副坐立不安的样子呢。特别是单身男生们。话说最近一堆人涌去天使大人那里想要邀请她一起过圣诞,结果被全部干脆拒绝了,尸体都快堆成座山了耶。据说是已经跟人约好了所以不行,这样」

% 「へえ」\\
「哦,是嘛」\\

% その約束の相手は、自分なような気がする。
感觉那个约好的人估计是自己。

% 体のいい断る理由になっている気がしなくもないが、断る事で痛む真昼の良心を考えればいくらでも使ってくれて構わない。どうせ名前は出していないのだから問題ないだろう。\\
虽然对被当成了拒绝他人的理由周有点在意,不过如果能减轻真昼拒绝他人给内心带来的罪恶感,那么再怎么用周也不介意。反正也不会蹦出自己的名字,应该没有问题。\\

% 「その時の男子の絶望の顔がやばかった。失礼ながら笑った」
「那个时候男生露出的绝望表情可真是。虽然很失礼,但我还是没忍住笑出来了」

% 「笑ってやるなよ」
「你别笑啊」

% 「だってさ、普段から関わりないのにイベントにかこつけて一緒に過ごしたいってそりゃ無理な話でしょ? その前から仲を築けなかった時点で出遅れてるんだしさ、仲良くないけどこれから仲深めよう一緒に過ごそうって都合がよすぎなんじゃないかなーって。あとそういう輩に限ってみんなでパーティーとか言いつつ二人きりになろうとするんだよ。女子からしたら怖いし」\\
「毕竟啊,平时明明没什么关系,一到圣诞节突然就想要一起过那当然是不可能的咯?没有趁早建立关系就已经迟了一步了,『虽然交往不深但让我们一起过圣诞节加深关系吧』哪可能有这么好的事情啊。还有,那种人还说着大家一起过party结果突然变成两个人诶。在女孩子看来很吓人的」\\

% そんなのについてく尻軽じゃないしなー、と舌を出してる千歳は、嫌なことを思い出したのか樹にくっついている。
哪有连那种人的邀请都会答应的便宜女人啊——千岁像是想起什么不愉快般咋舌说到,然后靠向了树。

% 千歳も真昼とは違ったベクトルでの美人なため、色々とあったのだろう。もてる女は人間関係に悩まされるのだな、とちょっと可哀想になった。\\
虽然和真昼方向不同,但千岁也是个名副其实的美女,所以也有各种各样的烦恼吧。受欢迎的女人人际关系真是难搞啊——周开始同情起千岁了。

% 「まあ椎名も大変だな、色々誘われて」
「嘛,有那么多邀请,椎名她也挺不好过啊」

% 「……周ってほんとに天使様に興味ないよね」
「……周你真的对天使大人没兴趣么」

% 「まあな」
「嗯哼」

% 「お隣さんが周にとっての天使様だもんねえ」
「对周来说邻居才是天使大人呢」

% 「追い出すぞ」
「赶你出去嘞」

% 「いやん」\\
「别嘛」\\

% しつこい、とほんのり強めに睨むとおどけた風に「こわーい」と樹にしがみつく。\\
真是烦人——被周这么狠狠瞪了的千岁以滑稽的姿势说着「好可怕~」抱住了树。\\

% 「でもお隣さんにお世話になってる事は否定しないんだね」\\
「不过倒是没有否认被邻居关照了呢」\\

% ぐ、と言葉をつまらせると、千歳は満足したように笑った。\\
周呜——地一时语塞,而千岁则一脸满足地笑了出来。\\

% 「睨まないでよー。ごめんって」\\
「别瞪人家啦~抱歉啦」\\

% あまり反省していなさそうな声音で謝った千歳をもう一度睨めば「きゃー」なんて可愛らしい声をあげて樹とべったりして……それから、ふと樹の背後にあった窓に目を向けた。\\
周又瞪了一眼以毫无反省的语气道歉的千岁,千岁则发出可爱的「咿呀~」声抱紧了树……然后,千岁的看向了树身后的窗子。\\

% そちらを見て固まったので何事かと周もつられて窓を見て、青空の背景にふわりと白いものが落ちていくのが見えた。\\
不知为何千岁盯着那边看的周也跟着把视线移了过去,映入眼帘的则是在那蓝天下纷纷扬扬的白色雪花。\\
%天蓝色的幕布前翩跹而下的白色精灵。\\

% 「……あ、いっくん見て! 雪だ!」
「……啊,数数快看!下雪了!」

% 「おー、ホワイトクリスマスってやつかー」\\
「哦哦,看来是个白色圣诞节啊」\\

% 十二月後半ともなれば、雪が降ってもおかしくはない。
已是十二月下旬,就算下雪也并不稀奇。

% 晴れているのに雪が舞い降りるというのはやや珍しいが、恋人たちにとっては嬉しいものだろう。
虽然天还在放晴却下起了雪是有些少见,不过对恋人们来说反而是值得高兴的吧。

% まだ夜ではないが、気温的に恐らく夜までちらちらと降っていそうで、聖夜を雪化粧してくれるにちがいない。\\
虽然还没到晚上,但从气温来看恐怕是要一直下到夜里,这么一来今年将会是个银装素裹的圣夜呢。\\

% さぞカップルが喜ぶんだろうなあ、と思いながら、身近なカップルが窓を開けてベランダに出るのを見守った周は「どうせしばらくそこでいちゃつくだろうし、温かいものでも準備しとくか」と立ち上がったところで――ベランダから、すっとんきょうな声が聞こえてきた。\\
这下情侣们可要兴奋啦——周边想着,边抱着「反正他俩要在外面亲热一阵子,我去准备点暖和东西吧」的想法看着身边的这对情侣兴奋地打开窗户跑出阳台,正打算起身——却听见了阳台上传来了不同寻常的声音。

% 「へ? な、何でここに」
「诶?为,为什么在这」

% 「え、え?」
「呃、唉?」

% 「あっ」\\
「啊——」\\

% 最後に聞こえてきた声は、最近聞きなれてきた、どこか甘さを感じる涼やかな声だった。\\
最后听见的那一声的音色,是最近周听惯了的,令人感到甘甜而清爽的声音。\\

% 猛烈に、嫌な予感がする。\\
一股不好的预感猛烈地涌了上来。\\

% ベランダで二人が固まる気配を感じながら慌てて駆けつければ、ベランダでは丁度雪を見に出たらしい真昼が、柵越しに二人と出くわしているところだった。
周察觉到阳台上的两人愣住了的气息,连忙跑出去这下——恰是看上去正好出来阳台想要看雪的真昼,与那两人隔着栏杆相遇的时候。

\subsection{天使大人与困惑}

太糟糕了。周看着旁边姿势端正地坐着的真昼,叹了一口气。\\

周迎来阳台遭遇这一惨剧之后,没有办法只得把真昼请来家里了。

反正就算试图糊弄,这两个人也毫无疑问会胡思乱想。所以干脆老实说出来,还更能防止一些多余的臆测和误会吧。\\

并且,不好好封上他们的嘴的话,之后的事情会很恐怖。\\

「……那个,真的很对不起……」

「……并不是你的错……」\\

虽然真昼以非常抱歉的细小声音道歉了,但唯独这件事上真昼并没有错。

在白色圣诞节,而且还是今年的初雪,真昼是不由自主便去阳台赏景了吧。\\

如果周听见了打开窗户的声音恐怕会阻止两人,不过由于房间里放着音乐所以并没有听见。

而且,真昼也有注意尽量不发出声音吧。周是完全没有注意到。\\

看着相互反省的两个人,千岁眼睛发着光把脸猛凑了上来。\\

「原来,周的邻居是天使大人吗!?」

「那个,天使这个称呼可不可以……」\\

真昼似乎再怎么说也并不愿意在眼前被称呼成天使大人,正在委婉地拒绝着,然而千岁一副笑嘻嘻的表情,不知道她有没有在听。\\

至于树,他一边挠着脸一边交替看着周和真昼一边微皱着眉头。\\

「嗯嗯。那么……根据目前为止的信息来推断,椎名住在周的隔壁,经常给周做饭,我说的对吗?」

「……嗯」

「嘛,嘛……那个,因为藤宫有恩于我,而且看藤宫那个样子就知道吃得不健康,所以很在意……」\\

真昼淡泊地说出了两人开始有交情的契机,并且解释了为什么交流会持续下去之后,树说着「原来如此」却同时挂着一副微妙的好像没有接受的表情。\\

如果周在树的立场,也是没法接受的吧——周这样的普通男生,竟然能得到真昼这样优秀的女性的照顾。\\

「嗯,我算是知道怎么回事了。不过嘛,椎名不管怎么说,对周没有其他意思才很神奇啊这个状况。基本算是走婚妻\footnote{走婚妻:原文为 {\jpfont 通い妻},意思是平时不同居,有需要时才到丈夫家里的妻子}了吧」

「噗」\\

听到了这平时完全没听过的单词,周不由得笑了出来。\\

走婚妻——被这么一说,只看状况的话或许是挺像的。周每天的晚饭都有给做,偶尔假期里还有她的午饭吃,并且偶尔还来帮忙打扫。听上去的感觉说不定确实挺有这么一回事的。

区别就在于,两边互相都没有带着爱情这样东西吧。\\

听到树这么一说,真昼虽然稍稍睁大了眼睛,不过很快就转变成了对外用的笑容坚决否定说「没有这个打算也不可能」。

想着真昼对树和千岁是用学校同样的方式相处,周心里便觉得有些痒痒的。\\

「我这也没有什么邪恶的想法,所以椎名才会来帮我的吧」

「周这么说倒是没问题啦。不过,还真的是谜之组合啊……那个才女给周做饭啊……布偶也是送给椎名的?」

「……算是」

「哦~」

「好烦」

「我还什么都没说呢?」

「你脸就好烦」

「好过分!」\\

千岁笑嘻嘻的……不如说是坏笑的表情让周暴躁的心非常地不舒服。\\

目前还只是确认事实,没怎么被千岁调戏,不过周可不愿意被她调戏。因为对真昼也会有影响,可以的话真是想要无视千岁。\\

「我说两个人都冷静一下」\\

树一开始就注意到了周的样子有变,所以没有千岁那样调戏周的样子。

树会在周真的不高兴之前停手,是个会察言观色、能为别人顾虑的男人。可以的话真希望他能在窥探之前就停下,不过这事还是没办法的吧。\\

周微妙地瞪着眼,而树劝诫了因为谜团解开而满心欢喜的千岁之后,不知为何端正了姿势,整个身体朝向真昼低下了头。\\

「……那个,椎名,咱家的周受你照顾了」

「我啥时候成你家孩子了」

「彼此彼此,谢谢你能为了藤宫和他这么要好」

「别顺着说下去啊,搞得我好像废人一样」

「确实挺废的」

「你这家伙」\\

确实,树也一直说着周,周自己也有所认知……不过被指出来还是十分心情复杂的。\\

真昼似乎懂得这种玩笑,抓住了机会故意装了个傻之后,看着周和树的对话嘻嘻地微笑着。\\

虽然这笑容不至于到只给周看到的真实面孔那样,不过也不完全是对外的装模作样,这让树也露出了有些愣住的表情。

周捅了捅树告诉他「有女朋友的人别看傻了」,接着不开心的千岁也同样……不,是更用力一些地捅了捅树,让周莫名觉得有趣。\\

只是,真昼有些疑惑地把脑袋稍微歪了过去,于是周便当作无事发生般恢复了原本的姿势。\\

「……所以说,虽然我们并不是你们那么甜的关系,不过要是给别的家伙知道了肯定会引来麻烦事的,这你懂的吧」

「知道知道,不会跟别人说的」\\

周这是暗中威胁说「要是告诉别人了会怎么样你懂的吧」,不过树很轻易就答应了让周倍感意外。\\

「千岁你也是」

「我也没那么多嘴啦~而且,这么可爱的家伙给周做饭什么的,说出去估计也没人相信」

「配不上还真是抱歉了」

「我没说到这个地步啦~」\\

千岁说的并没有错,而且周也对自己有认识。

普通的男生,正在得到学园偶像级别的天使大人照料,这种话谁都不会相信的吧。

就算有人相信,肯定也会骂周不配。\\

这种事情周也不是料想不到,所以周才不想让周围人知道这个事实。麻烦事可还是算了吧。\\

真是低三下四啊——千岁笑着看着周,不过她的视线忽然像是被吸过去一样转移到了真昼身上。\\

千岁先是用热情的眼神注视着,然后紧接着叹了口气之后又开始注视上去了。

真昼也一副不太舒服的样子,看上去是不知如何是好。\\

「那个,怎么了?」

「……我又觉得啊,椎名这怎么这么可爱的啊」

「诶?谢谢……?」\\

千岁正面夸奖了真昼,接着便目不转睛地盯着真昼的容貌。\\

「这么近看还是第一次,果然漂亮得能说是天使大人呢。相貌端正皮肤白皙长得漂亮睫毛又长头发顺滑身体苗条还有凹凸」

「那,那个……?」\\

千岁的老毛病似乎又犯了,周大大地叹了口气。\\

周不擅长与千岁相处。

周并不讨厌千岁,至于人格周还挺欣赏的……然而无论如何都有不擅长的地方。比如那高涨的情绪,比如偶尔会太关心别人的事情的地方,这些都让周疲于应付。因为周的家人里也有类似的人,所以这种意识就更加强烈了吧。\\

也就是说,千岁这与母亲的相似之处让周难以应付。\\

不只是性格,千岁的嗜好也和母亲很相似……特别喜欢漂亮的东西和可爱的东西。\\

不阻止的话,感觉真昼实在有些可怜了,于是周骂了一声「真是的」,同时轻轻拍了拍快要把手伸出去的千岁的脑袋。\\

因为目的是制止和吐槽,周花的力气是真的很轻,不过受到冲击的千岁还是小声喊了句疼收起了伸向真昼的手。\\

「这点事不至于拍脑袋吧」

「这家伙很怕生的,没熟悉之前不要身体接触」

「熟悉之后就没问题了吗?」

「这个你问椎名。注意阶段啊阶段」\\

真昼明显是摆出了要逃的姿势,阻止千岁应该是正确的选择吧。\\

看到了有一些……不如说相当困扰的真昼,千岁似乎也理解了阻止的理由。\\

「对不起,我太兴奋就想摸摸了」

「哈,哈啊……」\\

就算突然暴露出想摸真昼也是困扰的样子,仿佛不知如何是好一般,用眼神朝着周求救。

「啊,椎名,千岁虽然是劲头猛的怪人不过不是坏人……吧」

「我说你这算是袒护我吗?不是吧其实是在损我吧?」

「你看看你现在这样子能否定吗?」

「不能!」\\

千岁光明正大否定之后又注视了一会儿真昼,接着以认真的表情再次把手朝真昼伸了出去。

这次是把手心伸出去的姿势。\\

「那么就从朋友开始吧,请多多指教」

「诶?啊,好,请多多指教……?」\\

被请求握手后,真昼也畏畏缩缩地握住了伸出来的手心。\\

千岁这一旦中意上就想要变得更要好,从她的性格来看感觉真昼是有的被折腾了。不过既然是普通的朋友关系周也没什么插嘴的余地吧。

真是希望千岁能在相处中有节制。\\

「嗯嗯,又诞生了新的友情啊」

「你倒是好好控制一下你女朋友啊」

「我尽量」\\

周对着每次都让千岁差点暴走的树吐槽一句之后,看着握着真昼的手笑嘻嘻的千岁再一次叹了口气。\\

\subsection{暴风雨过后}

「真的对不起啊」\\

到了傍晚,树与千岁回去之后,周对着略有表现出疲惫的真昼做出了道歉。\\

一下子被不认识的人缠上还被知道了秘密,真昼估计也困惑、疲劳了吧。

感觉这样的对话在志保子那时候也有发生过一遍。\\

「不,毕竟原因是我自己的粗心」

「她够吵吧」

「……是个活泼的人」

「老实说她吵也没问题的哦」

「虽然精力有点旺盛不过是个有趣的人」

「这哪只是有点……嘛你不在意的话那倒没事啦」\\

周是觉得那绝对能说是吵闹啰嗦了,不过客气的真昼对她的评价实在是非常委婉。\\

虽然真昼没有多么讨厌千岁,这点是值得庆幸的,然而周并不晓得和那人成为朋友到底会不会是件好事。

千岁和真昼算是迥异的类型……在新鲜感的意义上或许是件好事,也说不定吧。\\

当然,要是真昼太困扰的话,周是打算提醒千岁的。周的想法是留着心眼默默守望这两人。\\

「我周围没有那样的人,所以还是有些开心的」

「嘛千岁那种家伙也确实是不怎么见得到吧……要是她太缠人就打她脑袋啊?」

「暴,暴力是不好的,我会努力用语言阻止」\\

尽管有种两个人都默认千岁会暴走的感觉,不过千岁确实是经常把满腔的热情往奇怪的方向上使,所以这种提醒还是有必要的。\\

周在心中立下了稍后直接去劝告千岁的誓言,同时转身朝向窗户的方向眺望起飘落的雪花。\\

如果不是这天气,也就不会暴露给那对情侣了……不过,下雪或许也表示着对恋人们的祝福,所以周也不好太多抱怨。

真昼似乎也是喜欢看雪的,她看到周在看雪便也同样望了过去。\\

因为是冬天,太阳也早早落下,周围已经暗了下来。

现在看天黑程度已经算得上是晚上了,雪下得也很薄,所以在家里的灯光下才勉勉强强能看得出雪。\\

「是白色圣诞节呢」

「是啊。嘛,和我们没什么关系就是」

「那么漂亮不也挺好吗」\\

因为两人完全没有交往关系,说实话白色圣诞节之类的其实和他们没有什么关系……不过,既然真昼喜欢,那下雪也不坏吧。

飘舞的小小雪花渐渐为夜幕低垂的世界饰上一层银白的淡妆。按照这个样子,就算雪下个不停,最后似乎也不会有多少积雪。\\

「不过,要是雪下得太大了会让公共交通瘫痪的,还是适度最好」

「这倒现实起来了啊」

「因为人只靠浪漫是活不下去的」

「说的是」\\

能有这样的对话,或许也是多亏了雪天所赐。\\

两人相互轻轻笑了笑之后,真昼站了起来。\\

「那我去把晚饭拿来咯」

「诶,拿来?」

「我先在那边做好了炖牛肉了。再怎么说,烤一整只火鸡两个人也吃不下吧……」

「我根本不会想到拿整只火鸡来烤啊」

「只是周君不会做菜而已……明天午饭就在蛋包饭上盖浇炖牛肉吧」

「听上去好好吃……」\\

这种东西吃之前就知道肯定美味,所以今天晚饭还没吃,就开始期待起明天的午饭了。\\

「我鸡蛋喜欢煎熟一点」

「巧了,我也更喜欢这种传统的方式。那我去把锅拿来」\\

周呆呆地望着从周家里啪嗒啪嗒地暂时回到自己家的真昼的背影,回忆起了吵吵闹闹的白天。\\

被发现真的是在预料之外。

因为原本就被怀疑着,如果只是疑念加深还是预料之内的……然而周根本没想过,那个时刻真昼竟然会露面。\\

结果上,情况都讲清楚了,也得到了理解……然而,周的心情还是有一点点的复杂。\\

——两人之间的秘密,再保持一会儿该有多好。\\

(想什么呢我)\\

不用再对两个人到处藏着掖着了,生活应该会轻松许多才对。然而周却感受到一阵微妙的郁郁的心情,他自己也困惑着不知道是怎么回事。\\

结果上看并不算差,可是却总有哪里觉得不舒服。\\

「怎么了吗?」

「……没事」\\

抱着锅回来的真昼看到周的样子疑惑地歪着脑袋,不过再怎么说周也不能把这说不清道不明的情绪吐露给她吧。\\

看着如同掩饰一样露出平常表情的周,真昼至始至终都是一副摸不着头脑的茫然模样。\\


\subsection{天使大人的幸福的味道}

% 「……はー、うまかった」\\
「……呼,真好吃」\\

%  相変わらず、真昼の料理はうまかった。\\
真昼的料理一如既往地好吃。\\

%  クリスマスという事でいつもより手の込んだ料理が出ていた。
因为是圣诞,做出来的是比平时更加精致的料理。

%  真昼によりじっくり煮込まれたビーフシチューはポットパイにされて、割りながら食べ進める形となっていた。\\
真昼用小火炖透的炖牛肉做成了肉馅饼,现在两人正边切边吃着。\\

%  パイを割る楽しみを味わってからのサクサクの食感にビーフシチューのコクのあるソースを絡めて食べるのは、至福のひとときとしか言えない。\\
享受完切饼的快乐之后,吃上去那酥脆的口感配合炖牛肉浓郁的酱料,正可谓是幸福一刻。\\

%  パイ生地をわざわざ仕込んだらしい真昼の謎に高い技術力に感服しつつ、本日二つ目のケーキを平らげたところで、一息つく。\\
真昼似乎是特意将其和进了面料里。周佩服着真昼谜一般的高技术,在吃完了今天第二个蛋糕的时候歇了下来。\\

%  ちなみにケーキまで真昼の手作りである。\\
顺带一提,连蛋糕都是真昼亲手做的。\\

%  ポットパイ用のパイ生地を仕込んだついででもう一つ菓子用の生地を同時に仕込んでいたらしく、ミルフィーユを作ってくれた。最早職人レベルである。\\
和进肉馅饼面料的同时似乎也顺便和进了另一份甜点用的面料,最后做出了千层酥。这已经是店里师傅的级别了。\\

% 「お粗末様でした。……よく食べましたね」
「你喜欢吃就好。……吃的还真是多啊」

% 「ん。美味しかったからな」
「嗯。因为好吃」

% 「それはありがとうございます」\\
「谢谢夸奖」\\

%  微かな笑みも、見慣れてきた。\\
周也逐渐开始习惯真昼的微笑了。\\

%  彼女は美味しいというと安堵したような笑みを浮かべるので、それを見るのが日課のようなものだった。
每次夸奖说好吃的时候,她都会露出安心一般的笑容,所以看到这副笑容已经像是周的日常一样了。

%  普段の表情よりもずいぶんと柔らかい顔が浮かぶのを見るのは周の特権のようで、なんだかくすぐったさがある。\\
看到这比平时柔和的多的表情就好像是周的特权,这让周心里莫名痒痒的。\\

% 「……明日はオムライスか……すげえ楽しみ」
「……明天是蛋包饭吗……超期待的」

% 「オムライス好きなのですか」
「喜欢蛋包饭吗」

% 「卵料理全般好き」
「有蛋的都喜欢」

% 「ああなるほど……だし巻き玉子とかすごい勢いで食べてましたからね」
「啊原来如此……记得蛋卷之类的都吃的狼吞虎咽的」

% 「うまいんだから仕方ないだろ」\\
「好吃嘛没办法」\\

%  いくら好物の卵料理でも、不味ければ食べない。あんなに食が進んだのは、真昼の料理が美味しいからだろう。\\
就算喜欢吃蛋,要是不好吃也是不吃的。能那么有食欲,还是因为真昼的料理美味吧。\\

%  独り占めしているのはすごく贅沢なんだろうな、と思うものの、誰かに譲る気はない。真昼が作るのをやめるまでいただき続ける所存である。\\
尽管周心里觉得自己独占非常奢侈,不过周并没有让给别人的想法,而是准备一直享受到真昼不做为止。\\

% 「……周くんって、ご飯食べてる時はすごく幸せそうですよね」
「……周君啊,吃饭的时候一脸幸福呢」

% 「事実幸せというか、真昼の料理がうまいからな」
「事实嘛,不如说是因为真昼做的好吃」

% 「それはありがたい限りですけど、お安い幸せですね」
「夸奖是谢谢啦,不过你这幸福还真廉价啊」

% 「いや割と高いぞ……お前自分の価値を把握しろよ……」\\
「不不还挺贵的……你搞清楚自己的价值吧……」\\

%  なにせあの天使様の手料理である。一部の男子には喉から手が出るほど食べる権利がほしいだろう。\\
毕竟是天使大人亲手制作。这可是一部分男生梦寐以求的权利吧。\\

% 「私にとっては毎日作ってるものですからねえ」
「对我来所就只是每天在做的东西呢」

% 「幸せもんですなあ俺も」
「我也真是幸福啊」

% 「……そんなにです?」
「……至于这么夸张吗?」

% 「そりゃ、うまい料理毎日食えてる訳だし」\\
「那是,毕竟每天都能吃到好吃的料理」\\

%  基本的には物欲があまりない周にとっては食欲の方が強く、毎日美味しい料理を出来立てで食べられるというのが一番の幸せだ。\\
周基本没什么物质欲望,还是食欲更加强烈,能每天吃到新鲜出炉的美味料理就是最大的幸福。\\

% 「どうやってこんなに料理を作れるようになったんだ」
「怎么样才能这么会做菜啊」

% 「お世話してくれていた人が教えてくれました。『必ず幸せにしてくれる人の胃袋掴むのよ』って」
「以前照顾我的人告诉我的,『如果有人能给你带来幸福,一定要抓住他的胃』」

% 「すまんな俺なんかの胃袋掴ませて」
「抱歉啊让你抓住了我这种人的胃」

% 「予行練習という事で」\\
「就当是预先练习了」\\

%  くすりと小さく笑んだ真昼に、不覚にもどきりとした。\\
看到真昼那小小的笑容,让周不禁心里一跳。\\

% 「……しかしまあその世話してくれた人もすげえな」
「……话说回来那个照顾你的人也很厉害啊」

% 「そうですね、あの人はすごく料理がうまかったですから。まだまだ私はあの人には敵いません、あの人の料理は、幸せの味がするんですよ」\\
「是啊,那个人的料理相当好吃。我比那个人要差的好远,那个人的料理能尝出幸福的味道」\\

%  微かながらに柔らかな笑みを浮かべて少し遠い目をした真昼に、周はひっそりと安堵した。\\
看着真昼微微露出的柔和的笑容和望着远处的眼神,周默默地感到了安心。\\

%  この言い方であれば、真昼はその世話役の人間に可愛がられていたであろう。真昼からもその人物を慕っているのがよく分かる。
听这个说法,真昼应该是很受那个照顾她的人疼爱吧。并且也能看出真昼同样敬爱着那个人。

%  親に見向きもされなかった代わりに、その人が真昼に色々と大切なものを教えてくれたようだ。\\
父母对她冷眼相待,而那个人则代替父母教会了真昼很多重要的事情。\\

%  そんな人が真昼の側に居てくれたのは、本当に僥倖だろう。
那样的人能在真昼身边,真的是万分侥幸吧。

%  恐らく話ぶりから女性だと思うが、彼女が居てくれたからこそ、真昼はこうしてまっとうに生きているのだと思う。\\
听语气那个人应该是女性。正因为有她在,真昼才能像现在这样过着正经日子吧。\\

% 「余程うまかったんだろうな。まあ俺にとっちゃお前のが幸せの味なんだが」\\
「听上去好吃得不得了吧。不过对我来说你的才是幸福的味道」

%  母親はさておき、父親の料理もうまいが真昼の方が周の好みの味付けだ。\\
母亲先不管,虽然父亲的料理也很好吃,但是真昼的更符合周的口味。\\

%  真昼の料理は、毎日食べても飽きないようなほっとする味がする。心休まる癖に心踊る料理で、全然飽きる気配がないしもっと食べたいとすら思うのだ。
真昼的料理是那种每天吃都不会腻的安心味道,让人心情平静的同时又让人满怀期待,不仅完全吃不腻甚至越吃越想吃。

%  まあ流石に真昼の負担が大きすぎるのでそんな事は言わないが。\\
不过真昼的负担已经太大了所以周是不会说出口的。\\

%  うんうんと頷いていたら、真昼が固まっていた。\\
周嗯嗯地点头之后,就看到真昼愣住了。\\

%  虚をつかれた、と言えばいいのか。
或许该说是猝不及防吗——真昼以呆呆的,毫不掩饰稚气的表情看着周。\\

% 「……真昼?」
「……真昼?」

% 「え、……なんでもないです」\\
「诶……没事」\\

%  声をかけられて我に返ったらしい真昼が慌てて首を振って、俯く。
听到声音后回过神来的真昼慌忙摇头之后低下了头。

%  お気に入りのクッションをぎゅっと抱き締め、そっと吐息をこぼしている真昼は、先程とは打ってかわって妙に色っぽさを感じた。\\
真昼紧紧抱着喜爱的坐垫,轻轻吐了口气。与刚才的样子一变,现在的真昼能感到有奇妙的魅力。\\

% 「どうかしたのか」
「怎么了吗」

% 「……単に、私なんかが幸せの味作れていたのかな、って」
「……只是想着,我这种人也能做出幸福的味道吗」

% 「何で卑下してるのか分からんが毎日食いたいくらいにはうまいぞ」
「虽然不懂你自卑的理由,不过你做的好吃得我想天天吃啊」

% 「……ありがとうございます」\\
「……谢谢夸奖」\\

%  ちら、とこちらを見上げて少し照れ臭そうに目尻を下げて小さく微笑んだ真昼に、今度は周が俯いて顔を隠したくなった。\\
真昼朝着周往上瞄了一眼,露出了有些害羞而又满足的表情轻轻地微笑着。倒是周想要低下头埋起自己的脸了。\\

%  本当に極たまに見せるこんな表情は、異性として好きという訳でなくても心臓を否応なしに跳ねさせる。\\
这极少给人看到的表情,即使心里没有作为异性的喜欢,也让周的心脏不由分说地跳动起来。\\

%  いつもの仮面を剥がして無防備といってもいい笑顔を見せてくれる真昼に、周は今すぐ顔を冷やしたい気持ちで一杯だった。
真昼取下平时的假面,露出了这能说是毫无防备的笑容,而周现在心里全是想着要让脸冷却下来。

%  じわじわとせり上がってくる熱がばれてしまうのは、嫌だ。お互いに照れていたら、確実に気まずくなる。\\
这慢慢涌上来的热量要是暴露出来就不好了。要是两人互相害羞的话,一定会发生尴尬的。\\

% 「あー、その、……そうだ真昼」
「啊,那个……对了真昼」

% 「はい」
「嗯」

% 「明日は、昼からでいいんだよな?」\\
「明天是中午开始对吧」\\

%  この空気に耐えきれず強引に話題を変えてしまったのだが、真昼はさほど気にした様子はなく周の言葉に思案している。\\
周受不了这样的气氛,强行改变了话题,而真昼没有在意的样子考虑着周说的话。\\

% 「はい、そういう約束ですよね? お昼ご飯作って、それから約束のげーむ、する……んでしたよね」
「嗯,就是这么约好的吧?中午做好饭,然后就是约好的游戏……对吧」

% 「おう」
「好」

% 「いや……でしたか?」
「不乐意……吗?」

% 「違う、確認しただけだ。……本当に、イブを過ぎたとはいえクリスマスにそういう過ごし方でいいんだな?」
「不是,只是确认一下……虽说平安夜已经过掉了,不过圣诞这种过法真的可以吗?」

% 「嫌なら言いません。……たのしみに、してます」\\
「不可以的话我就不会提出来了……我很期待」\\

%  また小さくふわっと緩んだ笑みが彼女に浮かんで、周は直視出来ずに「おう」とおざなりに返事して真昼と反対側の肘置きにもたれて羞恥を隠すしかなかった。\\
她脸上再次露出了小小的柔和的微笑,让周没法正视她,只能敷衍了句「噢」然后靠在真昼反方向的扶手上以掩盖住羞耻了。\\


\subsection{33 天使様とクリスマス}

% 翌日家にやって来た真昼は、少しそわそわした様子だった。\\


% よくある休日に異性の家に遊びに来た時の緊張……な訳もなく、真昼たっての希望でゲームをする事になったため、その興奮が漏れている、といった所だろう。


% なんでもテレビゲームなんて初めてらしい。その点では世間知らずのお嬢様と言ってもいいだろう。\\


% 「先にお昼ご飯作っちゃいますね」


% 「ん。堅焼きで頼むぞ」


% 「分かってますって」\\


% 注文の多い客に機嫌を悪くした様子はなく、エプロンを翻してさっさとキッチンに向かって昼ご飯の準備をしだす真昼は、おそらく上機嫌であろう。


% そんなに楽しみにしていたのか、と思うと妙に気恥ずかしいというか、くすぐったい。\\


% (まあ、ゲームを楽しみにしてるってだけなんだけどな)\\


% 決して、こうして二人きりで遊ぶ事が楽しみ、という事ではない。\\


% 束ねた髪がゆらゆらと揺れるのを見ながら、周はそっと苦笑をこぼした。\\


% \\


% 「……どうやって操作するのですか?」\\


% 昼食後、二人でテレビの前のソファに座ってテレビ画面を見つめていた。\\


% 何のゲームがしたいのかと聞いてみたら種類すらもよく分かっていなかったので、有名な国民的2Dゲームを起動してコントローラーを渡してみたのだが……やはりどうしていいのか分からずあたふたしていた。\\


% 「えーとだな、まあ移動はこのスティック、ジャンプはこのボタンで……」\\


% 基本的には冷静沈着なあの真昼がめちゃくちゃ戸惑いながらコントローラーとテレビを交互に見ながら操作していて、なんだかとてもほっこりとしてしまった。\\


% 慣れていないとはいえ、こんなにものんびりとしたプレイははじめてである。\\


% 避けもせずに敵に突撃してお陀仏になるのを何回も繰り返していると、天使様も苦手な事はあるんだなあというのを実感出来た。\\


% 「……勝てません」


% 「ステージクリアはおろか最初の敵すら倒せてないからな」


% 「うるさいです」


% 「まあ慣れだ慣れ。こんなの体で覚えていけ」\\


% 何事も挑戦だ、と言い聞かせると、真昼は素直にゲームに戻っている。


% 娯楽であるゲームに真剣な表情で挑んでいる真昼を見ていると微笑ましさすら覚えて、つい、笑みが浮かぶ。\\


% ただまあ、あまりに初手で負け続けているため、いつまで経っても進まない画面に段々と笑いより不安が勝ってきたのだが。\\


% 彼女がこちらを見てくる。


% 表情にむすっ、と効果音がつきそうなのは気のせいか。\\


% 「あーほら、ここはこうしてだな」\\


% 流石にこのままだとやる気に関わるので、周は彼女が握るコントローラーに手を添えて一時的に周がお手本という事でプレイしてみせる。\\


% 周もこのゲームは幾度となくクリアしているので、彼女がつまる場所も難なく突破出来る。


% というか真昼が下手くそすぎるだけで、普通の人間はここではひっかからないのだが……それは黙っておいた。\\


% 「ほら、この敵は一定速度で不規則に移動するが、こっちを視認するとキャラクターに向かって速度を上げて近付くんだ。タイミング見計らって跳んで……」\\


% 小さな手に重ねるようにしてコントローラーを握って操作、分かりやすく説明しながらお手本を見せる。\\


% 画面では、周が説明した通りにキャラクターが動き、敵を避けていく。


% なんて事のない動きなのだが、失敗し続けた真昼には新鮮だったらしく「おー」と感嘆の声を漏らしていた。\\


% 長い睫毛に縁取られた瞳がぱちりと開かれ、表情も明るくなる。


% 距離が近い故に下睫毛まで長いんだな、なんて新たな発見をしつつ、喜んでいる真昼を眺めて、小さく笑った。\\


% 端整な横顔を眺めていると、視線に気付いたのか真昼がこちらを向く。\\


% 彼女の手にあったコントローラーに触れるように近付いていたため、思ったよりも距離が近い。


% というか、二の腕と手が触れ合っているし、なんなら彼女の吐息がほんのりと肌を撫でるくらいには、近かった。お陰で、真昼の温もりと甘い香りが直に伝わってくる。\\


% 「ごめん」\\


% ほぼ真昼の手を包んでいた事に気づいて体ごと慌てて離すと、真昼はぱちりと大きく瞬いた後今更ながらに近かった事に気付いたらしく視線がさ迷い始めた。\\


% 「いえ……別に。こちらこそ、すみません」\\


% うっすらと色づき始めた頬を見て、やってしまったと後悔が襲う。


% あまり、真昼は接触が好きではない。いくら大分慣れてきたとはいえ、手を握られるというのは不愉快かもしれない。


% やや恥ずかしそうにしているものの、嫌悪感がないとも限らないだろう。\\


% 「ほんとごめん」


% 「あの、そこまで気にしてないですよ?」


% 「嫌じゃないのか」


% 「……びっくりはしましたけど、嫌とは。知らない人ではないですし」\\


% 寛大な天使様はどうやら無礼を許してくれるようだった。


% あっさり水に流してくれた真昼に安堵しつつ、ゲームを再開する。\\


% 今度こそ真昼にゲームを進ませようと画面を見て……やっぱり倒されている真昼の姿に、周はどうしたら彼女がゲームが上手くなるのか真面目に悩んだ。\\


% \\


% 結果的に一面をひいひい言いながらも何とかクリアしたところで、一旦このゲームは止める事になった。\\


% ずぶの初心者に死に覚えをさせ続けると、やる気に著しく関わってくる。他のゲームに目を向けてもらって、ストレスを抜こうという魂胆だ。\\


% 「真昼、傾いてる」\\


% という訳で次は現実世界でも馴染みがありそうなレースゲームをプレイさせてみたのだが……真昼の体が傾いている。


% このゲームはジャイロ操作は必要としていないし、コントローラーにもジャイロセンサーなどついていない。\\


% 体を傾ける必要は全くないのだが……本人は無意識なのか、コントローラーを持った状態で左右に揺れていた。\\


% 本人は、ゲームに集中しているのか、返事はない。\\


% 先程のゲームとは違い、車を操るゲームであり車に乗る機会がある現代人には馴染みやすかったらしい。学習した甲斐もあるのだろうが、つたない運転ながらもプレイ自体は出来ていた。


% 大真面目な表情でゆらゆらと揺れながら頑張って車を動かしている。\\


% (なんだこれ可愛い)\\


% 振り子のようにふらふらしている真昼が、妙に可愛らしい。大真面目に、そして一生懸命にやっているから、余計に可愛く見えるのだろう。\\


% 大きなカーブを曲がれば、自然と真昼の体も大きく傾く。


% ぽてん、と周の腿の上に体が倒れてきた時は、もう周も笑いをこらえるのに必死だった。\\


% 「……別に、体は傾けなくてもいいんだぞ?」


% 「わ、わざとじゃないです」


% 「うん知ってる。でも傾いてたから」\\


% 唇がぷるぷると震えるのをなんとか抑えつつ、真昼を起こしてやる。


% やはりというか、柔らかくて軽い。小柄なのはもちろんなのだが、折れてしまわないか心配なくらいには細くて、触る事をためらうほどだ。\\


% 周に起こされた真昼は、羞恥からか頬を染めて震えている。


% それがまた小動物のようで可愛らしく、とうとう耐えきれなくなって笑ってしまった。\\


% 「ば、馬鹿にしてませんか」


% 「いやいや。微笑ましいなと」


% 「それが馬鹿にしてるんです」


% 「俺が真面目にやってるやつを馬鹿にしてると?」


% 「そうは思いませんけど……」


% 「だろ? 単に可愛かったなと」


% 「……その可愛いは確実に子供っぽくて微笑ましいの意です」\\


% どこか拗ねたような響きの言葉で、あまりからかいすぎても不機嫌になってしまいそうなのでこの辺で感想を述べるのはやめておく。


% 内心でいくら思おうが顔に出なければ問題ないので、心の中でひっそりと思っておくようにしよう。\\


% ほんのり不服そうな表情を浮かべている真昼に小さく笑えば、ぷいとそっぽを向かれた。


\subsection{34 天使大人与圣诞礼物}

% 天使様が途中拗ねそうになるという事態はあったものの、その天使様もゲームをしていたらすっかり頭から抜けたらしくまた一生懸命な表情に戻っている。\\
尽管中途出现了让天使大人闹别扭的事态,但天使大人一回到游戏中,这些事情就被抛诸脑后,再次回到了一脸拼命的表情。\\

% ゲーム自体には大分慣れたのか、たどたどしいながらもプレイは出来ているし、なんとかついていけている。\\
虽然跌跌撞撞,但游戏本身真昼应该是大致熟悉了,还算能玩起来,于是她便坚持着玩下去了。\\

% 最初にやったゲームとは違い車を操る、というコンセプトのゲームだからだろう。
大概原因是这个跟一开始玩的游戏不同,是以控制车辆为主题的游戏吧。

% 本来のコースから外れてダートに突っ込んだり壁にぼこすかぶつかったりしているものの、それでも前進出来ている。\\
尽管总是脱离赛道跑上泥地啦撞上墙啦什么的,但还算是在前进。\\

% ゲーム下手な真昼の事なので逆走しっぱなしにならないかとか不安に思っていたものの、思っていたより順調に進んでいてほっとした。\\
本来周还在担心不擅长游戏的真昼会不会开起倒车,不过看起来真昼的表现比周预想的还顺利。\\

% ついでなので周も画面分割して一緒にプレイしているが、真昼が無意識の妨害をしてきてちょっと辛い。\\
顺便周让画面分成两半也玩了起来,不过因为真昼无意识的干扰,操作起来有点难。\\

% やはり彼女は自然と体を傾ける癖があるらしく、時折ぽすっと二の腕辺りに頭が寄りかかっては離れという事を繰り返している。
果然真昼似乎是有不自觉地歪身子的癖好,不时地把手臂或是头靠过来移开去。

% その度にふんわりといい匂いが漂うので、周としては落ち着かなかった。\\
而靠过来的时候空气就会泛起轻柔的香味,让周实在是冷静不下来。\\

% まあ、それでも最弱CPU相手なので独走するのだが。\\
嘛,虽说如此,但毕竟对手是最弱的电脑玩家,周还是能做到一马当先的。\\

% 「……なんでそんな早いのですか」
「……为什么你开的这么快啊」

% 「年季と慣れ」\\
「玩久了习惯了咯」\\

% 幾度もプレイすればコースは覚えるしコーナリングも自然と上手くなるものだ。相手からの妨害も、慣れればカメラワークや遮蔽物などを駆使してある程度は防げる。\\
多玩几遍的话,赛道也都记下来了,转弯也自然地熟练了起来。对手的妨碍也能利用视野和障碍物在一定程度上躲开。\\

% 納得のいかなそうな顔をしている真昼には苦笑を返して、そっと一人プレイに戻してやる。
周以苦笑回应一脸无法接受的表情的真昼,再次打开了单人模式。

% 彼女には経験が足りないので、大きな画面でまず練習させてからだろう。周の腕を見て自分の腕にがっかりするより、CPUに慣れていく方がいい。\\
考虑到她经验不足,周让她在大画面上先进行练习。比起看着周秀技术对自己失去信心,还是先习惯下电脑的操作为好。\\

% 幸い彼女はやる気があるらしく、一人プレイに戻っても熱心に画面を見つめていた。
还好真昼的干劲倒是很足,回到单人模式又开始认真地看向屏幕。

% この調子なら、まあ何とかCPU相手になら立ち回れるようになるだろう。\\
保持这样的话,再怎么说也应该可以跟上电脑玩家吧。\\

% 努力家な面がこういう所でも見る事が出来て、やはり微笑ましくてひっそりと笑いをこぼせば気配で分かったらしい真昼がぺちぺちと不服げに膝を叩く。
勤奋的人这一特质在玩游戏上也能体现出来啊——周想着这可真是让人不禁想要微笑啊,结果真笑出来让真昼发现了,惹得一脸不服的真昼砰砰地拍着周的膝盖。

% それが面白くて余計に笑えば、真昼が眉を寄せた後「周くんのばか」と小さく呟いたのだった。\\
但连这也很有趣,让周没有忍住,笑个不停,搞得真昼皱着眉头「周君大笨蛋」地小声嘟哝到。\\

% \\


% 「勝てました」\\
「赢了」\\

% 苦節二時間強。
苦战二小时有余。

% 画面の端に燦然と輝く一位の文字を得たままゴールを果たした真昼は、ほんのり自慢げに周を見た。\\
总算得偿所愿收获了在画面边缘闪耀的第一的文字的真昼,稍显自慢地看向周。\\

% 長い間テレビに向かって格闘してようやく得た栄光の一位。
面对电视苦斗良久总算获得的光荣的第一。

% 何度も何度もビリを経験して、それでも諦めずに走り続けて順位を少しずつのぼり、やっとの思いで一位をとったのだから、感激もひとしおだろう。\\
不知重复了多少次的倒数第一,却依旧没有放弃,一名一名慢慢地提升名次,最后总算获得了想要的第一,因此带来的感动也特别强烈吧。\\

% やりきったと言わんばかりの達成感のある表情に、周は素直に称賛の拍手を送った。\\
看着这如同说着「总算做到了」的洋溢着成就感的表情,周也直率地送上了称赞的掌声。\\

% 「よかったな。頑張ったの見てたぞ」
「太棒了。很努力了啊」

% 「はいっ」\\
「嗯!」\\

% 褒められて嬉しかったのか、いつもの表情が少し照れ臭そうに緩んでいる。
大概是被夸了很高兴吧,真昼平常的那副表情因害羞而稍稍柔和了一些。

% にこにこ、といった分かりやすい笑顔ではなくて、ほんのりと嬉しそうに緩んだ淡いはにかみは、いつもの彼女のクールさからは考えられないほど甘い。\\
并不是「嘻嘻」这样一目了然的笑容,而是略微泛起的夹带着喜悦的柔和的害羞神情,与她平时的那种冷淡感简直是天差地别。\\

% 最近普段のクールさの合間に年頃の少女らしい面を見せるようになってきた真昼だが、今日の真昼はいつにもまして年相応の顔を見せていて、無性に可愛らしかった。
虽然最近真昼那副平常的冷淡脸上,也时不时会露出与年纪相称的少女表情,但今天真昼这副比起以往更加与年纪相称的表情,实在是可爱到过分。

% どこか無邪気とも言えるあどけない微笑みは、周の理性の紐を緩めて頭をなでくりまわしたいという欲求を浮かばせるほどだ。\\
这泛着天真无邪感的微笑,甚至令周的理性开始松弛,脑子里开始冒出想要抚摸她的头的欲望了。\\

% 猫を撫でたいという欲求にも似た、可愛がりたいという衝動はつい腕に指令を出してしまって……思わず手が持ち上がりかけて慌てて下ろした。\\
顺着这股,想要抚摸猫的欲望般的,想要疼爱她的冲动,周的大脑不自觉地向手臂发出了指令……然后慌忙地收回了无意识间抬起了的手。\\

% 「どうかしましたか?」
「怎么了吗?」

% 「ああいや、なんでもない。うまくなったなあと」
「啊啊没,没什么。就是想着干的不错嘛」

% 「上達しましたか?」
「进步很大吗?」

% 「したした。最初に比べればすごくよくなった」
「很大很大。跟一开始相比简直是天差地别」

% 「ありがとうございます。楽しくて、つい頑張っちゃいました」\\
「谢谢夸奖。因为很高兴,所以自然地就很努力了」\\

% ふふ、とまた笑みを浮かべた真昼が見てられなくて、周は誤魔化すように部屋の棚にあったかごにいれておいた小さな箱を取り出した。\\
周实在无法继续看着「哼哼」地笑着的真昼,只好兼作糊弄地从放在屋内柜子里的篮子中把一个小箱子取了出来。\\

% 「一位のごほうびにこれをやろう」
「给,得了第一的奖励」

% 「え、あの、別にそんな」
「诶,那个,其实没必要」

% 「ごほうびが嫌なら白い髭をたくわえた恰幅のいいおじさまからの預かりものって事で」\\
「不要奖励的话就当作是从某个蓄着白胡子的胖胖的老爷爷那拿到的吧」\\

% そう、昨日うっかり渡し忘れていた、クリスマスプレゼントである。\\
如你所料,这是昨天周不小心忘了送出去的圣诞礼物。\\

% 誕生日とクリスマスがそう離れてないので再度プレゼントに困る事になったのだが、今回はまああてがあったので誕生日ほどの苦労はなかった。\\
虽然生日和圣诞隔得不久,要再选个礼物送出去有点难办,不过这回正好有个看中的东西,因而并没有像生日那次一样那么折腾。\\

% クリスマスプレゼント、という言葉に今がクリスマスという事を改めて思い出したらしくぱちりと瞬きをしていた真昼だったが、おずおずと受け取っている。
似乎是听到圣诞礼物这一词汇,才重新想起来今天是圣诞节这回事的真昼,啪哒啪哒眨了眨眼,然后小心翼翼地接过了礼物。

% 開けてもいいぞ、と声をかければ、また丁寧に梱包をほどいていった。\\
告诉她「现在开开也可以」后,真昼再次小心地解开了包装丝带。\\

% (まあ、大したもんじゃないんだけどな)\\
(嘛,虽然也不是什么贵重东西)\\

% 箱を開いてゆっくりと取り出したのは、レザー製のキーケースだ。\\
真昼打开纸盒,慢慢地从里面取出的,是皮制的钥匙盒。\\

% あまり高価なものでも気後れするだろうから、ブランド物という訳でなく、純粋にデザインとして真昼に合いそうなものを選んできた。
送太贵的东西反而会让对方很难办,因此周并没有选择大牌子的东西,而是单纯地挑了一件感觉和真昼挺配的东西。

% 花と蔦の模様が刻印されたシンプルなもので、普段使いには困らない程度のデザイン。あまり花には詳しくないので何が刻まれているのかは分からないが、繊細な形のそれはきっと真昼に似合うだろうという事で選んだ。\\
这是件印着花朵和常春藤图案的,设计上适合日常使用的朴素的钥匙盒。虽然周对花不怎么熟悉并不清楚上面刻着的是什么花,但看着那纤细的形状便觉得和真昼很配于是便选了这一件。\\

% 「ま、合鍵渡したしな。まあ使わないならそれでもいいから」
「嘛,不是给了你把多的钥匙吗。要是用不上的话就算了」

% 「いえ、ありがたく使わせていただきますよ。……周くんって思ったよりもセンスいいですよね」
「不,我会感激地使用的。……周君的眼光比想象中更好呢」

% 「思ったよりもって何だよ」
「比想象中是什么啦」

% 「いえ、普段スウェットとかジャージばかりですし……服装だけならセンス以前の問題ですし……」
「嗯,毕竟平常总是运动衫配运动裤……要论起穿衣那甚至是比眼光还基本的问题吧……」

% 「こんな機能性ある服は他にないぞ」\\
「穿着这么方便的衣服可没别的了啊」\\

% 真昼には着飾った姿なんてまず見せる機会がないし、そういうのは面倒でなるべく避けたいので、学生服か緩い部屋着しか見せていない。
毕竟首先就没有让真昼看见自己认真搭理外表的机会,再说周也是个怕麻烦的人,因而真昼见过的周只有穿校服和宽松的室内装这两种样子。

% なのでセンス云々の前にだらしないとかそういう印象がついてるのだろう。まあだらしないのは事実なので、払拭なんて出来そうにないが。\\
因此在谈及眼光之前,大概周已经给了真昼一种邋遢的印象吧。嘛不过周确实是邋遢,这点上周是想掖也掖不住。\\

% 「……ちゃんとしたら、かっこよくなるかもしれませんよ? 中学生の周くん、ちゃんとしてたじゃないですか」
「……要是好好打理打理说不定会变得挺帅的哦?初中那时候的周君样子不倒还挺不错的嘛」

% 「あれは母さんが無理矢理……待て何で知ってる」
「那是我妈强行把……话说你咋知道的」

% 「志保子さんが『ちゃんとしたらこんな風になるのにねぇ』と写真を……」
「志保子阿姨说着『要是好好搭理周他可是这个样子的呢』把照片给……」

% 「あんにゃろ」\\
「可恶啊那家伙」

% まさか母親の仕事に付き合わされていかにも外行き用の格好をさせられた時の写真が流出してるとは思ってもいなくて、周はここには居ない母親に内心で大量の文句を送りつけておいた。\\
实在没有料到因为母亲工作需要才勉强换上外出用打扮的周那时的照片会被流出的周,在内心里向着不在场的母亲送去了大量的报怨。\\

% 「……俺はああいうの似合わないから」
「……那样的打扮不适合我啊」

% 「そうですかね。周くん、他人とあまり目を合わせないようにしたり髪形で隠してるだけで、別に目鼻立ちは整ってると思いますけど……」\\
「是呢。周君,为了不和别人过多的对上眼神让前发遮住了眼睛,但其实五官看上去挺整洁的啊……」

% 小さな手が、周の顔に伸びた。
一只小小的手,伸向了周的脸庞。

% 伸びた前髪をかき上げるように白い手のひらが額に触れ、視界がいつもより広くなる。\\
为了将垂下的前发捋起一般,洁白的手掌触碰着周的额头,让周的视野变得比平常更加宽广。\\

% 風呂以外では久々に開けた視界で真昼を見れば、少しだけ驚いたような表情を浮かべる真昼が居た。
周以除泡澡以外许久没有过的宽广的视野看向真昼,映入眼帘的真昼脸上则浮现出略带惊讶的神情。

% 別に驚く事はない、不細工でも美形でもない普通の顔だろうに、こちらをじっと見る真昼が不思議でならない。\\
周想着这并不是什么值得吃惊的事情,自己的脸不过是个不算丑也不算帅的普通脸罢了,可盯着这边的真昼却是真的一脸不可思议。\\

% 「……なんだよ」
「……怎么了啊」

% 「いえ。前より瞳が生き生きとしてるな、と」\\
「嗯,感觉眼睛比以前有生气啊,这样」\\

% 数ヵ月前は目が死んでましたからね、と非常に失礼ながら否定出来ない言葉を送った真昼は、じいっと周を見上げている。
嘴里说着「几个月前还是一副死鱼眼来着呢」这般虽然非常过分却令周无法反驳的话的真昼,仰着头盯着周。

% そんなに見ても楽しいものではないだろうに、静かにこちらを見つめていた。\\
明明再怎么看也没啥好看的,真昼却一直静静地看着周。//

% なんだかこうして異性に、それもとんでもない美少女に凝視されるのは、気恥ずかしい。\\
这样子被异性,还是这般不得了的美少女凝视,周不知为何感到很羞耻。\\

% ただ、やられてばかりなのはつまらないのでお返しとばかりに頬にかかる真昼の髪に触れて綺麗な顔を露出させる。
不过,周也不是个甘愿一直被搞的人,作为报复没有多想便撩起了真昼的一束落在脸上的侧发,让她姣好的脸庞露了出来。

% 触るのには躊躇いがあったが、真昼が何の気なしに周の髪に触れたので、これくらいなら許されるだろう。頭を撫でている訳ではないのでセーフだと思いたい。\\
虽然对触碰真昼有点犹豫,但周转念一想反正真昼也没多想就碰了自己的头发,自己这样做应该也没事吧。反正也不是摸头,应该是安全的吧。\\

% (しかしまあ、ほんと美人なんだよなあ)\\
(不过啊,她这确实是个美人啊)\\

% 改めて見れば分かる、真昼の美貌の凄まじさだ。
重新这么一看,周再次体会到真昼那美貌的迷人。

% かつて周の部屋に落ちていた雑誌に載った美女とやらより余程彼女の方が綺麗で、魅力的だろう。\\
就算跟那曾经落在周屋子里的杂志上印着的美女相比,恐怕也还是真昼这边更加美丽迷人吧。\\

% そもそも、写真というのはあまり信用にならない。
再说,照片什么的本来就不大可信。

% 一瞬を切り抜いて加工出来るそれは、ありのままを写す事も、美しさを際立たせる事も、偽る事も出来るのだから。\\
不过是加工过的那抓拍的一瞬间,不论是素颜,还是美化,抑或是伪造都是可以做到的吧。\\

% 目の前に居る真昼は無加工でも可愛らしく、綺麗だ。\\
与之相对,眼前的真昼虽然毫无加工处理,却还是那么可爱而美丽。\\

% 飽きそうにない端整な顔立ちをじいっと見つめていれば、真昼が次第に視線をさ迷わせ始める。\\
周一直盯着这这令人总看不够的端正的面容,却让真昼的目光开始迷离了起来。\\

% 何なんだ、と思った瞬間には真昼が周の髪から手を離し、瞳を伏せる。
正当周想着怎么了的时候,真昼突然把手从周的头发上移开,然后垂下了眼。

% もぞ、と居心地悪そうにしている真昼は、コントローラーを完全に手放して側にあったクッションを抱き締めていた。\\
扭扭捏捏地,看上去很不自在的真昼,把手柄完全放开,然后抱起了放在一边的靠垫。\\

% 「あの。その……そうだ。私からもクリスマスプレゼントがあります」 
「那个。呃……嗯。我也有给周君的圣诞礼物」

% 「お、おう、ありがとう」\\
「哦,哦哦,谢了」\\

% 一体何なんだと問いかけようとして、真昼が話を遮るように側に置いてあった鞄からラッピングされた袋を取り出して周に押し付ける。\\
周正想开口问是什么,真昼却如同不让周问出口一般从放在一边的包里拿出了一个包好的袋子塞给了周。\\

% 「じゃあ、私夕ご飯の支度しますので」
「那,我就先去准备晚饭了」

% 「え? そ、そうか……?」\\
「诶?哦,哦……?」\\

% それだけ言い残してさっさと席を立った真昼に、周はあまりに早い展開で困惑するしかなかった。
说完这两句话,真昼便迅速从沙发上站起来,只留下对着这过于快速的发展感到困惑的周。

% 35 天使様とお正月の予定


%  クリスマスを過ぎれば、世の中は年末ムード一色になる。\\


%  夜景のためのイルミネーションこそ残されているが、あれだけ飾られていたクリスマスツリーはすでに撤去され、目に鮮やかな飾りつけは和のものに変わっている。


%  売り出しているものも全面的にお正月の飾りや食材になり、もう聖夜の面影は残っていなかった。\\


%  変わり身が早いものだな、とすっかり年越し準備に入りつつある周囲を眺めながら、周はマフラーに顔を埋めて温もりをとる。\\


%  モノトーンの千鳥柄のマフラーは、真昼にクリスマスプレゼントとしてもらったものだ。


%  なんでも首もとのおしゃれも大切です、との事で、非常に手触りがよくしっかりと風を遮って熱を溜めてくれる実用性と装飾性を兼ね備えた一品をいただいた。\\


%  普段マフラーなんてしなかったのでありがたく使わせてもらいつつ、腕に提げた買い物袋の中身を確認する。\\


%  基本的に買い出しは分担という事だが、料理を作る真昼の負担を減らすために基本は周がメモを携えて買い揃えている。


%  今日は寒いので鍋にするらしく、野菜やらきのこやら肉が袋に納められていた。野菜多めなのは、しっかり栄養をとりなさいという真昼の無言の主張だろう。\\


%  足りないものはないな、と改めて確認し、やはり厳しくなりつつある寒さに身を震わせつつ足早に帰宅した。\\


% 「お帰りなさい」\\


%  家に帰れば、夕方だったために真昼が出迎えてくれた。


%  赤の他人が家主を迎え入れるというちょっとおかしな事態ではあるが、最近は慣れつつあった。\\


% 「ん、ただいま。……薄切りの餅買ってきちまったけどいいか?」


% 「鍋でしゃぶしゃぶしたいのですね」


% 「おう。あと〆にラーメン買ってきた」


% 「……私、そんなに食べられませんよ?」


% 「俺が大半食うから関係ないな」\\


%  前はそう食べるタイプではなかったのだが、真昼の料理のお陰で晩ごはんは割と食べている。\\


%  彼女もカロリーに気を付けているのか食事は太らない程度のものであるが、彼女より量を食べる身としては微妙に心配なので筋トレしたりし始めている。\\


%  真昼としては、周は細いからもう少し肉をつけるべきでは? といった感想らしいので、なるべく脂肪ではなく筋肉をつけたいところだった。\\


% 「まあ、周くんが食べてくれるならいいですけど。それ、貸してください。冷蔵庫に入れてきますから。周くんは手洗いうがい」


% 「分かってますよっと」\\


%  真昼に荷物の入ったレジ袋を渡して、周は素直に洗面所に向かった。\\


\vspace{2\baselineskip}

% 「そういえば真昼は正月どうするんだ」\\


%  本日も相変わらず非常に美味な晩ご飯を平らげ後片付けをした所で、ふと気になった事を真昼に聞いてみる。\\


% 「正月……帰っても無駄ですしここに居ますよ」\\


%  あまりに淡々とした口調で返されて自分の失敗を悟ったものの、真昼はさして気にした様子もなさそうだ。


%  親との折り合いがよくないために、どうしても家族関係の話題はそっけない態度になっているのだろう。\\


%  ただ、そうなると真昼は一人で正月を過ごす事にならないだろうか。


%  周は基本的に半年に一度は顔を出す事、という約束があるため、真昼と出会う前は長期休暇は実家に帰るつもりでいたのだが。\\


% 「周くんは実家に帰るのですよね」


% 「そうだな、一応顔を見せろとは言われてるんだが」\\


%  ちら、と真昼を見ると、いつもの表情より心なしかひんやりとした眼差しの真昼が居る。\\


%  一人で過ごす事を当たり前だと思っているらしく、別に周が帰省する事を疑っていない。\\


% 「……帰ったらお前の事をしつこく聞かれそうでなあ」


% 「大変ですね」


% 「父さんは多分母さんの話聞いてそっかー程度で済ませるだろうけど、母さんは多分話聞きたがるからな」


% 「しょっちゅうやりとりしてるのに不思議ですね」


% 「ほんとお前いつの間にか母さんと馴染んでるよな……」\\


%  何故か母親といつの間にか仲良くなられた挙げ句知らない間に写真やら裏話が流出しているのか……とちょっと虚しくなるのだが、真昼もこの調子だと割と好きで相手しているようで、それならまあいいかという気持ちにもなる。


%  志保子にはまた余計な事を言うなよ、と釘を刺しておくとして、どうしたものかと真昼を見る。\\


%  時折見せる虚ろな表情や、寂しげな眼差しを思い出すと、どうしても……一人にしたくない。\\


% 「まあ、この間母さんとは会ったし、父さんには悪いけど今回は帰省しなくてもいいかなと。どうせ春休みに帰るし」\\


%  だから、彼女が迷惑でないのなら、いつも通りに夕食を共に出来れば、と思うのだ。\\


% 「……そうですか」


% 「ん。お前の年越し蕎麦食べたいし」


% 「食い意地はってますねえ」


% 「真昼の料理だからなあ」


% 「……ほぼ市販品なのに?」


% 「それでもだな」\\


%  たとえそばが市販のものを茹でただけでも、いいのだ。


%  二人でゆっくりと食べて時を過ごす、という事の方が重要なのだから。\\


% 「……変な人ですね」


% 「うっせ」\\


%  失礼な感想を述べてきた真昼にわざとらしく不機嫌そうに返してみれば、小さな微笑みが返ってきた。\\


% 「……ありがとうございます」


% 「何がだよ」


% 「何でも、です」\\


%  真昼はそれ以上は何も言わず、幾分機嫌がよくなったのか明るい表情を浮かべて、お気に入りのクッションを抱き締めた。



\subsection{36 天使大人与新年准备}

% 十二月三十一日、大晦日。
12月31日、除夕。

% その年最後の一日であり、年の締めくくりの日である。
今天是一年的最后一天,是为一整年画上句号的日子。

% 基本的には来年に向けての準備や大掃除をして慌ただしく過ごす一日なのだが――。\\
虽然基本上这一天都是在为新年做准备和大扫除中匆忙匆忙度过的——。\\

% 「あの、真昼さんや」
「那个,真昼啊」

% 「なんですか?」
「怎么了」

% 「……俺何もしなくて良いのか?」\\
「……我在这干闲着真的行么?」\\

% リビングのソファにゆったりと座った周は、エプロン装着で朝からキッチンに立っている真昼の背を眺めていた。\\
悠闲地在客厅的沙发上坐着的周,望着从早上就开始穿着围裙在厨房里忙活的真昼的背。\\

% 朝から来ているのは、おせち作りのためである。
真昼一大早就过来,是为了准备年菜。

% 二人で年越しをすると決めたので、当然おせちも二人前要る。\\
既然决定了两人一起跨年,那当然也要准备两人份的年菜。\\

% てっきり市販のおせちを買うのかと思いきや、なんと手作りするらしい。主婦でも大変な作業を華の女子高生が一人でこなすのだから驚きだ。\\
周本想着出去外面买一些对付过去,但看起来真昼是打算自己做的样子。连家庭主妇都感到头疼的家务,这位年芳十六的女高中生居然一个人搞定,实在是令人惊讶。\\

% すごいなと感心しきりなのだが、真昼いわく、\\
周不禁真心叹服真昼的能干,但本人却:

% 「そもそもそういうのって事前予約要るから無理です」\\
「就算要买这种东西也要事先预订,现在已经没办法了」\\

% との事。
如此说到。

% そう言われると確かにと納得してしまったのだが、それでもわざわざおせちを作ろうとしている真昼には脱帽である。\\
「确实是这样来着」听完真昼的解释,周虽然理解了真昼亲自动手的原因,但还是不禁对不辞辛劳亲自制作年菜的真昼脱帽致敬。\\

% もちろん手を抜けるところは手を抜くらしく、黒豆とかは煮るのに時間がかかるしコンロひとつ使えなくなる、との事で市販品を買ってきていた。\\
当然了,能省事的地方真昼也会省事,像黑豆这种东西煮起来花时间不说,还要占掉一个炉子,所以便直接去外面买来了成品。\\

% 「周くん、なにもしなくていいのとか不安そうにしてますけど、お手伝い出来るので?」
「周君,就算你因为在那闲着感觉不安稳,你觉得你能帮上忙吗?」

% 「出来ません」
「不能」

% 「でしょうね。邪魔されるよりは大人しくしていてもらった方が楽です」\\
「那不就是嘛。比起过来碍事,还不如在那老实待着更好」\\

% 実にシビアな観点の真昼に諭されて大人しくソファに座っているものの、やはりというかなにもしないというのは落ち着かない。\\
周乖乖遵从观点实在辛辣的真昼的旨意,老实地坐在沙发上,但什么都不干还是让周冷静不下来。\\

% 周とて、仕事を全くしなかった、という事はなかった。
虽然,周也不是什么活也没有干。

% 大掃除は昨日終わらせているし、 しばらく出掛けなくてもいいようにとおせちの材料を含めた大量の食材の買い出しをしてきた。
毕竟大扫除昨天就已经完事了,家里也已经屯好了包括年菜要用到的不出门也可以支撑一段时间的大量食材。

% 完全になにもしなかった、という訳ではないのだが、今の真昼に比べたら労力はかかっていないだろう。\\
虽然周不是什么也没做,但要是跟真昼比起来那就是没干多少事了。\\

% 「昨日は家具家電動かしてしっかり掃除しましたしお疲れでしょうから、ゆっくりしていてください」\\
「昨天你把家具家电这些都搬开来全部打扫也累了吧,今天你就休息会吧」\\

% 力仕事を担当した周を気遣うような言葉を向けた真昼は、相変わらずこちらには振り返らず調理を続けている。\\
说着关心着负责力气活的周的话的真昼,没有回头继续打理着料理。\\

% ちなみに真昼は自宅の大掃除は既に終わらせていたらしい。そもそも定期的に掃除をしっかりとしていたらしく、そう手間もかからなかったそうだ。\\
顺带一提,真昼自己家的大扫除似乎是已经完成了的样子。再说真昼她似乎有在认真地进行着定期打扫,大扫除并没有费她太多时间。\\

% 「いやー、それでもなんというか……悪いなあ、と」
「哎呀,就算是这样但还是感觉……抱歉呐,什么的」

% 「別に料理好きですから苦ではありませんよ」
「我也是喜欢做饭才做的并不觉得累哦」

% 「それでもさあ」
「就算是这样啦」

% 「いいんですよ、楽しいですから」\\
「没事啦,我很享受的」\\

% なんて事のないように告げて作業に集中し出した真昼に、周はどうしたものかと頭を抱えた。\\
说着没事的真昼再次集中精神在料理上,让周想着「到底该怎么办啊」的周抱住了头。\\

% \\


% 「真昼、昼ご飯買ってきたぞ」\\
「真昼,午饭买来了」\\

% 流石におせちで手一杯の彼女に 昼ご飯を用意させるのも酷なので、コンビニに行って適当なご飯を買ってきた。元々さほど量は食べない真昼なので、サンドイッチ一袋で問題ないだろう。
考虑着让已经在年菜上忙得不可开交的真昼再准备午饭实在有些过分,周便去出门去超市里随便买了些午饭。反正真昼本来食量就不大,一袋三明治应该就够对付过去了吧。

% そろそろ休憩に入ろうとしていたらしい真昼もエプロンを一旦脱いでいたので、タイミング的にも丁度よかったようだ。\\
真昼正好也暂时脱下了围裙,打算稍事休息的样子,从时间上来说是正正好。\\

% 「わさわざありがとうございます。そこまで手が回ってなくて申し訳ないですね」
「特意去准备午饭,谢谢你了。抱歉哦,我实在是抽不开空准备午饭」

% 「いやもうおせちつくってもらってる時点で圧倒的にこっちの方が申し訳ないっつーか……ほら、食べようか」\\
「哎呀要说的话拜托你准备年菜这边就已经很对不起你了啦……嘛,开饭吧」\\

% 休憩も兼ねての昼食であり、真昼は素直にリビングに戻ってきた。\\
到了兼作休息的午饭饭点,真昼也老实地回到了客厅。\\

% 「サンドイッチとコーヒーでよかったか?」
「三明治和咖啡没问题吧?」

% 「ええ、ありがとうございます」\\
「嗯,谢谢你」\\

% 周から渡されたご飯に小さく頭を下げて受け取り、周の隣に腰かける。\\
真昼微微点头,从周那接过午饭,然后在周身边坐下。\\

% 「ちなみにどんくらい出来た?」
「话说做怎么样了?」

% 「ある程度は既製品で賄ってますし、品目も抑えてますからほとんど終わってますよ。あとは冷まして詰めるのを待つものが多いです。周くんは伊達巻好きそうですからそちらは手作りしましたよ」
「算上有一部分是买来的已经做好的,另外也控制了下品种数目,现在差不多已经做完了。另外还剩下很多东西等着放凉了装起来。另外周君你好像很喜欢鱼肉末鸡蛋卷的样子,那部份是我自己做的哦」

% 「なぜ分かった」
「为什么你会知道啊」

% 「卵料理好きって言ってたでしょう」\\
「你说过自己喜欢鸡蛋料理的吧」\\

% 些細な言葉だったのだがきっちり覚えていたらしく、わざわざオーブンで焼いてくれていたようだ。オーブンの稼働音がしたので何を作っているのかと思えば、伊達巻だったらしい。\\
虽然对周来说只是一些琐碎的话,但真昼却认真地记下来了,还特意用烤箱去烤制了。听见烤箱的工作声,周还想着是在做什么,看来做的是鱼肉末鸡蛋卷。\\

% 「ほんのりと甘い感じがお好きですよね?」
「喜欢微带甜味的口感对吧?」

% 「よく分かってらっしゃる」
「你很了解我嘛」

% 「流石に数ヵ月もすれば好みくらい覚えます」\\
「再怎么说也已经好几个月了喜好什么的还是能记住的啦」\\

% なんとも嬉しい事を言ってくれた真昼は、ハムレタスサンドを口にする。\\
看起来说着很是高兴的事的真昼,开始吃起了火腿生菜三明治。\\

% 周も買ってきたおにぎりをかじりながらキッチンの方を見れば、目につくところに真昼が持参した小さめの重箱が置かれている。
周也边吃着买来的饭团,看着厨房那边,眼睛望着的地方放着真昼带来的小尺寸的重箱。

% あの重箱に詰めるのだろう。\\
应该是装在了那里面吧。\\

% まさか一人暮らしの身で重箱が出てくるとは思っておらず、漆塗りに金箔のあしらわれた高級そうな重箱が出てきた時はちょっとびびった。\\
明明是一个人住居然连重箱都有就已经让周始料未及了,当真昼拿出那个不但涂漆还贴了金箔的,看上去就很高级的重箱的时候,周是真的有点被吓到了。\\

% 「ほんと、ありがたい限りっつーか。……なんつーか、一人暮らし始めた時には想像出来ないぐらいに、今年の後半は充実した食生活だったなあ」
「实在是,让人只能说感激不尽了啊。……该怎么说呢,刚刚开始独居生活的时候还真是想象不到,今年后半年伙食上能这么充实啊」

% 「私としてはあなたが今までよく生きてこれたなと思ってますよ」
「我倒是想感叹亏你还能活到今天呢」

% 「ひでえ。案外コンビニとか市販品でなんとかなるんだぞ?」
「过分诶。其实超市啦卖的东西啦意外的还凑合哦?」

% 「健康的ではありませんね。まったくもう」\\
「但是实在是不健康哦。真是的」\\

% 呆れたようにため息をついている真昼だが、表情は仕方ないなと言わんばかりの苦笑混じりのもので、すこしどきりとしてしまう。\\
虽然真昼无奈地叹了口气,但望着她那仿佛说着实在是拿你没办法的混着苦笑的表情,周微微被惊到了。\\

% 「私が居るからには、不健康な食生活は許しませんよ?」
「我的话,可是不会允许不健康的饮食生活的哦?」

% 「おかんか」
「你是我妈么」

% 「周くんが不摂生だったのが悪いのです。来年はもっとしっかりとした食生活をしてもらいますからね」\\
「明明是周过的太不健康了。明年的饮食我可是要安排的更健康的哦」\\

% 微妙に気合いの入った真昼の姿を見て、来年も一緒にいるつもりで一杯なんだな、と思うと妙に気恥ずかしさを覚えて、目をそらす。\\
看着微妙的起来了干劲的真昼,想着明年也要一直在一起啊,突然莫名的害羞起来的周,偏开了眼睛。\\

% ただ、その態度を怠惰に過ごしたいという意味だと見なした真昼が少し不服そうに周を見たので、周は違うと言い訳するのに少し時間を費やす羽目になったのだった。
不过,把周这样的态度理解成想要过邋遢日子的真昼,以略带不悦的表情看向周,结果周为了解释清楚不是那个意思,花了好一阵功夫。

\subsection{天使大人与跨年}

太阳快落下的时候,真昼已经做完了所有东西并且塞进了重箱,现在开始准备起了晚饭。\\

话是这么说,因为是过年的荞麦面,所以只需要买来已经加工到煮之前那步的荞麦面、然后煮面并且准备好放到面里的佐料就好了。\\

鱼板是年菜的素材有剩,正好加进面里吧。菠菜只要烫一烫就行,葱也是只需要剁碎就好。

最费工夫的是鲜虾天妇罗,不过真昼对于麻烦的油炸食品也是丝毫没有不情愿的表情在油炸着。\\

「还有,南瓜多出来了,就顺便做成天妇罗了」

「哦……真是豪华的除夕荞麦面啊」

「偶尔来一次也挺好的吧」\\

如此说道的真昼最终完成的除夕荞麦面,比起老家吃的版本果然更为奢侈。\\

大鲜虾天妇罗准备了一人两只虾,旁边附带的南瓜天妇罗也是清爽的样子。里面有大量的菠菜和葱花,切成扇形的鱼板点缀在上面。\\

另外,真昼似乎喜欢先放面再放天妇罗。周的那一份也并没有直接把天妇罗浇在面上,而是分盛在另一个盘子里,这一点细节上的体贴让周非常感激。\\

「哇」

「来吃吧」\\

或许是觉得周只吃这些不够,真昼把年菜多出来的东西也盛在小盘里端了出来。

周看到真昼坐下,各自合掌对食物表达感谢之后,开始吃起了面。\\

虽然说是店里卖的东西,或许因为买来的是贵一些的荞麦面,周一咬下去,荞麦面的香味就扩散开来。

面汤也浓淡适中,是让人放松下来的,在寒冷的冬日中从胃里温暖全身的味道。\\

「哈……这才是快过年的感觉……」\\

周喝下面汤,长吐了一口气,发出了小声的感慨。\\

看着电视悠闲地吃着荞麦面等待新年的来临,这种感觉果然是不错。

在老家,每年吃着荞麦面看着年末特别节目和一年一次的歌曲节目已经是惯例了。今年能以同样的方式过年也是难得。虽然说在旁边的不是家人而是勤劳的外人少女。\\

「吃着荞麦面,一下子就涌出了一年过去的感觉呢」

「说的是……今年发生了很多事情啊」\\

虽然这么说,与真昼的交流占据了这「很多事情」的绝大部分。

刚开始独居生活的时候,周一丝一毫都没有想过会有这样的美少女来给他做饭。\\

「这是周君第一年一个人过日子呢,肯定会很辛苦的吧」

「你倒是挺习惯的啊」

「嗯,我大部分的事情都能自己完成呢。是什么都不会还想一个人过的周君不行哦?」

「唔……话是这么说啦」

「你可算是个让人头疼的人啊,真是的」\\

比起看不下去受不了,真昼责备周的时候更像是觉得愉快的样子。她的表情也很柔和。

真昼似乎不以照顾周为苦,始终都是一副温和的表情。\\

「……今年真的受你照顾了」

「就是说」\\

微微笑着的完全肯定稍微有些扎心,不过幸好真昼并没有不愿意的样子。\\

「……明年也请多多照顾」

「知道啦。周君要是没我,就是笔直奔向自甘堕落的颓废生活了」

「无法反驳」

「……知道的话自己注意点啊?」

「就当作明年的抱负吧」\\

恐怕就算有所留心,感觉也会让真昼勤勤恳恳照料一阵子之后使得决心融化掉。不过这样的想法还是留在心里没有对本人说出来。\\

当然,身边物品的收拾整理之类的周会去做——不过会拜托她做饭这事是不会有差了吧。

尽管周自己也发现自己离不开真昼的饭菜,不过周对此已经无可奈何了。\\

就算对真昼宣言要改善也只会被笑话,于是周摆出一副僵硬的表情,而真昼只是愉快地露出了小小的笑容。\\

\vspace{2\baselineskip}

「快要到新的一年了呢」

「是啊」\\

周吃完荞麦面,在沙发上看着歌曲节目,便觉得时间很快就过去,马上就是0点要进入新的一天了。

或许真昼只在必要的时候才会看电视,看样子她对现在的歌并不熟悉。她静静地,同时又开心地看着歌曲节目。周看着她这样子,时间就过去得比想象更快了。\\

画面变成了转播含有除夜钟\footnote{在午夜之前,日本全国的佛教寺庙会敲响108下新年钟声。}的风景,让周再次感受到新年即将来到。\\

坐在旁边的真昼俯着眼睛静静聆听着除夕钟声。\\

片刻之后,听到第107次钟声——\\

「新年快乐」\\

到达0点的瞬间,真昼看着周挺直了身子后弯下腰来,周也附和着端正姿势同样做出了新年的祝福。\\

「新年快乐……感觉有些不可思议啊,两个人跨年」

「呼呼,是啊……今年也请多多照顾」

「我才是……不如说应该是我拜托你来着」

「这个没法否定呢」\\

周朝着哧哧笑着的真昼苦笑出来,同时注意到了膝盖上的手机在振动。\\

似乎是树和千岁他们发来了新年的祝福,App的图标上多出了几个数字。

而真昼也是一样……不如说或许真昼那边更多吧。她的手机也在振动着。\\

最近发条消息就能完成新年祝福,真是变得方便了。\\

「我稍微回几条」

「我也是」\\

恐怕真昼那发来了大量新年祝福吧。不过周莫名觉得,真昼应该没有告诉过男生联系方式。\\

看着真昼熟练地连按着屏幕打字回消息,周佩服地感叹着「这方面倒是很像女高中生啊」同时自己也给树和千岁准备起回信。

他们的消息既有普通的『新年快乐』也有『和椎名要好地跨年了吗?』这种多余的窥探,虽然被说中了不过周还是回以否定。

很快,树就回了一句像是在开周玩笑的『少来了』,于是树也重复着被开玩笑和否定的过程,享受着这段对话。\\

然而,周的上胳膊突然压来一阵重量。接着,周感受到了空气中飘着的甜甜香气。\\

周战战兢兢地往旁边看去,便看到了合上眼睛的真昼靠在了周的胳膊上。\\

(——等下等下等下)\\

虽然没有发出声音,但周现在其实相当惊慌失措。\\

尽管以前也有打瞌睡这事发生过,然而谁能想到,真昼会在自己旁边,而且还靠在自己肩膀上睡着呢。\\

真昼为什么会睡着,这不需要考虑也能明白。\\

现在的时刻是已过零点半的深夜。\\

作息规律的真昼应该不怎么会熬夜,而且今天她一直在忙活做年菜,尽管没有表现出来不过想必是相当劳累了吧。

肯定是她没有抵抗睡魔的体力了吧。\\

理由是可以理解的。\\

尽管如此,却没想到会偏偏在这个时候中途睡着。\\

靠在周身上睡着的真昼,仿佛无视周的混乱和狼狈一样,露出了安详的睡脸。长长的睫毛、端正的鼻梁、粉红的嘴唇,都处于无防备的状态之中。

尽管周并非第一次见到真昼的睡脸,但距离这么近却还未有过,使得周的身体僵硬起来。\\

「真昼,醒醒」\\

周保守客气地说了一句,不过没有得到反应。

或许真昼是相当疲劳了,现在她似乎陷入了深深的梦乡,跟她讲话轻摇她肩膀都没有醒来的样子。\\

轻轻拍她的腿,摇晃接触的身体,都没法让她起来。\\

做出这样的事情之后,真昼靠着的部分偏移了一点,开始往前倾斜,周便慌忙接住真昼拉了过来……结果却意外地形成了抱到怀里的姿势,让周变得更加慌张了。\\

(……味道真香啊)\\

吃完饭后,真昼回了一次家,做了些洗澡之类的事情。或许是洗发露的花香加上本人的味道,有一股清甜的香气,让周非常的不自在。

另外,还感觉有什么柔软的东西接触着,让周更加冷静不下来。\\

要叫醒她也因为睡得太熟而于心不忍,并且周甚至感觉如果不来点强硬方式的话是根本叫不醒的。\\

(该怎么办啊)\\

新年刚开始就遇到这样的事件,让周抱住了自己的头。\\

\subsection{无防备的天使大人}

新年伊始,便被不得了的事态袭击了的周,以僵硬的表情看向怀里的真昼。\\

她真的睡得很熟。\\

真昼大概是想着「周是可以放心的人」,毫无警戒地熟睡了,而周在着急感与害羞感的侵袭下,理性已是摇摇欲坠,真的想要一头撞在墙上。\\

明明不想,可周还是忍不住把意识集中在真昼的触感上。\\

这副纤细的身体,虽然紧致却不失柔软,无处不体现着那女性特有的娇柔。

特别是在身体互相接触的部分,比起画面,那富有质量感的感触,更加无情地削磨着周的理性。\\

(——这可怎么搞啊)\\

过于出乎意料的事态,与这从未感受过的柔软一起向周袭来,令周陷入了极大的混乱之中。\\

女孩子居然这么又软又香啊……对这第一次知道的事实,周产生了微妙的感慨,但接着理性便突然刹车阻止周产生非分之想。\\

周意识着怀中传来的柔软感,大脑陷入了一片混乱,以至于都得刻意阻止自己产生非分之想。\\

虽然周还是试着思考该如何解决这一事态,但他感觉完美平安地解决是不可能的。\\

总之,周还是总结出了三个解决方案。\\

一、强行让真昼醒来

二、搬回真昼的家

三、让真昼睡周的床自己睡沙发\\

第一个的话,主要问题是周不想把现在正熟睡着的真昼弄醒。毕竟是自己的原因才让她这么累的,可以的话还是想让她能安稳地睡着。\\

第二个的话,初看上去应该是风险最小的,但是,这样面临着要掏真昼的衣服找到钥匙然后擅自进入女性的屋子这样的大难题。做到这个地步的话,就算是真昼,事后知道了也可能会对自己产生厌恶吧。\\

那么第三个,让她睡自己的床这一选项应该是最为安全而且容易实施的了……但要是这么做,周有自信自己精神上会死掉的。\\

就算平常两人就一直在一起,让露出了谁见了都会迷上的天真可爱的睡脸的真昼睡在自己床上,周有种自己的理性啊什么的会坏掉的预感。\\

让女孩子睡在自己的床上这种场景,已经是让男生欲罢不能了,好巧不巧对方还是个勤劳努力的美少女。

尽管这么做可能让真昼有些不满,不过这也是没办法的事。\\

但是,这就是最安全,也是周能做出的最好的体贴与妥协了。\\

下定决心的周,各用一只手放在了靠在自己身上的真昼的背上和膝内,缓缓地把她抱了起来。

也有睡着的缘故吧,真昼的身体轻得如同羽毛——那倒也不至于,但真昼的身体抱起来感觉还是很轻。\\

虽然感觉应该没那么容易弄醒,但周还是尽量平稳地把她抱到了自己的屋子里。横抱的状态下门把手开得很是勉强,但过了这一道坎之后,就只要让她躺在床上就好了。\\

纤细的躯体沉入床中。

周把毯子和被子给真昼盖好,便完成了晚安的准备。\\

真昼没有要起来的样子,传进周耳中的只有规律的呼吸声。

仍带着几分幼气的端正美貌,在平日里的美丽之上又添上天真的睡脸,令周不禁心跳加速。\\

让真昼好好地躺在床上之后,周在床边蹲了下来。\\

(……难受啊)\\

要论的话,真昼睡在自己床上的这一场景、柔软的感触、这毫无防备的可爱睡脸、在男性家里能睡着的信赖、由此产生的不设防,这一切的一切都是原因吧。\\

当然被如此信任周也很高兴,但却也让周不由得感觉自己完全没被当成男的看。

估计在真昼眼里,周只是个『实在没用的必须要人照顾的安全放心的无害的男孩子』吧。\\

周偷偷瞄了一眼真昼,但真昼则对周内心的纠结毫无察觉,依旧是一副安宁的睡脸。\\

(睡得那么香,都不知道我的烦恼)\\

既然这么没有防备,那要不然我也躺进去好了……周一瞬间闪过了这样的念头,但转念一想,两人没有交往的情况下,一起睡那就实在是太过分了,便否决了冒出的想法。\\

要真这么干了,感觉真昼起来的瞬间可能就不肯对自己说话了,而且还感觉她会以冷淡的眼神说出一句「你到底在想些什么啊」。所以,为了自己好还是不要付诸行动吧。\\

不过,只是稍微摸一摸应该情有可原吧。这么想着,周把手伸向了真昼的头。\\

丝滑如绢,舒润如绸,光亮如玉——周用指尖轻轻梳过这正如此般词汇所言的光泽长发,没有一丝阻碍便顺滑地直达发梢。\\

连这地方的保养也是十分上心啊——周一边对女性的努力感到赞叹和畏惧,一边轻轻地把指尖滑向了真昼的脸蛋。\\

也许因为真昼体温不算太高,那水润光泽的雪白肌肤,比起周的手来还略凉一些。

周轻轻抚过真昼的脸颊,而后,看着那无比安心的睡脸,静静地露出了苦笑。\\

「晚安」\\

明天……准确来说,今早醒来之后,她肯定会很吃惊吧。周这么想着,但又觉得,她都已经让自己这么抓狂了,这点小事应该算是容许范围之内吧。\\

真是拿你这个家伙没办法啊——周这般苦笑着,再次轻轻地抚过真昼那软软的脸蛋。

\subsection{天使大人的醒来与害羞}

早上,周起床之后也没有听到日常生活的声音。\\

家里安静得能听到窗外的鸟鸣,可见睡在周房间的真昼没有起来的样子。

时间上已经过了日出时分,不过真昼可能是因为昨天太累所以睡得很熟吧。\\

另外,要说周的话,尽管睡是睡着了,不过想着自己床上有真昼就不怎么睡得下,结果到最后也没能睡深,现在这个时间就起床了。

周身体上倒是并没有多难受,然而在另一种意义上却很难受。\\

周做了做拉伸以放松因为睡沙发而僵硬的身体,同时缓缓站了起来。

总之周打算先去观察一下真昼的情况。虽然说去拿衣服换才是主要目的,不过周也是顺便准备去观察真昼情况的。\\

周静静地打开自己的房间门。\\

里面一片安静,熟睡在床的真昼果然还是那副样子。\\

不过,要说不同之处的话,真昼或许是翻了几回身子而横了过来,头发也像河流一样摊在床上。\\

呼,呼,真昼发出着可爱的呼吸声,而周蹲下来望着她。\\

真昼睡觉时真的非常天真可爱。

也许真昼平时都绷紧了神经,很多时候都是挂着高冷的表情……而睡觉时,这松弛下来的表情实在可爱,可爱到让周想去摸。\\

(……睡觉时真是可爱啊)\\

当然,她即使起床也无疑是美少女而且很可爱,然而现在的周则更接近于观赏小动物时的感情。

想要抚摸那柔顺的秀发,想要戳戳那柔软的脸颊。正因为平时无机可乘,到了现在这样无防备的状态才更想要调戏她。\\

周情不自禁地把手伸向那看上去很软的脸蛋摸了摸。\\

光滑的脸蛋传到指尖的是和昨天一样的柔软。这软乎乎的样子让人想要一直摸下去,便不由得让周用手指肚戳了上去。\\

因为软软的很舒服,周就像是疼爱真昼一样的感觉摸了摸。尽管周有留意少使力气,然而还是让静静睡着的真昼发出了「嗯……」的沙哑而甘甜的声音。

接着,周还没来得及把手拿开,真昼合上的眼睛就缓缓睁开了。\\

一双焦点没有重合的,湿润的焦糖色眼睛……正是看着周的方向。

真昼松垮的表情上还留有稚嫩睡脸的余韵,十分天真可爱。不如说,惺忪的眼神明明有意识同时又在恍惚中,使得现在这个样子显得比刚才更加稚嫩。\\

真昼露出了疏忽大意显露无疑的表情,接着又落下眉梢再次闭上了眼睛。

周正想收回手指,真昼却把脸往周的手指上磨蹭起来,同时喉咙里发出了撒娇一般的咕噜咕噜的细声。她把脸蹭上来的样子,就好像是在说不要离开一样。\\

周明白这显然是睡迷糊了。

真昼没道理对周那么撒娇,而且平时的真昼也不会做出这么松垮的表情和动作。\\

即使如此——真昼做出的撒娇的小猫一样的动作,使得周的心脏和理性一大早就受到了考验。\\

是该收回手呢,还是顺着感情摸脸疼爱呢。\\

心情上,周相当倾向于后者。

这么软萌的真昼可是不怎么见得到,而且周也对真昼会撒娇到什么地步很有兴趣。\\

不过,周感觉,如果真的付诸行动的话,真昼清醒的瞬间就会闹别扭不说话了。由于周非常清楚真昼会羞耻得不能自拔,所以不知道现在应该如何是好。\\

总之,因为很可爱,周停留在了观察睡迷糊的真昼这一步。\\

尽管真昼的意识已经苏醒了相当一部分,不过不知道是因为脑袋不清醒还是因为没注意到这是周的手,现在真昼正把脸蹭到周的手指上打着盹儿。\\

周原本只是打算来观察情况和拿衣服来换,不知为何却变成了这样的身体接触。由于说不清的心痒,周感受到了自己脸上集中的热量。\\

「嗯,嗯……」\\

过了一阵子,也许终于是清醒了,真昼再度睁开了眼睛……\\

「……诶」\\

四眼相对,接着真昼把目光移动到旁边的周和碰到脸的手指,僵住了。\\

再接着,真昼一跃而起。\\

「早上好」

「……早,早上好……」

「你在我家睡着了所以把你搬这儿了。没有其他意思。我都想让你谢谢我什么都没干了」\\

周先解释了真昼躺在周床上的理由,这样真昼也没有吵闹而是老实下来了。

不过,因为睡在了男性的床上这个事实,真昼的脸渐渐发红,捏着被子提起来掩在嘴角上。\\

这套动作也微妙的很可爱,让周不禁别开了眼睛。\\

(这什么状况)\\

自己姑且是借出床铺的人,现在却搞得是自己不好一样。

确实擅自摸脸是很对不起,不过周只是稍微摸了一下下,也并没有打算要做什么事情。\\

周因为真昼的可爱心跳不止,又因为罪恶感而觉得痛心,心里五味杂陈。而看真昼那边,她的脸依旧一片朱红,有些略微的不开心……这倒不至于,但朝向周露出了有话要说的眼神。\\

「……周君喜欢摸脸吗」

「诶?」

「圣诞节那会儿,还有昨天睡觉之前你不都摸了吗」

「……你醒着啊」\\

昨天应该是真昼熟睡的时候摸的,本人应该意识不到才对。

然而真昼却知道有这事儿,说明那时真昼是醒着的。\\

「……那,那个是,嗯……放到床上的时候醒过来了……那个,除了装睡还能怎么办嘛」

「就不觉得我会做什么事情么?」

「……周君,应该是不会做那种事的……而且,也有为了确认这点,才装睡的,嘛」\\

周似乎是被确认是否真的可以信任了。\\

结果上看得到了信任是件好事,不过真是希望以后她不要做出在男人面前睡着这种没有戒备的事情了。

就算是周,也不觉得下次再见到的话能只戳脸就完事了。\\

「……嘛,能得到信任倒是好啦,以后别这么干了。我也是男的」

「那,那还是知道的,嗯」

「还是说想要我做出什么事来?」

「怎么可能啦」\\

满脸通红强烈否定的真昼又钻进了被窝里。周把「这是我的床啊」这句吐槽咽到了肚子里。\\

在真昼的害羞消退之前,周只能把窝成一团浑身发抖的真昼静静放在一边了。

\subsection{天使大人的害羞与不悦}

从羞耻中回过神来的真昼先回了趟家,换好了衣服回来了。\\

不过,似乎还在害羞着的真昼每每与周对上眼便会微妙地偏开视线,搞得周也开始尴尬了起来。

虽说万幸真昼还愿意一起坐在沙发上,可周却觉得如坐针毡。\\

「……原谅我吧」\\

不知为何总觉得不好受的周下意识地向真昼道歉,而真昼则瞄了一眼周,然后轻轻叹了一口气。

大概是脸上的害羞已经褪去了吧,真昼姑且算是恢复了一如往常的表情。\\

「我没有生你的气。周君并没有跟我道歉的必要」

「不过啊」

「我只是对被看见了那样见不得人的脸的不像样子的自己感到后悔罢了」

「见不得人什么的……其实只是普通地很可爱啊」\\

那不负天使这一外号的,实在如同天使般的睡脸、醒来之后的惺忪睡眼、还有那毫无戒备放松下来的天真表情,全都十分可爱。\\

与平时那冷静而沉稳的表情截然不同,在睡迷糊的时候真昼会露出十分幼气的表情,这是周的新发现。

这个表情可爱到了让周想要更多地看看的程度,不过真昼应该是不想自己疏忽大意的表情被看到吧。\\

周并不觉得那表情很不像样子或者见不得人所以想要否定那一部分,结果不知为何却让真昼咬着嘴唇用抱在怀里的靠枕嘭嘭地拍起了周。\\

周并不痛,真昼感觉也不是认真的,但周还是搞不明白真昼怎么突然就拍起自己来了。\\

「干嘛啊」

「……周君这种地方真的是不行呢」

「什么啊……那你要我咋样」

「这种事情是不能这么轻描淡写地说的」

「我又不是跟别的人这么说……」\\

数起周身边的女性,除了真昼和千岁就没了。

虽说千岁可爱倒也名副其实,但一提到她周下意识就觉得是个麻烦,因而也没必要当面称赞她,因此除了真昼周也没有谁能夸了。\\

看到僵住的真昼,周感到有些疑惑,耸了耸肩。\\

「唉,你的话早该习惯被这么说了吧?咋还这么介意」\\

再说周向真昼表达自己觉得她可爱早就不是一次两次了,事到如今真昼还在介意这里,让周匪夷所思。\\

真昼的话应该对自己长得有多漂亮心知肚明,被夸奖也应该早就习惯了。\\

照理只是被周一个人说了几句,不至于让真昼害羞成这样吧。\\

周这么想着,真昼却不知为何变得一脸不快。\\

「所以说你到底咋了从刚才开始」

「……什么也没有啦」\\

最后又嘭地用靠枕补了一发物理攻击的真昼,哼地扭过了头,丢下一句「我做年糕汤去了」后,便穿起围裙去了厨房。\\

手上拿着强塞过来的靠枕的周,一时间只得呆呆地望着心情有点不好的真昼的背影。\\

\vspace{2\baselineskip}

吃完年糕汤之后,真昼恢复了一如往常的表情。

刚开始吃年糕汤的时候真昼还板着个脸让周微微有些违和感,但这年糕汤和年菜都很美味,让周吃着吃着就入了迷,等周回过神来,就发现不知什么时候真昼的心情已经恢复了。\\

一起离开餐厅坐回沙发上的时候,一切都恢复了往常。\\

「说起来啊真昼,新年参拜你去不?」

「新年参拜吗?去倒是不大想去……毕竟我不喜欢人挤人的地方。总有种被盯着的感觉」

「那还不是因为你……」\\

是个不得了的美人啊——周正想这么说,却突然想起自己刚刚还坏了真昼心情,便把这话咽了回去,回答道「嘛这也是没办法啊」。\\

「周君打算去新年参拜吗?」

「在老家那时我都是跟着爸妈一起去的,不过现在还拿不定主意。至少我是想着没必要挤着元旦去」

「同意」

「千岁他们的话估计是在千岁家里培养感情吧,嘛要说的话现在的孩子们倒也不怎么会去新年参拜的样子。反正以后搞也没差」\\

要跟过去比的话……特别是十几二十多岁的这些年轻人们去做新年参拜的比例似乎少了不少,并不是说周这些人有什么特殊的。

虽然也不是不想去,但周明白人多得动都动不了只会让人筋疲力尽,因而想着等人少下来了再去也不迟。\\

「再说了,前三天还是想悠闲点过啊。我的话福袋什么的倒也不在意」

「我的话倒是对福袋有点兴趣呢」

「你是要去购物中心吗?」

「……我是没有朝着那人堆突击的勇气呢」

「同意」\\

周作出了跟刚才的真昼很像的回复,把身子靠在了沙发上。\\

反正,也不是说新年就非得去哪里不可。

大致上想着要避开麻烦事的周,只要能这样悠悠闲闲地过着日子便十分满足了。而且考虑到方便做饭,整个新年期间真昼似乎都打算在周家里过的样子,这下不管是聊天对象还是伙食都不用愁了。\\

这可真是个豪华的新年啊——周这么想着,偷偷瞄了一眼坐在一旁的真昼,微微地笑了起来。

\subsection{天使大人与初次见面}

『明天可以去周家里吗』\\

三号,真昼回去之后,周收到了这条父亲发来的短信。\\

『周你不回老家那就算了,但我还是想看看儿子的脸啊。顺便听志保子说了,感觉也得和邻居打个招呼』 \\

似乎是母亲她好好地把真昼这号人物和自家儿子受了人家多少照顾告诉了父亲,于是作为家长父亲便觉得有必要向她打一个招呼了。\\

要是志保子她不知道这回事的话周肯定是要全力拒绝的,但现在她不但知道了,真昼自己和志保子之间还有不少来往,周便觉得就算拒绝也是无济于事了。

反正事到如今也没啥要藏着掖着的了,周对父母来视察不回家的儿子这件事本身也没有什么抗拒感。\\

父亲——修斗如果和志保子一起来的话,反而是可以让容易暴走的志保子冷静下来。\\

周料到就算自己现在拒绝了,过后志保子也会强行跑来见真昼,便向先来约时间的父亲回了一个肯定的答复,然后给真昼发了条消息。\\

\vspace{2\baselineskip}

「嗯,那个,我待在你们家庭团聚的地方没问题吗。不会打扰你们?」\\

第二天一早真昼便来到周的家里,显得稍有些紧张。

某种意义上也是当然。毕竟正受着自己照顾的男生的父母突然说要想要见自己。\\

真昼似乎是暗地里和志保子有过联络……不如说是经常从志保子那里发来联系的样子,大概已经是熟悉了吧。光是志保子那还好,偏偏这回父亲也要过来,真昼会感到紧张这也是情有可原。\\

「我说啊,我爸是要跟你打个招呼才来的,我妈的话似乎很中意你的样子不如说你在她还高兴哩。这么一来反而是你不在不行」

「就,就算你这么说……」

「好啦别那么畏畏缩缩的啦。稍微忍一忍的话我会很高兴的」\\

让真昼和自己的父母打招呼这事虽然听起来有些超现实,但既然父母已经有了见面的意思,那就没办法了。\\

虽然占用了真昼的时间这点上有点对不起她,但从父亲的性格来看要是不跟真昼见个面怕是不会善罢甘休的,所以希望真昼能忍一小会儿。\\

「……志保子阿姨她是怎么介绍我的呢」

「放心吧。我跟父亲强调了好多次是恩人的关系,讲清楚了不是我妈那自我幻想时间里的那种关系」\\

好像在志保子她脑子里,真昼已经是儿媳,或者说是可爱的女儿了,所以周全力否定了这一点。

修斗那时也苦笑了会,回答说『是志保子她平日里的坏毛病呢』并接受了,这么一来应该就不会有误解了吧。

\vspace{2\baselineskip}

看着真昼放心下来抚过胸口的样子,周一边苦笑着说着「抱歉啦」一边等待着。正好这时门铃响了起来。\\

公寓大门的话他们手上有钥匙可以直接打开,因而周已经料想到他们会直接进到家门前。

看见真昼的身体悚地一抖,周一边微笑着安慰着她一边起身走向玄关,解开防盗链拧动了门把。\\

打开门之后,门前站着的是周已经见惯了的父母的身影。\\

「半年不见了呢周」

「好久不见,爸」\\

看见露出平和笑容的父亲——修斗,周也同样露出了略带安心的微笑。

身边萦绕着安稳的氛围的修斗,是那种在一起就能让人平静下来的性格,周也一见面就不禁放松了下来。\\

「对妈的时候就是一副鬼态度呢……」

「还不是妈你突然就不请自来。事先说一声的话我就会好好接待啦」\\

主要那时候有真昼在,所以周才是那样的一副态度,要是只有周一个人的话他的态度也会缓和些吧。\\

「总之,进来吧。……这提的都啥啊」

「带了各种各样的东西啦。嘛这个先放一边,小真昼呢?」

「里面」\\

周简短地回答之后,便陪着脱掉鞋子的父母回到客厅,正赶上稍稍有些坐立不安的真昼看向这边——她的眼睛瞪得大大的。\\

真昼的惊讶也不无道理。\\

修斗他那年轻的外貌实在难以想象已经是个三十大几的人了。就算除开从儿子眼里来看的加分,那容貌也还是三十左右的水平。

看着那几乎可以称得上是娃娃脸的年轻而端正的容貌,周已经不知几次想着要是能多继承点那基因该多好了。\\

由于这个男人有着与自己不同的柔和长相,看上去实在是个友善的好青年(虽然年纪上已经算是中年了),所以两人常常被怀疑血缘关系。虽说两人一起走的话看上去倒像是年龄差大的兄弟来着。\\

「小真昼,好久不见呢~」

「啥好久不见啊,还没一个月吧」

「在我心里已经好久不见了哦」\\

看到志保子跑向自己露出满脸的笑容,真昼也摆正了坐姿微微露出外出用的笑容回答说「很久没有见过您了」。

不过,真昼还是以略带困惑的眼神望向修斗,而注意到这视线的修斗则保持着一脸平和的笑容站在了志保子身边。\\

「初次见面。我是周的父亲,藤宫修斗。椎名的事情我已经听志保子说过了。儿子一直都受你照顾了」

「初次见面。我是椎名真昼。我才是,一直都在受周的照顾」\\

配合行了漂亮的鞠躬礼的修斗,真昼也做出了礼貌的问候。\\

真昼担心的,大概是修斗他会不会是跟志保子一样的性格。不过修斗是个温厚而有常识的人,所以真是希望真昼能尽快安下心来。

能控制住志保子的只有修斗一个,志保子也对修斗强硬不起来。虽说喜欢得一塌糊涂也是一个理由吧。\\

「哎呀,没必要那么谦虚哦?反正周是个邋遢仔啦」

「身为邋遢仔真是对不起了」

「好啦志保子,这种话不能说啦。……周,平常一直受人家照顾,有好好地感谢过人家没?」

「有尽我所能」

「那就好」\\

以「应好好对待女性」为教育方针的修斗,似乎是在担心周有没有好好感谢过真昼。

再怎么说,把事全部丢给真昼,自己在一边享受这种事情,周自己心里也过意不去。因而周觉得自己是最大程度上关照着真昼的。\\

修斗听了周的回复放下心来,再次与真昼合上了视线。\\

「……实在是,该怎么谢谢你才好呢。好像不但平常做饭都是靠你,连年菜都麻烦你来做了吧……」

「我一直都很感谢人家,也尽我所能慰劳她啦」

「嗯。……周君也意外地挺关心我的」

「意外是什么啦意外」

「毕竟嘛……」\\

「看上去一副大大咧咧的样子却很能注意到细节呢」,周听到真昼这么说,无法反驳大大咧咧这个事实,结果一时语塞,而看着的修斗则露出了柔和的笑容。\\

「看你们关系这么好就再好不过了。周你也不要给人家椎名添太多麻烦啊」

「……知道的」

「椎名也是,要是周有什么地方不好的话希望你能好好指出来。虽然看上去不像,但这孩子其实还是很坦率的,要是有你讨厌的地方应该会很快改正过来的」

「……周君很温柔,所以,讨厌的地方什么的……那个,只有一点点」

「有呢」

「……与其说讨厌……说是缺点更准确吧」\\

真昼略微害羞,好像有些难以启齿一样,搞得周都想问问自己到底是什么缺点会让真昼说起来这么害羞了……\\

志保子则是不知为何,似是有了头绪般「哈哈」地咧着嘴笑着看向这边。周能做到的就只有瞪着她说「搞什么啊」了。

\subsection{天使大人的憧憬}

「请用」\\

就算是亲生父母也一样是客人,该招待也是当然的,不过真昼坚持说要自己端茶上来,于是周便拜托她了。

周还真没想到,真昼为了自己喝而拿来的茶具和红茶,竟能在这种地方派上用场。\\

周的父母坐在这平时周和真昼两个人坐的沙发上,露出了满脸的温和笑容。\\

「哎呀小真昼真是谢谢啦,你已经完全适应了呢」

「嗯,是」

「这事原本应该得让周来做的哦?」\\

让周泡茶的话,恐怕只能泡出红茶的涩味,所以真昼才会亲自动手的。然而,这让志保子露出了有些无奈的表情。\\

「没有,只是我自己愿意的……」

「嘛,要是周来泡的话,热水温度太随便了,也没办法」\\

虽然说得并没有错,但是被指出这些还是有些让人不爽的。

话虽如此,周也无法反驳,只能老老实实地闭上嘴,结果却被志保子笑嘻嘻地看着了。\\

「说起来啊周,开始好好用名字称呼小真昼了啊」\\

听到这突然的一言,周和真昼都僵住了身子。\\

因为已经称呼得很自然了,所以周就把这事儿忘了。上次见母亲时,周还是用椎名这个姓氏称呼真昼的,而真昼叫周叫得也很别扭。

而现在,两人相互之间称呼得那么顺畅自然,就志保子那个性,肯定会胡思乱想的吧。\\

「……有什么关系」

「嗯,挺好挺好,关系亲密是件好事」\\

志保子故意没有进一步追问下去,只是眉开眼笑地观察着周这边。周则感觉到自己的脸上一阵抽搐。

说不定被开玩笑反而还更好一点。这种时候的志保子,脑袋里绝对是在快乐脑补着两个人这样那样的关系。\\

「志保子,别再逗周了」\\

不过,修斗这时踩下了刹车。\\

「志保子这习惯不好啦。别开周太多玩笑了」

「行咯,虽然很可惜但就算了吧」\\

只要修斗说的话志保子都会乖乖听,作为被折腾的儿子真是感激不尽。\\

「话说回来,看到儿子和可爱的女孩子关系那么要好,果然还是很棒的吧」

「我倒是一直担心着志保子的坏习惯会不会失控哎」

「哎呀,修斗会阻止我的吧?」

「虽然说自己都知道了最好还是改掉吧,不过志保子这种地方我也喜欢所以没办法呢」

「嘛……我说修斗你啦」\\

虽然说修斗是阻止了志保子,不过这次父母又开始微妙地形成了二人世界,周也不掩饰自己的叹息了。\\

修斗大体来说是个有常识的人,不过却会无意识间疼爱自己的老婆,有时会产生让其他人难以接近的气氛。

幸好这个样子只会在家人前才会表现出来,在外是不会产生这么露骨的氛围的。然而,可能是因为这里是周的家里,所以修斗就放松下来了吧。\\

长年不减的恩爱在儿子眼里算是表示夫妻和睦的好事,不过周还真是希望他们设身处地,为旁边看到这些场景的自己着想着想。\\

一旦变成那副模样,周也不想要进去打断了,于是就死了心坐到了餐厅拿来的椅子上,再次深深叹了一口气。

真昼也坐到了准备在旁边的椅子上,静静地看着周。\\

「……你父母关系真好啊」

「是啊。虽然在外面不是那个样子的,不过在家里就是那种感觉了」

「是吗」\\

周苦笑着回答之后,真昼眯缝起眼睛看向志保子和修斗。\\

真昼的表情并没有表示出不快,相反地,是如同看到耀眼的东西时那样的感觉。

她的眼神中渗透出憧憬和艳羡,就好像看到了什么崇高的东西一样。\\

看到真昼以虚幻渺茫的微笑望着两人,周情不自禁差点把手伸了过去——\\

「啊,周,怎么了嘛?」\\

听到了似乎已经回到了现实世界的志保子的声音,周立刻把手收了回去。\\

「怎么了个什么啦。还不是你们俩进入二人世界让我们待不下去了嘛」

「羡慕了?」

「没有没有,不存在的。我是觉得这种事情给我在家里做啦」\\

似乎两人并没有注意到周差点去握住了真昼的手。真昼似乎也同样是没注意到,正因周说的话而露出着苦笑。\\

周不知道自己为什么会把手伸出去。

只是,总觉得……不希望,让那样的真昼孤单一人。\\

真昼现在已经回到了平时的样子。周稍微变得放心了一些,为了不被察觉而回到了平时板着面孔的模样。\\

「所以,爸妈看到儿子的脸满意了么」

「看到真昼倒是挺满意了……」

「喂」

「有一半是开玩笑的啦。目的还没有完成呢」

「目的?」\\

周还以为志保子的目的是新年的走访和给真昼打个招呼,然而志保子似乎还有其他的目的。\\

「你们还没去新年参拜吧?」

「我准备人少一点之后再去的」

「对吧?小真昼也还没去吧。发的消息里是这么说的」

「是的」

「就猜到是这样,所以咱把和服拿来了哟~」\\

看来志保子是想和真昼去新年参拜的样子。

事到如今周终于明白了志保子满脸笑容提着一大包行李过来的理由,不知是今天第几次叹了口气。\\

志保子喜欢可爱的东西,也喜欢给人穿衣打扮,肯定是不想放过这次机会的吧。

和服的话,光是周知道的范围里家里就有几件。他们似乎是把这些给带过来了。\\

「我本来梦想就是给女儿穿上和服去新年参拜……小真昼的话我觉得肯定搭的」

「妈你就是想要个换衣服的洋娃娃吧」

「没有的事哦?不过很大原因倒是想让真昼穿上呢」\\

志保子「毕竟肯定很搭的」的自信满满的见解是正确的。

不如说,感觉没什么衣服会和真昼不搭。\\

在周所见的范围里,男性化的服装、大小姐那样高雅的打扮、还有平时带着饰边和蕾丝的很少女的服装,真昼都穿过几次,每一种都和真昼非常搭配。所谓美少女,大概是不择衣装的。

和服恐怕也会非常搭配吧。\\

藤宫家里周是独生子,所以想给女儿打扮的志保子似乎是无法放过这个机会。\\

「……嘛,要是真昼愿意的话,就让她穿上过去呗」

「为什么说得好像周不去一样?」

「要是和真昼出门让学校里那帮子人知道就不好了吧」\\

如果只是父母和真昼的话,就算去新年参拜看上去也是一家人一起去,不会有问题。

而如果带上了周就有问题了。\\

外表不显眼的周和真昼一起参拜,如果给同年级的同学见到了,可以想象寒假过去之后将是哀声一片的地狱场景。

再怎么说,周也不觉得承担这个风险依旧还想去新年参拜。\\

「不被发现就可以了吗?」

「可以是可以啦不过正常来说肯定会……我说妈啊,不会是」

「哼哼,就是为了这种时候才拿来了这么多东西的哦?」

「哪种时候啊!?」\\

和服、衬衣、小饰品,周就觉得如果只是这些和服相关的东西的话行李不会那么多,结果看来是为了欺负周而带来了更多的行李。\\

「修斗也很来劲的」

「爸……」

「难得的机会,不是挺好的吗。我是觉得,既然是年度活动,可以的话最好还是一起去吧」\\

被这么一说,周就难以拒绝了。

志保子的提议也包含了修斗重视家庭的意向,要是拒绝的话感觉有些不好意思。\\

「可是啊」

「没问题,相信妈吧。肯定会把周打扮得帅气到判若两人的」

「这是在说现在的我很挫吧」

「和修斗长得那么像当然是不挫的,不过发型和给人的感觉都是土里土气的啦。这种的是叫不阳光吧」

「吵死了」\\

周自己也知道自己土里土气,但周是自愿打扮成这样的,不希望别人一一指出。\\

「要是打扮好的话明明还挺能看的,就是周嫌麻烦……」

「多管闲事」

「真是可惜。……我说小真昼啊,你也想看周整理好的打扮吧?」

「诶?」\\

突然提到真昼的话题,这让真昼肉眼可见地惊慌失措着。

尽管周希望志保子不要那么对真昼步步紧逼,然而志保子却是毫不客气。\\

「周要是打扮好的话,我觉得真昼也应该会对周刮目相看的。别看周这样,其实长得还挺不错的哦?周虽然性格不坦率,但是遗传了修斗还挺绅士的,只要好好打理就真的是个好男人啦」

「呃,那个……是,是啊……?」

「不想一起去新年参拜吗?」

「想,想去是想去啦,可是」

「喂别出卖我啊」\\

不怕一万就怕万一,周是希望尽可能地拒绝的,而真昼却瞄了一眼吐槽的周。\\

「……周君不愿意的话,那就算了」\\

真昼发出了有些沮丧的声音微皱着眉头,让周突然感到一阵窒息。\\

真昼本人似乎没打算表现出来,然而她明显是一副遗憾的样子。这副样子似乎并不是故意彰显出的,而是自然流露出来的。

真昼静静摇着长睫毛朝下看的样子,让周产生了强烈的罪恶感。\\

志保子丢来了好像在说「让小真昼伤心」的指责般的目光,而修斗的视线则好像在说「放弃才更快一点」。在两道视线下,周发出了唔唔的小声。

这岂不就像是自己在欺负真昼一样了吗。\\

「……行吧」\\

看到那副表情,周不得不败下阵来。

\subsection{天使大人与新年参拜}

「好了,已经可以了」\\

周被志保子这也不是那也不是地摆弄头发、折腾脸部、搭配服装之后,到了总算解放的时候尽管没做什么却依然感到了疲劳。\\

周对衣着没有太大兴趣,所以这段时间很是痛苦。不过周照了照镜子,看到辛苦确实有了成效,镜子里映出的是平时的周无法比拟的端正样貌。\\

志保子选择的是深灰色的切斯特大衣、白色的高领衫、黑色的运动裤,这是简洁而又不那么休闲的搭配。

因为是要去新年时期可喜可贺的活动,志保子似乎注意有使衣服不要显得太轻便,目前的搭配给人一点微微的正式感。\\

周并不喜欢花花绿绿的衣服,这黑白而稳重的打扮也是符合周的喜好的。\\

周还确认了一下发型,偏长的前发经由剪子、打蜡和志保子的手艺巧妙地往旁边梳开,露出了平时常常藏在前发后面的眼睛。

将眼睛露出来,使得周给人的印象明朗了许多。不仅如此,头发也做成了更加厚实的造型,酝酿出优雅的气质。\\

被母亲和树嘲笑不阳光的周已经不在这里,站在镜子前面的是一个让人刮目相看的清爽男儿。\\

「明明稍微弄弄就是个好青年了,为啥就是不做呢」

「没有兴趣」

「周你老是这样。嘛,不过因为板着个脸,不笑的话也清爽不起来就是」\\

板着脸这句话是多管闲事,然而这是事实所以周也无法否定。\\

「那我给小真昼调整去了,你在客厅等着啊」\\

周是在自己房间搞了这些的,所以并不知道回自家换了趟衣服的真昼是什么样子。

真昼似乎是会自己穿和服,所以是先回一趟家里穿上再过来。从会自己穿和服这一点,就已经能窥见真昼的能干了。\\

周目送志保子离开房间之后,再一次看了看镜子里的自己。\\

由于很久没有打扮成这样过了,周觉得简直是不像自己一样。\\

「……嘛,应该不差吧」\\

尽管站在真昼旁边可能还显得有些寒酸,不过比起平时的周要好上几倍了吧。

稍稍摆弄着不再遮挡视线的前发,周小声地自言自语说,「偶尔这样或许也并不坏」。\\

\vspace{2\baselineskip}

周在客厅和修斗一起等了几十分钟之后,听到了家门打开的声音。

周有听说女人出门的准备需要花费大量的劳力和时间,所以对等待这事本身是没有不满的。然而,他很是担心真昼有没有被志保子性骚扰。\\

周迫不及待从沙发上站了起来往门口看了过去,只见真昼静静地走进了客厅。\\

看到真昼的第一眼,周就不禁出了神。\\

平时,真昼不会穿和服,周也没有看到的机会。周尽管觉得应该会很合适——但却没有想到竟然会这么合适。

由于穿着振袖和服在人流中难以行动,所以选择了小纹和服。淡粉色基调的梅花纹小纹和服非常合身,以至于让人怀疑这身衣服是不是原本就是真昼的。

真昼平时不怎么穿粉色衣服,而目前的打扮在典雅的感觉中还带着女人味。

淡色的长发只留一束在旁,其他部分都用发簪固定在上方。雪白的脖子和摇摆的装饰更加凸显出女性的感觉,非常美丽动人。\\

配合上衬托出原本的美丽的化妆,这一切将清秀美女的氛围体现到了极致。\\

「怎么样?我觉得还满可爱的。小真昼底子好,我这么花心思打扮真的是值了」

「嗯,相当好看」\\

听到修斗直率地笑着这么夸奖道,真昼也有些难为情地把头低了下来。连这个动作都那么迷人,美人还真是可怕啊。\\

「喂,周,不好好说出感想可不行啊」

「我觉得挺好的吧」\\

再怎么说周也没法在父母面前对真昼赞不绝口,于是就送上了不痛不痒的称赞,不过志保子好像非常不满意的样子。\\

「……周就是这种地方不好哦?」

「吵死了」\\

尽管周被志保子批评了,但他并不打算在父母面前说出更多的夸奖,所以把脸朝向了别的地方。

志保子尽管对周感到有些无奈,然而她似乎是很了解周的性格,所以叹了口气就把周放过了。\\

「真是的……话说,小真昼,你觉得怎么样?周这样简直就是变了个人吧?」

「嗯,是的。和平时完全……」

「平时要是打扮成这样肯定能受欢迎的,可他就是不干,真是亏啊」\\

对周来说这是多管闲事,但志保子正叹着气,好像是真的觉得可惜一样。\\

「明明长得和修斗这么像了,结果还不好好利用,真是让人失望啊。太可惜了~」

「嘛嘛志保子。周也是这个年纪了嘛」

「那不是应该更想要受欢迎吗?」

「要说的话周是那种只要有一个人就好的性格,会觉得其他人都很烦吧」

「哎呀」\\

修斗原本打着圆场,却反而给志保子的妄想点上了一把火。

确实,周比起被众人喜欢,更希望只有一个人在旁边……修斗是这么对志保子说的,实际上周也很赞同,然而在现在的情况下,说得不就好像真昼就是那个人一样吗。\\

在志保子灿烂的笑容下,周把抽着筋的脸转了过去。\\

尽管周很不满志保子的胡思乱想,但他也知道事实上在其他人看来就是这样的。

至少,周可以断言说真昼对他而言是特别的。\\

虽然这是事实——\\

周不让真昼发现地偷偷看了真昼一眼之后,轻轻叹了口气。\\

(要说喜欢的话,那肯定是喜欢的)\\

周确实觉得真昼很讨人喜欢。

不过,要断言这是恋爱感情的话,周觉得还是有区别的。\\

「妈你想的事情都是不存在的。别说这么多废话了,赶紧去准备开车吧」

「真是没劲……算了算了,修斗,去准备开车吧」

「是啊」\\

周似乎成功地转移了话题,两个人都开始做起了出门的准备。

周将去哪个神社的选择交给爸妈,目送着两人出门前往停车场的背影。\\

「……要带的东西我都在包里了,没什么要准备的。真昼呢?」

「嗯,都在这个包里」

「这样」\\

突然就只剩下了两个人,周感到有些坐立不安的同时,检查了家里窗户有没有锁好,并且拔下了多余的电器的插座。\\

周关掉客厅的灯,再次看向真昼。

果然,就算不仔细看,周也依然觉得真昼很漂亮。周在父母跟前没有能尽情称赞,然而无论让谁来评判都无疑是和服美人的真昼实在是非常养眼。\\

「怎么了吗,周君」

「嗯,就是觉得这和服真适合你啊。就是那种清秀的和服美人的感觉,挺可爱的」\\

女性打扮的时候应该称赞,这件事周本来就从修斗那儿学到了。周应该在看到之后立刻称赞的,不过在父母眼前称赞实在是难为情,所以便作罢了。\\

周说出坦率的感想之后,真昼连着眨了几次眼,接着微微染红了脸抿紧着嘴唇。

想起之前真昼也是这个反应,周露出了小小的苦笑。\\

「啊,你是不喜欢被夸是吧?抱歉」

「不,不是的,不过……周君,还挺」

「还挺?」

「……没什么」\\

看着真昼扭开了脸,周虽然不明不白的,但是看真昼也没有说出口的打算,便只好老实放弃,和真昼一起走向门口。

考虑到走路,真昼穿的不是木屐而是长筒靴,是和洋折衷的风格。即便如此,大概也能看到她可爱的样子吧。\\

真昼叮铃铃地摇着发簪的装饰,同时穿上了长筒靴,接着,静静走向了提前一步到外面去顶住门的周。

两人的距离,比想象中要更近。很少见的,这次是真昼往周那儿接近之后,轻轻地踮着脚尖。\\

这意思是让我听她说话吗。周做出了这样的判断,于是把门锁上后弯下了腰。接着,真昼将手形成环状捂在嘴前,靠近了周的耳边。\\

「周君」

「嗯?」

「那个……很帅,哦?」\\

小声耳语了短短一句之后,真昼穿过了周的身侧,快步走向了电梯间里面。而周就这么把头嗵地压到了门上。\\

「……太狡猾了」\\

刚才那句宛若回击一样的耳语,让周的心好像揣了只兔子一样怦怦直跳。\\

因为真昼的事情,周花了好一会儿才让一下子变得火辣辣的脸颊冷却下来,结果被提前等在停车场的父母投以了怀疑的视线。


\end{document}
