\subsection{名为赠物的天降之恩}

连跑便利店都觉得麻烦的周,一边吸着家里常备的果冻饮料,一边靠在栏杆上呼吸着屋外的空气,结果却恰好遇上了走到阳台上的真昼。

看见周样子的真昼跟周一样把头稍微探出栏杆,然后注意到了周正吸着的果冻饮料,稍稍皱了皱眉。\\

周完全没想到自己会被搭话,呆呆地愣住了一会儿。\\

「看了就明白吧。花不了一分钟便能补充能量的果冻——」

「……你晚饭不会就是这个吧?」

「那还能是啥」

「……明明是个食欲旺盛的男高中生就吃这点?」

「多管闲事」\\

平常的话周是靠着便利店便当就着点配菜过活的,不至于就这么简单。不过今天懒得去弄晚饭,又没心情吃杯面,便靠着果冻饮料应付过去。

估摸着这点量也是不够,等会可能还要来点零食补一下量。\\

「……不做饭么」

「不做也不会做。你不也是知道的么」

「……而且还不会打理卫生真亏你能一个人活下来呢……」

「啰嗦。跟你没关系吧」\\

被戳到痛处的周,皱着眉头把剩下的果冻饮料吸完了。\\

扫除什么的前几天已经吃了亏了本就打算处理一下的。天天说来说去反而搞的人不想干了。\\

周反而非常好奇真昼为何总是这么啰啰嗦嗦的。然而真昼盯着周,接着轻轻叹了一口气。\\

「……请稍等一会」\\

话音刚落,真昼就从阳台走回了房间。\\

听着阳台关上窗户的声音,周小声嘟哝了一句「到底怎么回事」。

光说让人等着,是要等什么啊。\\

周疑惑地看向真昼的家里,理所当然的没有回应。\\

(差不多也凉下来了,我想要回屋啊)\\

虽说也照着对方说的正在等着,但秋天的晚上比预想的还要冷。一件汗衫实在是不够。\\

不如说周也不知道为什么自己会就这么乖乖地等着。\\

外面的眼看就要降到呼吸会有白雾的气温。周长吐了一口气,这时从玄关传来了一阵电子音。

听见这来客的门铃声,周扭过了头。\\

心里有数的来客只有一位。\\

周实在不知她为何会来,避开散乱一地的衣服和杂志,走到了玄关。

即便不看猫眼,也知道来的是谁。周用脚把拖鞋撩到门口,解开防盗栓打开门——不出所料,比周眼睛稍低处,有一丛亚麻色的头发。\\

「……你干啥呢」

「你过的太不像样子了,我都看不下去了。……虽然是剩下的,你拿去吧」\\

真昼语气冷淡地说着,把手往前伸了过来。

比周小上一圈的娇小的手,端着一个特百惠的饭盒。半透明的盒盖模糊地透出里边煮食的影子。\\

或许里面的东西还温着,盖子上起了一层水雾,虽然看不清,但里面是煮食没错吧。\\

周连着眨了几次眼睛。真昼似是理解了眼神里询问原因的意思,深深地叹了口气。\\

「还不是因为你不好好吃饭。补品只是辅助,当不了主食」

「你是我妈么」

「我自认为自己的主张算是很普通的。另外你房间改收拾收拾了吧?现在这样落脚都难」\\

真昼瞄了一眼周的身后,一副受不了的样子明显地眯起了眼睛,让周无言以对。\\

「……走还是能走的嘛」

「根本没有啊。一般衣服就不该掉在地上」

「衣服是会掉的」

「洗好晾干叠好收起来就不会掉了。读完了的杂志也请打包收拾好。踩到滑倒了可就是大问题了」\\

虽说话里略微带刺,但周明白真昼不知为何是纯粹在关心他,自然也不能一一回嘴。\\

确实上次来照看他那次,两人就差点因为房间太乱而摔倒了。被说也是理所当然。

一脸苦涩回不了嘴的周,只好从抿着嘴的真昼那接过饭盒。

慢慢扩散到手掌上的温度,在这渐渐转凉的天气里,让人很是心暖。\\

「……那,我可以吃这个么」

「你不要的话我只好倒掉了」

「别别我吃我吃。天使大人亲自做的晚餐一般可是吃不到的啊」

「……能别那么叫我么,真的」\\

抱着报复的念头试着用了下学校里的外号,结果真昼的脸易懂地染上了红晕。\\

对本人来说这个外号也许是太羞耻了。站在她的角度,周也肯定会觉得不舒服,这倒是理所当然的。\\

看见脸上泛起红晕,甚至还有点哭相地瞪着自己的真昼,周不禁笑了出来。\\

「抱歉,以后不这么叫了」\\

再这么叫估计就真的要坏对方心情了,周也不想玩到过分。再说双方的关系也没好到那种能随便开玩笑的程度,过分了也不太好吧。\\

真昼似乎也不想再被这么叫了,清了清嗓子,表示自己重新振作了一下精神。

然而她脸上微微泛红,看上去和刚才并没有多大区别。\\

「嘛,那我就满怀感激的收下了。你也别再想着那事过意不去了」

「那倒没,反正照顾了下生病的你也算相抵了。这个只是我的自我满足……不如说是你过这样的废人生活我实在是看不下去而已」

「是是是」\\

被看见的时候都是一副邋遢样子,她这么看自己某种意义上来说也是理所当然的。

就连现在周身后的走廊也是乱七八糟的,而且照看他那时已经被看了个光,事到如今想瞒也瞒不住了。\\

「……要好好吃饭,规律作息哦?」

「真要当我妈啊」\\

看着一本正经说着的真昼,周一脸疲倦地吐槽道。\\

\vspace{2\baselineskip}

端着分到的赠物回到家里的周,摸出一双超市拿的一次性筷子,坐在了客厅的沙发上。\\

虽说是真昼强塞过来的,但味道究竟如何呢。

给煮的粥倒是很好吃。虽说感冒舌头有些不灵敏,但那个认真从生米煮出来的粥,味道是一点一点温和地渗透到胃里。\\

可以推测真昼的料理手腕应该不错,那实际上是怎样的呢。\\

怀着几分期待,又有几分犹豫的周打开了盒盖。淡淡飘出的无疑是煮菜的香味。

这是几种根菜和鸡肉煮成的。汤的颜色略淡,清楚地映出了鲜艳的胡萝卜的颜色以及旁边点缀的扁豆。\\

全部切成一口吃下的尺寸的各色食物,强烈地刺激着只吃了点果冻饮料的周的食欲。\\

周迅速掰开一次性筷子,首先夹起了一块萝卜。\\

「好吃」\\

口味如何,迅速见了分晓。\\

味道清淡,有高汤的风味,十分有注重健康的真昼的风格。而且还不是买来的那种颗粒调味料,而是拿鲣鱼和海带认真煮出来的汤吧。口味完全不同。\\

周细细地咀嚼,享受着在嘴里渐渐散开的高汤、调味料、以及蔬菜本身的味道。

即便是不爱吃蔬菜的周,面对这不但充分发挥了蔬菜本来鲜味,还彻底入了味的煮菜,也可以大快朵颐。\\

如同说着要好好吃蔬菜一般的少量鸡肉,吃起来也十分鲜嫩,毫无干柴感。除了量以外无可挑剔。\\

作为女高中生的料理,菜式的选择很朴素,但完全体现出了制作者的水准。

可以说,这味道和刚刚学会做菜的人做出的东西有天和地的差别。\\

要是再来点饭啊味噌啊酱油啊啥的那就更棒了——虽然这么想,但是不巧并没有煮饭……不如说家里的米甚至都用完了,这点小愿望也无法实现。

虽说事到如今,但周还是后悔着。早知道就去买两包速食饭包回来了。\\

「天使还真是厉害啊」\\

周用着这本人听了怕是又要不高兴的叫法,称赞着学习运动加上家务样样全能的真昼,一刻不停地享受着这味道理想的煮根菜。
