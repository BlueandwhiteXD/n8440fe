\subsection{天使大人与生日}

向树和千岁寻求完建议后,总算选好了礼物的周,在生日当天以一副易懂的紧张表情看着真昼的身后。\\

周以车站前的可丽饼屋卖的特制可丽饼(冬日限定莓类特辑)为报酬,总算是说动了千岁帮自己忙挑了个礼物……可现在周却在苦恼该什么时候把这礼物送出去。\\

但那过生日的本人,却在一如既往地做着晚饭。\\

虽然不清楚菜单,但看上去是在做和食的样子,怎么看都没有什么特别的感觉。完全就是一副平常心的样子。

从当事人身上完全感受不到生日的氛围。不如说那淡定程度,简直让人觉得她是不是根本就不记得这回事。\\

甚至到了晚饭做好,两人上了餐桌吃饭时的对话也是一如往常。\\

真的拿不定注意该什么时候把东西给真昼的周,看向藏在沙发后面放着礼物的纸袋,皱起了眉。\\

过了一会,周收拾好餐桌回到客厅的时候,真昼正坐在那刚好两人位的沙发上看着似乎是自己带来的书。

就连看书的模样也美如画作,实在是不虚天使之名。\\

虽说不知为何周对坐在真昼旁边有些微妙的犹豫……但退缩也不是个办法,周便提起放在一旁的纸袋,坐在了真昼旁边。\\

周突然抬起了头。

大概是注意到了周的气息和纸袋擦过的声音,真昼那焦糖色的双眼看向了周,然后又移向了纸袋。

依旧一脸不解的表情的真昼,看来是到了这个地步还没有注意到自己生日的事情。\\

「嗯,给你的」\\

周半是强塞地把纸袋放在了真昼膝上,使真昼连上更增添了几分疑惑。\\

「这是什么」

「今天不是你的生日吗」

「是倒是……话说为什么你会知道啊。我不记得我有跟谁说过这回事啊」\\

真昼的眼里微微露出警戒的态度,但听到周“你上次把学生证落屋子里了吧”的解释明白了一般,恢复了平时的表情。\\

「其实,不必要在意就好。反正我也不过生日的」\\

那冷淡而透出排斥感的声音,自己决没有听错——

当周望见真昼那如同对生日这词汇本身便不知为何抱着忌讳感的眼神时,他如此确信。\\

原来如此——周想到。\\

明明是生日,她的态度却毫无变化,其原因,并非是忘记了生日的事情。

因为生日曾经很烦人,所以忘了。

若非如此,她也不会用那种语调吧。\\

「啊这样啊。那就当作是平常受你照顾的回礼吧。权当我一厢情愿想要报恩」\\


