% 35 天使様とお正月の予定


%  クリスマスを過ぎれば、世の中は年末ムード一色になる。\\


%  夜景のためのイルミネーションこそ残されているが、あれだけ飾られていたクリスマスツリーはすでに撤去され、目に鮮やかな飾りつけは和のものに変わっている。


%  売り出しているものも全面的にお正月の飾りや食材になり、もう聖夜の面影は残っていなかった。\\


%  変わり身が早いものだな、とすっかり年越し準備に入りつつある周囲を眺めながら、周はマフラーに顔を埋めて温もりをとる。\\


%  モノトーンの千鳥柄のマフラーは、真昼にクリスマスプレゼントとしてもらったものだ。


%  なんでも首もとのおしゃれも大切です、との事で、非常に手触りがよくしっかりと風を遮って熱を溜めてくれる実用性と装飾性を兼ね備えた一品をいただいた。\\


%  普段マフラーなんてしなかったのでありがたく使わせてもらいつつ、腕に提げた買い物袋の中身を確認する。\\


%  基本的に買い出しは分担という事だが、料理を作る真昼の負担を減らすために基本は周がメモを携えて買い揃えている。


%  今日は寒いので鍋にするらしく、野菜やらきのこやら肉が袋に納められていた。野菜多めなのは、しっかり栄養をとりなさいという真昼の無言の主張だろう。\\


%  足りないものはないな、と改めて確認し、やはり厳しくなりつつある寒さに身を震わせつつ足早に帰宅した。\\


% 「お帰りなさい」\\


%  家に帰れば、夕方だったために真昼が出迎えてくれた。


%  赤の他人が家主を迎え入れるというちょっとおかしな事態ではあるが、最近は慣れつつあった。\\


% 「ん、ただいま。……薄切りの餅買ってきちまったけどいいか?」


% 「鍋でしゃぶしゃぶしたいのですね」


% 「おう。あと〆にラーメン買ってきた」


% 「……私、そんなに食べられませんよ?」


% 「俺が大半食うから関係ないな」\\


%  前はそう食べるタイプではなかったのだが、真昼の料理のお陰で晩ごはんは割と食べている。\\


%  彼女もカロリーに気を付けているのか食事は太らない程度のものであるが、彼女より量を食べる身としては微妙に心配なので筋トレしたりし始めている。\\


%  真昼としては、周は細いからもう少し肉をつけるべきでは? といった感想らしいので、なるべく脂肪ではなく筋肉をつけたいところだった。\\


% 「まあ、周くんが食べてくれるならいいですけど。それ、貸してください。冷蔵庫に入れてきますから。周くんは手洗いうがい」


% 「分かってますよっと」\\


%  真昼に荷物の入ったレジ袋を渡して、周は素直に洗面所に向かった。\\


\vspace{2\baselineskip}

% 「そういえば真昼は正月どうするんだ」\\


%  本日も相変わらず非常に美味な晩ご飯を平らげ後片付けをした所で、ふと気になった事を真昼に聞いてみる。\\


% 「正月……帰っても無駄ですしここに居ますよ」\\


%  あまりに淡々とした口調で返されて自分の失敗を悟ったものの、真昼はさして気にした様子もなさそうだ。


%  親との折り合いがよくないために、どうしても家族関係の話題はそっけない態度になっているのだろう。\\


%  ただ、そうなると真昼は一人で正月を過ごす事にならないだろうか。


%  周は基本的に半年に一度は顔を出す事、という約束があるため、真昼と出会う前は長期休暇は実家に帰るつもりでいたのだが。\\


% 「周くんは実家に帰るのですよね」


% 「そうだな、一応顔を見せろとは言われてるんだが」\\


%  ちら、と真昼を見ると、いつもの表情より心なしかひんやりとした眼差しの真昼が居る。\\


%  一人で過ごす事を当たり前だと思っているらしく、別に周が帰省する事を疑っていない。\\


% 「……帰ったらお前の事をしつこく聞かれそうでなあ」


% 「大変ですね」


% 「父さんは多分母さんの話聞いてそっかー程度で済ませるだろうけど、母さんは多分話聞きたがるからな」


% 「しょっちゅうやりとりしてるのに不思議ですね」


% 「ほんとお前いつの間にか母さんと馴染んでるよな……」\\


%  何故か母親といつの間にか仲良くなられた挙げ句知らない間に写真やら裏話が流出しているのか……とちょっと虚しくなるのだが、真昼もこの調子だと割と好きで相手しているようで、それならまあいいかという気持ちにもなる。


%  志保子にはまた余計な事を言うなよ、と釘を刺しておくとして、どうしたものかと真昼を見る。\\


%  時折見せる虚ろな表情や、寂しげな眼差しを思い出すと、どうしても……一人にしたくない。\\


% 「まあ、この間母さんとは会ったし、父さんには悪いけど今回は帰省しなくてもいいかなと。どうせ春休みに帰るし」\\


%  だから、彼女が迷惑でないのなら、いつも通りに夕食を共に出来れば、と思うのだ。\\


% 「……そうですか」


% 「ん。お前の年越し蕎麦食べたいし」


% 「食い意地はってますねえ」


% 「真昼の料理だからなあ」


% 「……ほぼ市販品なのに?」


% 「それでもだな」\\


%  たとえそばが市販のものを茹でただけでも、いいのだ。


%  二人でゆっくりと食べて時を過ごす、という事の方が重要なのだから。\\


% 「……変な人ですね」


% 「うっせ」\\


%  失礼な感想を述べてきた真昼にわざとらしく不機嫌そうに返してみれば、小さな微笑みが返ってきた。\\


% 「……ありがとうございます」


% 「何がだよ」


% 「何でも、です」\\


%  真昼はそれ以上は何も言わず、幾分機嫌がよくなったのか明るい表情を浮かべて、お気に入りのクッションを抱き締めた。


