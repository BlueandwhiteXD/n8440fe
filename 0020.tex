\subsection{天使大人与生日}

向树和千岁寻求完建议后,总算选好了礼物的周,在生日当天以一副易懂的紧张表情看着真昼的身后。\\

周以车站前的可丽饼屋卖的特制可丽饼(冬日限定莓类特辑)为报酬,说动了千岁帮自己忙买了个东西,并把这东西也加进了礼物……可现在周却在苦恼该什么时候把这礼物送出去。\\

而那过生日的本人,正一如既往地做着晚饭。\\

虽然不清楚菜单,但看上去是在做和食的样子,怎么看都没有什么特别的感觉。完全就是一副平常心的样子。

从当事人身上完全感受不到生日的氛围。不如说那淡定程度,简直让人觉得她是不是根本就不记得这回事。\\

甚至到了晚饭做好,两人上了餐桌吃饭时的对话也是一如往常。\\

真的拿不定主意该什么时候把东西给真昼的周,看向藏在沙发后面放着礼物的纸袋,皱起了眉。\\

过了一会,周收拾好餐桌回到客厅的时候,真昼正坐在那刚好两人位的沙发上看着似乎是自己带来的书。

就连看书的模样也美如画作,实在是不虚天使之名。\\

虽说不知为何周对坐在真昼旁边有些微妙的犹豫……但退缩也不是个办法,周便提起放在那里的纸袋,坐在了真昼旁边。\\

真昼突然抬起了头。

大概是注意到了周的气息和纸袋摩擦的声音,真昼那焦糖色的双眼看向了周,然后又移向了周拿着的纸袋。

依旧一脸不解的表情的真昼,看来是到了这个地步还没有注意到自己生日的事情。\\

「嗯,给你的」\\

周把纸袋推出去放在了真昼膝上,使真昼变得一脸茫然。\\

「这是什么」

「今天不是你的生日吗」

「是倒是……话说为什么你会知道啊。我不记得我有跟谁说过这回事啊」\\

真昼的眼里微微露出警戒的态度,但听到周「你上次把学生证落屋子里了吧」的解释,或许是明白了便恢复了平时的表情。\\

「其实,没必要在意的。反正我也不过生日」\\

那冷淡而透出排斥感的声音,应该不是自己听错了吧。

当周望见真昼那如同对生日这词汇本身便不知为何抱着忌讳感的眼神时,他如此确信。\\

原来如此——周想到。\\

明明是生日,她的态度却毫无变化,其原因,并非是不记得生日的事情。

因为生日很烦人所以故意忘掉的——应该说是这么回事。

若非如此,她也不会用那种语调吧。\\

「啊这样啊。那就当作是平常受你照顾的回礼吧。权当我一厢情愿想要报恩」\\

平日来的回礼是另一回事所以不过生日也没差,这就当作我表达的感谢而不是生日礼物,周这么解释,把礼物塞了过去。

每天都吃着这么好吃的饭,偶尔还来帮忙打扫屋子,虽然都是小事,但也实在是受照顾了。即便只是慢慢来,周也想要回报真昼。\\

周虽然很轻易地就退让了,却还唯独要把礼物送过来,这让真昼有些混乱,但真昼还是对着打算把礼物送上的周困扰地微微皱着眉接过了礼物。

真昼的视线,集中在纸袋里面的,用另一层袋子包装的东西上面。\\

「我可以现在打开吗?」

「嗯」\\

看见周点头,真昼紧张地从纸袋里把箱子拿出来,小心地打开包装纸解开缎带。

怎么说呢,看着礼物在自己面前慢慢被打开格外地令人紧张。\\

里面放着的是树推荐的护手霜。因为是套装一起卖的,这个大盒子里还附带着一点小点心。\\

顺带一提这并不是那种带有香味的时尚品,而是以没有香味、适合家务、亲和肌肤,滋润保湿为卖点的东西。

周也确认过网上的评价,效果应该是不用担心的。\\

「嘛抱歉,不是什么值钱东西。看你干家务手应该会干吧。虽然也有那种有香味的,不过你估计有了所以没买。这种的对皮肤挺好,似乎挺有效的」

「实用品呢」

「你的话要选肯定是选实用的吧」

「是呢。谢谢你了」\\

看着微微露出笑容,似是在说你还挺了解我的嘛的真昼,周也稍稍放松了嘴角。

看来印象不坏。\\

之后虽然还有一件东西……但要当面打开周还是觉得有些害羞,因而可以的话周还是想真昼回到家再发现那个东西。\\

可事不如愿,在把护手霜放回纸袋里的时候,真昼似乎是注意到了纸袋里还有一件东西,眨了眨眼。\\

「……是还有一件东西吗?」

「啊—。呃,那个,怎么说。就是个充满独断和偏见的附赠」

「附赠?」

「……啊,附赠」\\

周撇开视线生硬地回答后,歪着头搞不明白意思的真昼觉得不如直接拿出来看来得快,便从纸袋里把那东西拿了出来。\\

周为了让那东西尽可能的不起眼而用了跟纸袋一个颜色的包装,塞在了最底下,但果然这个大小还是很起眼。不如说真亏得能在打开护手霜的盒子之前都没被发现。\\

那东西的包装并非盒子,而是聚酯塑料袋。其大小,正好够真昼双手抱住。

看着真昼把那深蓝色的丝带慢慢解开的周想着自己能不能站起来离开的时候,真昼正好把里面的东西取了出来。\\

两只手小心地把里面的东西取出来的真昼,看起来是真的吃了一惊,眨巴着她那两颗大眼珠。\\

「……熊?」\\

真昼说着的,便是那东西的原型。\\

不算太大,大概小学生抱着大小正适合的,布偶。

布偶那近于真昼头发的淡色的软毛是其特征,而那附在不知为何透出一股天真气息的脸上的乌黑圆润的眼珠里映着真昼的身姿。\\

都高中生了还玩偶啊——说不定会被这么想。\\

尽管如此,听了千岁「女孩子可是不论什么时候都喜欢可爱的东西的」这样的建议后,周选择了这个。\\

再怎么说男的一个人跑去买这东西实在是太令人害羞了,周便以车站前的可丽饼为报酬让千岁陪着自己去买了。

结果从挑选到打包周一直在被千岁笑嘻嘻地看着,说不定其实一个人去买羞耻感还会少一点。\\

「……觉得女孩子会喜欢这个吧所以」\\

周挠着头,不知是在跟谁解释般嘟哝到。\\

这种事周实在是不擅长。

不如说给异性送礼自从小时候送妈妈的以外就没有干过,周甚至没有料到自己能做到这种事情。\\

从男的那里收到这么可爱的玩偶会不会让她受不了啊……周偷偷瞄了眼真昼,而真昼则紧紧盯着熊的脸不放。

也不知是高兴还是不高兴,真昼只是呆呆地望着布偶熊。\\

「嘛,不喜欢的话扔了也行」\\

如果不喜欢的话那也没办法,周这么想着,玩笑般地说了一句,结果真昼却刷地皱着眉把头扭了过来。\\

「那种事不会做的!」

「啊,呃。看椎名的性格我想应该也不会的」\\

真昼的否定比预想要强烈,令周不禁退缩地点了点头。真昼则再次看向手中的熊布偶。\\

「……我不会做,那么过分的事情的。会好好珍惜」\\

真昼纤细的手腕,像是要将其拥入怀中般,紧紧抱着熊布偶。

那姿态看上去,像是不愿喜欢的玩具被拿走的孩子,又像是慈爱地抱着孩子的母亲。

一句话来形容,便是极为珍重地抱着布偶。\\

以感觉快要蹦出啾~的音效的抱法抱着的真昼,稍稍垂下眼帘向下看着怀里的布偶。\\

而那脸上的表情,既不是平常的那种冷淡的表情,也不是被周的脱线惊呆时的表情,而是心安而柔和,泛着慈爱的,爱惜的表情。

如此这般,甚至还泛着天真感的纯洁的微笑,其美丽以及可爱程度,令周不禁屏息。\\

(——不该看的)

望见这样的表情,周不由自主地有了反应。\\

就算没有恋爱意义上的喜欢,单纯是让数一数二的美少女露出了这样的表情,还被自己看见了这一事实,也足以让周心跳加速了。\\

那珍惜地抱着布偶,露出淡淡微笑的姿态,其可爱已经到了恐怕不论谁看见了都会看入迷的程度吧。就算是自认为无欲无求的周也被迷住了。\\

为了确认自己脸上积蓄了多少热度,周伸手按了下自己的脸,手上则传来了比平时更加明显的热感。

想着自己害羞起来太好懂了的周,以真昼听不到的声音「……靠」地骂了一句。\\

幸好,真昼正紧紧抱着熊布偶,把半张脸埋进里面,因而并未注意到周。

看着那可爱的样子,周好不容易才忍住了发出怪声的冲动。\\

「……这么喜欢的话倒也正合我意」\\

周想着说些什么挤出了这么一句,真昼则瞄了这边几眼。\\

「……我是第一次,收到这种东西」

「唉,以你的人气这算是日常贡品吧……」

「你把我当成什么了……」\\

这带有稍许无奈的声音与表情,反而让周安心了下来。大概是因为不再需要直视那样的表情了吧。\\

「……我没有告诉过别人过我的生日。因为不喜欢生日,所以从来都不说的」\\

不喜欢——如此断言的真昼的视线移向了布偶熊。

真昼看着布偶的眼神与嘴中的话语截然相反,充满了安心感,不知为何却让周觉得不大自在。\\

「一般,不认识的人或者没什么关系的人送我礼物我也觉得可怕所以不会收」

「我送的倒是收了啊」

「……藤宫同学又不是不认识的人」\\

真昼小声地回答着,然后把脸埋进布偶里仰头看向自己。周后悔直视她了。\\

无意中向上看着自己的真昼那放松下来的,流露出的与年龄相应的天真表情,实话说,相当令人怜爱。

那可爱令人不自觉地产生了想要摸摸头的冲动,周慌忙用力收回了自己不经意间便伸向了真昼的头的手。\\

「……怎么了吗」

「没,没啥」\\

不知是注意到了周一瞬间动了的手,还是察觉到了周那几近爆发的心痒感,真昼咚地歪了歪头。

仅是这样便差点抓住了周的眼球,美少女这种生物还真是可怕。\\

但是再怎么说,直接回答因为可爱所以看呆了,周还是会感觉羞耻,而且就算说了周也确信真昼只可能回答「哈?」。\\

再说,如果那样的话周在各种意义上都会死亡,因而决定还是把这个冲动深藏于心。\\

「……谢谢你,藤宫同学」\\

真昼小声说出的话,再一次传进了撇开脸的周的耳里。

%************************************************
% 后记:※娇化输出 20%
